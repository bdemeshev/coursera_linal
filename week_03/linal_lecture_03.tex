\documentclass[14 pt,xcolor=dvipsnames]{beamer}

% !TEX root = linal_lecture_01.tex

\usepackage{epsdice}

\usepackage[absolute,overlay]{textpos}

\usepackage[orientation=portrait,size=custom,width=25.4,height=19.05]{beamerposter}

%25,4 см 19,05 см размеры слайда в powerpoint

\usetheme{metropolis}
\metroset{
  %progressbar=none,
  numbering=none,
  subsectionpage=progressbar,
  block=fill
}

%\usecolortheme{seahorse}

\usepackage{fontspec}
\usepackage{polyglossia}
\setmainlanguage{russian}


\usepackage{fontawesome5} % removed [fixed]
\setmainfont[Ligatures=TeX]{Myriad Pro}
\setsansfont{Myriad Pro}

% why do we need \newfontfamily:
% http://tex.stackexchange.com/questions/91507/
\newfontfamily{\cyrillicfonttt}{Myriad Pro}
\newfontfamily{\cyrillicfont}{Myriad Pro}
%\newfontfamily{\cyrillicfontbs}{Myriad Pro}
\newfontfamily{\cyrillicfontsf}{Myriad Pro}


% https://tex.stackexchange.com/questions/175860/why-does-unicode-math-break-the-kerning-of-accents-in-combination-with-amssymb
% "You shouldn't be using amssymb together with unicode-math"
\usepackage{amsmath,amsxtra,amsthm} % amssymb

%\usepackage{bm}

\usepackage{fdsymbol} % \nperp

\usepackage{unicode-math}


\usepackage{centernot}

\usepackage{graphicx}
\graphicspath{{img/}}

\usepackage{wrapfig}
\usepackage{animate}
\usepackage{tikz}
%\usetikzlibrary{shapes.geometric,patterns,positioning,matrix,calc,arrows,shapes,fit,decorations,decorations.pathmorphing}
\usepackage{pifont}
\usepackage{comment}
\usepackage[font=small,labelfont=bf]{caption}
\captionsetup[figure]{labelformat=empty}
\includecomment{techno}

\usefonttheme[onlymath]{serif}


%Расположение

\setbeamersize{text margin left=15 mm,text margin right=5mm} 
\setlength{\leftmargini}{38 pt}

%\usepackage{showframe}
%\usepackage{enumitem}
%\setlist{leftmargin=5.5mm}


%Цвета от дирекции

\definecolor{dirblack}{RGB}{58, 58, 58}
\definecolor{dirwhite}{RGB}{245, 245, 245}
\definecolor{dirred}{RGB}{149, 55, 53}
\definecolor{dirblue}{RGB}{0, 90, 171}
\definecolor{dirorange}{RGB}{235, 143, 76}
\definecolor{dirlightblue}{RGB}{75, 172, 198}
\definecolor{dirgreen}{RGB}{155, 187, 89}
\definecolor{dircomment}{RGB}{128, 100, 162}

\setbeamercolor{title separator}{bg=dirlightblue!50, fg=dirblue}

%Цвета блоков

% Голубой блок!
\setbeamercolor{block title}{bg=dirblue!30,fg=dirblack}
\setbeamercolor{block title example}{bg=dirlightblue!50,fg=dirblack}
\setbeamercolor{block body example}{bg=dirlightblue!20,fg=dirblack}

\AtBeginEnvironment{exampleblock}{\setbeamercolor{itemize item}{fg=dirblack}}
%\setbeamertemplate{blocks}[rounded][shadow]

% Набор команд для удобства верстки

\newcommand{\RR}{\mathbb{R}}
\newcommand{\ZZ}{\mathbb{Z}}
\newcommand{\la}{\lambda}

% Набор команд для структуризации

%\newcommand{\quest}{\faQuestionCircleO}
%\faPencilSquareO \faPuzzlePiece \faQuestionCircleO  \faIcon*[regular]{file} {\textcolor{dirblue}
%\newcommand{\quest}{\textcolor{dirblue}{\boxed{\textbf{?}}}
\newcommand{\task}{\faIcon{tasks}}
\newcommand{\exmpl}{\faPuzzlePiece}
\newcommand{\dfn}{\faIcon{pen-square}}
\newcommand{\quest}{\textcolor{dirblue}{\faQuestionCircle[regular]}}
\newcommand{\acc}[1]{\textcolor{dirred}{#1}}
\newcommand{\accm}[1]{\textcolor{dirred}{#1}}
\newcommand{\acct}[1]{\textcolor{dirblue}{#1}}
\newcommand{\acctm}[1]{\textcolor{dirblue}{#1}}
\newcommand{\accex}[1]{\textcolor{dirblack}{\bf #1}}
\newcommand{\accexm}[1]{\textcolor{dirblack}{ \mathbf{#1}}}
\newcommand{\acclp}[1]{\textcolor{dirorange}{\it #1}}
\newcommand{\todo}[1]{\textcolor{dircomment}{\bf #1}}
\newcommand{\graylink}[1]{{\fontsize{11}{12}\selectfont \textcolor{gray}{#1}}}
\newcommand{\figcaption}[1]{{\fontsize{18}{20}\selectfont #1}}


\newcommand{\videotitle}[1]{
    {\fontsize{33}{30}\selectfont \textcolor{dirblue}{\textbf{#1}} }

    %\todo{название видеофрагмента}
}

\newcommand{\lecturetitle}[1]{
  {\fontsize{33}{30}\selectfont \textcolor{dirblue}{\textbf{#1}} }

    %\todo{название лекции}
}





\newcommand{\spcbig}{\vspace{-10 pt}}
\newcommand{\spcsmall}{\vspace{-5 pt}}

%\usepackage{listings}
%\lstset{
%xleftmargin=0 pt,
%  basicstyle=\small, 
%  language=Python,
  %tabsize = 2,
%  backgroundcolor=\color{mc!20!white}
%}



%\newcommand{\mypart}[1]{\begin{frame}[standout]{\huge #1}\end{frame}}

\setbeamercolor{background canvas}{bg=}

% frame title setup
\setbeamercolor{frametitle}{bg=,fg=dirblue}
\setbeamertemplate{frametitle}[default][left]

\addtobeamertemplate{frametitle}{\hspace*{-0.5 cm}}{\vspace*{0.25cm}}


%Шрифты
\setbeamerfont{frametitle}{family=\rmfamily,series=\bfseries,size={\fontsize{33}{30}}}
\setbeamerfont{framesubtitle}{family=\rmfamily,series=\bfseries,size={\fontsize{26}{20}}}


% удобнее знать номер слайда, чтобы вносить правки!  

\setbeamercolor{footline}{fg=dircomment}
\setbeamerfont{footline}{series=\bfseries, size={\fontsize{12}{14}}}
%\setbeamertemplate{footline}[page number]

\defbeamertemplate{footline}{custom footline}
{%
  \hspace*{\fill}%
  \usebeamercolor[fg]{page number in head/foot}%
  \usebeamerfont{page number in head/foot}%
  page: \insertpagenumber\,/\,\insertpresentationendpage%
  \hspace{20pt}%
  slide: \insertframenumber\,/\,\inserttotalframenumber%
  %\hspace*{\fill}
  \vskip2pt%
}
\setbeamertemplate{footline}[custom footline]

\usepackage{physics}
\newcommand{\R}{\mathbb{R}}
\newcommand{\Rot}{\mathrm{R}}
\newcommand{\HH}{\mathrm{H}}
\newcommand{\Id}{\mathrm{I}}


\usepackage[outline]{contour}


\usepackage{pgfplots}
\pgfplotsset{compat=newest}

\usepackage{tikz}
\usetikzlibrary{calc}
\usetikzlibrary{quotes,angles}
\usetikzlibrary{arrows}
\usetikzlibrary{arrows.meta}
\usetikzlibrary{positioning,intersections,decorations.markings}
\usetikzlibrary{patterns}

\usepackage{tkz-euclide} 

\newcommand{\grid}{\draw[color=gray,step=1.0,dotted] (-2.1,-2.1) grid (9.6,6.1)}

\newcommand{\ba}{\symbf{a}}
\newcommand{\be}{\symbf{e}}
\newcommand{\bb}{\symbf{b}}
\newcommand{\bc}{\symbf{c}}
\newcommand{\bd}{\symbf{d}}
\newcommand{\bx}{\symbf{x}}
\newcommand{\by}{\symbf{y}}
\newcommand{\bff}{\symbf{f}} % \bf is already def
\newcommand{\bv}{\symbf{v}}
\newcommand{\bzero}{\symbf{0}}
\newcommand{\red}[1]{\textcolor{red}{#1}}
\newcommand{\green}[1]{\textcolor{green}{#1}}
\newcommand{\blue}[1]{\textcolor{blue}{#1}}


\DeclareMathOperator{\Lin}{Span}
\DeclareMathOperator{\col}{col}
\DeclareMathOperator{\row}{row}

\DeclareMathOperator{\adj}{adj}

\DeclareMathOperator{\sign}{sign}


\DeclareMathOperator{\Span}{Span}
\DeclareMathOperator{\Image}{Image}


\DeclareMathOperator{\LL}{L}

%\tikzset{>=latex}

\colorlet{veca}{red}
\colorlet{vecb}{blue}
\colorlet{vecc}{olive}


\tikzset{cross/.style={cross out, draw=black, minimum size=2*(#1-\pgflinewidth), inner sep=0pt, outer sep=0pt},
%default radius will be 1pt. 
cross/.default={5pt}}





\begin{document}


\begin{frame} % название лекции


\lecturetitle{Определитель и обратная матрица}

\end{frame}


% !TEX root = ../linal_lecture_03.tex

\begin{frame} % название фрагмента

\videotitle{Идея определителя}

\end{frame}



\begin{frame}{Краткий план:}
  \begin{itemize}[<+->]
    \item Определитель на плоскости;
    \item Определитель в пространстве.
  \end{itemize}

\end{frame}




\begin{frame}
    \frametitle{Идея определителя}

    Рассмотрим оператор преобразования плоскости, $\LL: \R^2 \to \R^2$. 

    Пара векторов $\ba$, $\bb$ переходит в пару векторов $\LL\ba$, $\LL\bb$. 

    \pause

    Как меняется площадь параллелограмма образованного двумя векторами?

    \pause

    Меняется ли направление поворота от первого вектора ко второму?
\end{frame}




\begin{frame}
    \frametitle{Идея определителя на картинке}


    \begin{center}
    \begin{tikzpicture}[
    scale=1.6,
    MyPoints/.style={draw=blue,fill=white,thick},
    Segments/.style={draw=blue!50!red!70,thick},
    MyCircles/.style={green!50!blue!50,thin}, 
    every node/.style={scale=1}
    ]
    %\grid;
    \clip (-5.5,-2.5) rectangle (7.5,6.5);


    %\draw[->, >=stealth] (-1,0)--(6.5,0) node[right]{$x_1$};
    \draw[-{Latex[length=4.5mm, width=2.5mm]}, >=stealth] (0,-1)--(0,5) node[above left]{$x_2$};

    \draw[-{Latex[length=4.5mm, width=2.5mm]}, >=stealth] (-4,0)--(6.5,0) 
    node[right]{$x_1$};


    %{\verb!->!new, arrowhead = 2mm, line width=4pt}
    %, arrowhead = 3mm
    %, arrowhead = 0.2

    % Feel free to change here coordinates of points A and B
    \pgfmathparse{0}		\let\Xa\pgfmathresult
    \pgfmathparse{0}		\let\Ya\pgfmathresult
    \coordinate (A) at (\Xa,\Ya);

    \pgfmathparse{4}		\let\Xb\pgfmathresult
    \pgfmathparse{3}		\let\Yb\pgfmathresult
    \coordinate (B) at (\Xb,\Yb);

    \pgfmathparse{4}		\let\Xc\pgfmathresult
    \pgfmathparse{0}		\let\Yc\pgfmathresult
    \coordinate (C) at (\Xc,\Yc);




    %\node [above right,darkgray] at (1,3.5) {$\det \operatorname{L} = \frac{S(\LL\ba, \LL\bb)}{S(\ba, \bb)} $};



    \pgfmathparse{4}		\let\Xg\pgfmathresult
    \pgfmathparse{0}		\let\Yg\pgfmathresult
    \coordinate (G) at (\Xg,\Yg);

    \pgfmathparse{4}		\let\Xh\pgfmathresult
    \pgfmathparse{0}		\let\Yh\pgfmathresult
    \coordinate (H) at (\Xh,\Yh);

    \pgfmathparse{4}		\let\Xi\pgfmathresult
    \pgfmathparse{0}		\let\Yi\pgfmathresult
    \coordinate (I) at (\Xi,\Yi);



    \begin{scope}[cm={1,1,1.5,0.5,(0,0)}]
    \draw[pattern=north west lines, pattern color=blue!50, draw=none ] (0,0) rectangle (2,2);
    \draw[-{Latex[length=4.5mm, width=2.5mm]}, >=stealth, vecb,thick] (0,0)--(0,2) node[below]{$\ba$};
    \draw[-{Latex[length=4.5mm, width=2.5mm]}, >=stealth, vecb,thick] (0,0)--(2,0) node[above]{$\bb$};
    \end{scope};


    \pgfmathparse{3}		\let\Xd\pgfmathresult
    \pgfmathparse{1}		\let\Yd\pgfmathresult
    \coordinate (D) at (\Xd,\Yd);

    \pgfmathparse{0}		\let\Xe\pgfmathresult
    \pgfmathparse{0}		\let\Ye\pgfmathresult
    \coordinate (E) at (\Xe,\Ye);

    \pgfmathparse{2}		\let\Xf\pgfmathresult
    \pgfmathparse{2}		\let\Yf\pgfmathresult
    \coordinate (F) at (\Xf,\Yf);


    \tkzMarkAngle[size=1, mark = none, arrows=->,line width=1pt, mkcolor=blue ](D,E,F);


    \begin{scope}[cm={-1,2,-1.5,0.5,(0,0)}]
    \draw[pattern=north west lines,pattern color=black!50, draw=none ] (0,0) rectangle (0.9,2);
    \draw[-{Latex[length=4.5mm, width=2.5mm]}, >=stealth, thick] (0,0)--(0,2) node[below]{$\operatorname{L} \ba$};
    \draw[-{Latex[length=4.5mm, width=2.5mm]}, >=stealth, thick] (0,0)--(0.9,0) node[above]{$\operatorname{L} \bb$};

    \end{scope}



    \pgfmathparse{-1}		\let\Xg\pgfmathresult
    \pgfmathparse{2}		\let\Yg\pgfmathresult
    \coordinate (G) at (\Xg,\Yg);

    \pgfmathparse{-3}		\let\Xi\pgfmathresult
    \pgfmathparse{1}		\let\Yi\pgfmathresult
    \coordinate (I) at (\Xi,\Yi);

    \tkzMarkAngle[size=1, mark = none, arrows=<-,line width=1pt, mkcolor=blue ](G,E,I);

%    \node [right,darkgray] at (-2.5,1.5) {$\operatorname{LF}$ }; 

 %   \node [right,darkgray] at (2,1.5) {$\operatorname{F}$ }; 


    %\tkzMarkAngle[size=1, mark = none, arrows=->,line width=1.5pt, mkcolor=red ](B,A,E);



    \end{tikzpicture}
\end{center}
        
%    \[
%    \det \LL  = \frac{S(\LL\ba, \LL\bb)}{S(\ba, \bb)}    
%    \]
    

\end{frame}




\begin{frame}
    \frametitle{Ориентированная площадь}


    \begin{block}{Определение}
        Возьмём площадь параллелограмма со сторонами $\ba$ и $\bb$.
        Если поворот от первого вектора ко второму идёт по часовой стрелке, то дополнительно домножим площадь на $(-1)$.

        Полученное число назовём \alert{ориентированной площадью} параллелограмма и обозначим $S(\ba, \bb)$.    
    \end{block}

    \pause

    Важен порядок векторов: 
    \[
        S(\ba, \bb) = - S(\bb, \ba).  
    \]
    
\end{frame}

\begin{frame}
    \frametitle{Ориентированная площадь}

    \begin{minipage}[H]{0.48\linewidth}


        \begin{tikzpicture}[
        scale=1.75,
        MyPoints/.style={draw=blue,fill=white,thick},
        Segments/.style={draw=blue!50!red!70,thick},
        MyCircles/.style={green!50!blue!50,thin}, 
        every node/.style={scale=0.75}
        ]
        %\grid;
        \clip (-1,-2.5) rectangle (5, 3.5);


        %\draw[->, >=stealth] (-1,0)--(6.5,0) node[right]{$x_1$};
        \draw[-{Latex[length=4.5mm, width=2.5mm]}, >=stealth] (0,-1)--(0,3) node[above left]{$x_2$};

        \draw[-{Latex[length=4.5mm, width=2.5mm]}, >=stealth] (-1,0)--(4.5,0) 
        node[right]{$x_1$};


        %{\verb!->!new, arrowhead = 2mm, line width=4pt}
        %, arrowhead = 3mm
        %, arrowhead = 0.2

        % Feel free to change here coordinates of points A and B
        \pgfmathparse{0}		\let\Xa\pgfmathresult
        \pgfmathparse{0}		\let\Ya\pgfmathresult
        \coordinate (A) at (\Xa,\Ya);

        \pgfmathparse{4}		\let\Xb\pgfmathresult
        \pgfmathparse{3}		\let\Yb\pgfmathresult
        \coordinate (B) at (\Xb,\Yb);

        \pgfmathparse{4}		\let\Xc\pgfmathresult
        \pgfmathparse{0}		\let\Yc\pgfmathresult
        \coordinate (C) at (\Xc,\Yc);




        \pgfmathparse{4}		\let\Xg\pgfmathresult
        \pgfmathparse{0}		\let\Yg\pgfmathresult
        \coordinate (G) at (\Xg,\Yg);

        \pgfmathparse{4}		\let\Xh\pgfmathresult
        \pgfmathparse{0}		\let\Yh\pgfmathresult
        \coordinate (H) at (\Xh,\Yh);

        \pgfmathparse{4}		\let\Xi\pgfmathresult
        \pgfmathparse{0}		\let\Yi\pgfmathresult
        \coordinate (I) at (\Xi,\Yi);



        \begin{scope}[cm={1,1,1.5,0.5,(0,0)}]
        \draw[pattern=north west lines, pattern color=blue!50, draw=none ] (0,0) rectangle (2,2);
        \draw[-{Latex[length=4.5mm, width=2.5mm]}, >=stealth, vecb,thick] (0,0)--(0,2) node[below]{$\ba$};
        \draw[-{Latex[length=4.5mm, width=2.5mm]}, >=stealth, vecb,thick] (0,0)--(2,0) node[above]{$\bb$};
        \end{scope};


        \pgfmathparse{3}		\let\Xd\pgfmathresult
        \pgfmathparse{1}		\let\Yd\pgfmathresult
        \coordinate (D) at (\Xd,\Yd);

        \pgfmathparse{0}		\let\Xe\pgfmathresult
        \pgfmathparse{0}		\let\Ye\pgfmathresult
        \coordinate (E) at (\Xe,\Ye);

        \pgfmathparse{2}		\let\Xf\pgfmathresult
        \pgfmathparse{2}		\let\Yf\pgfmathresult
        \coordinate (F) at (\Xf,\Yf);


        \tkzMarkAngle[size=1, mark = none, arrows=->,line width=1pt, mkcolor=blue ](D,E,F);





        \pgfmathparse{-1}		\let\Xg\pgfmathresult
        \pgfmathparse{2}		\let\Yg\pgfmathresult
        \coordinate (G) at (\Xg,\Yg);

        \pgfmathparse{-3}		\let\Xi\pgfmathresult
        \pgfmathparse{1}		\let\Yi\pgfmathresult
        \coordinate (I) at (\Xi,\Yi);


        \node [above right,darkgray] at (1,-1) {$S(\ba,\bb)>0$};



        %\tkzMarkAngle[size=1, mark = none, arrows=->,line width=1.5pt, mkcolor=red ](B,A,E);



        \end{tikzpicture}




    \end{minipage}
    \begin{minipage}[H]{0.48\linewidth}


        \begin{tikzpicture}[
        scale=1.75,
        MyPoints/.style={draw=blue,fill=white,thick},
        Segments/.style={draw=blue!50!red!70,thick},
        MyCircles/.style={green!50!blue!50,thin}, 
        every node/.style={scale=0.75}
        ]
        %\grid;
        \clip (-1,-2.5) rectangle (5, 3.5);


        %\draw[->, >=stealth] (-1,0)--(6.5,0) node[right]{$x_1$};
        \draw[-{Latex[length=4.5mm, width=2.5mm]}, >=stealth] (0,-1)--(0,3) node[above left]{$x_2$};

        \draw[-{Latex[length=4.5mm, width=2.5mm]}, >=stealth] (-1,0)--(4.5,0) 
        node[right]{$x_1$};


        %{\verb!->!new, arrowhead = 2mm, line width=4pt}
        %, arrowhead = 3mm
        %, arrowhead = 0.2

        % Feel free to change here coordinates of points A and B
        \pgfmathparse{0}		\let\Xa\pgfmathresult
        \pgfmathparse{0}		\let\Ya\pgfmathresult
        \coordinate (A) at (\Xa,\Ya);

        \pgfmathparse{4}		\let\Xb\pgfmathresult
        \pgfmathparse{3}		\let\Yb\pgfmathresult
        \coordinate (B) at (\Xb,\Yb);

        \pgfmathparse{4}		\let\Xc\pgfmathresult
        \pgfmathparse{0}		\let\Yc\pgfmathresult
        \coordinate (C) at (\Xc,\Yc);




        \pgfmathparse{4}		\let\Xg\pgfmathresult
        \pgfmathparse{0}		\let\Yg\pgfmathresult
        \coordinate (G) at (\Xg,\Yg);

        \pgfmathparse{4}		\let\Xh\pgfmathresult
        \pgfmathparse{0}		\let\Yh\pgfmathresult
        \coordinate (H) at (\Xh,\Yh);

        \pgfmathparse{4}		\let\Xi\pgfmathresult
        \pgfmathparse{0}		\let\Yi\pgfmathresult
        \coordinate (I) at (\Xi,\Yi);



        \begin{scope}[cm={1,1,1.5,0.5,(0,0)}]
        \draw[pattern=north west lines, pattern color=blue!50, draw=none ] (0,0) rectangle (2,2);
        \draw[-{Latex[length=4.5mm, width=2.5mm]}, >=stealth, vecb,thick] (0,0)--(0,2) node[below]{$\ba$};
        \draw[-{Latex[length=4.5mm, width=2.5mm]}, >=stealth, vecb,thick] (0,0)--(2,0) node[above]{$\bb$};
        \end{scope};


        \pgfmathparse{3}		\let\Xd\pgfmathresult
        \pgfmathparse{1}		\let\Yd\pgfmathresult
        \coordinate (D) at (\Xd,\Yd);

        \pgfmathparse{0}		\let\Xe\pgfmathresult
        \pgfmathparse{0}		\let\Ye\pgfmathresult
        \coordinate (E) at (\Xe,\Ye);

        \pgfmathparse{2}		\let\Xf\pgfmathresult
        \pgfmathparse{2}		\let\Yf\pgfmathresult
        \coordinate (F) at (\Xf,\Yf);


        \tkzMarkAngle[size=1, mark = none, arrows=<-,line width=1pt, mkcolor=blue ](D,E,F);





        \pgfmathparse{-1}		\let\Xg\pgfmathresult
        \pgfmathparse{2}		\let\Yg\pgfmathresult
        \coordinate (G) at (\Xg,\Yg);

        \pgfmathparse{-3}		\let\Xi\pgfmathresult
        \pgfmathparse{1}		\let\Yi\pgfmathresult
        \coordinate (I) at (\Xi,\Yi);


        \node [above right,darkgray] at (1,-1) {$S(\bb,\ba)<0$};



        %\tkzMarkAngle[size=1, mark = none, arrows=->,line width=1.5pt, mkcolor=red ](B,A,E);



        \end{tikzpicture}

    \end{minipage}
    

    

\end{frame}



\begin{frame}
    \frametitle{Идея определителя}

\begin{block}{Определение}
    Возьмём любые два вектора $\ba$ и $\bb$, для которых $S(\ba, \bb)\neq 0$.

    \alert{Определитель} оператора $\LL:\R^2 \to \R^2$ показывает во сколько раз изменяется
    ориентированная площадь
    \[
    \det \LL = \frac{S(\LL\ba, \LL\ba)}{S(\ba, \bb)}    
    \]
\end{block}    
    

\end{frame}






\begin{frame}
\frametitle{Определитель отражения}

Оператор $\LL : \begin{pmatrix}
  a_1 \\
  a_2 \\
\end{pmatrix} \to 
\begin{pmatrix}
  a_2 \\
  a_1 \\
\end{pmatrix}$ отражает относительно $x_1= x_2$.\pause




\begin{center}


\begin{tikzpicture}[
scale=1.8,
MyPoints/.style={draw=blue,fill=white,thick},
Segments/.style={draw=blue!50!red!70,thick},
MyCircles/.style={green!50!blue!50,thin}, 
every node/.style={scale=1}
]
%\grid;
\clip (-1,-2.5) rectangle (6.5, 6.5);


%\draw[->, >=stealth] (-1,0)--(6.5,0) node[right]{$x_1$};
\draw[-{Latex[length=4.5mm, width=2.5mm]}, >=stealth] (0,-1)--(0,6) node[above left]{$x_2$};

\draw[-{Latex[length=4.5mm, width=2.5mm]}, >=stealth] (-1,0)--(6,0) 
node[right]{$x_1$};


%{\verb!->!new, arrowhead = 2mm, line width=4pt}
%, arrowhead = 3mm
%, arrowhead = 0.2

% Feel free to change here coordinates of points A and B
\pgfmathparse{0}		\let\Xa\pgfmathresult
\pgfmathparse{0}		\let\Ya\pgfmathresult
\coordinate (A) at (\Xa,\Ya);

\pgfmathparse{4}		\let\Xb\pgfmathresult
\pgfmathparse{3}		\let\Yb\pgfmathresult
\coordinate (B) at (\Xb,\Yb);

\pgfmathparse{4}		\let\Xc\pgfmathresult
\pgfmathparse{0}		\let\Yc\pgfmathresult
\coordinate (C) at (\Xc,\Yc);




\pgfmathparse{4}		\let\Xg\pgfmathresult
\pgfmathparse{0}		\let\Yg\pgfmathresult
\coordinate (G) at (\Xg,\Yg);

\pgfmathparse{4}		\let\Xh\pgfmathresult
\pgfmathparse{0}		\let\Yh\pgfmathresult
\coordinate (H) at (\Xh,\Yh);

\pgfmathparse{4}		\let\Xi\pgfmathresult
\pgfmathparse{0}		\let\Yi\pgfmathresult
\coordinate (I) at (\Xi,\Yi);



\begin{scope}[cm={1.25,0.75,1.75,0.25,(0,0)}]
\draw[pattern=north west lines, pattern color=blue!50, draw=none ] (0,0) rectangle (2,2);
\draw[-{Latex[length=4.5mm, width=2.5mm]}, >=stealth, vecb,thick] (0,0)--(0,2) node[below]{$\ba$};
\draw[-{Latex[length=4.5mm, width=2.5mm]}, >=stealth, vecb,thick] (0,0)--(2,0) node[above]{$\bb$};
\end{scope};

\begin{scope}[cm={0.75,1.25,0.25,1.75,(0,0)}]
\draw[pattern=north west lines, pattern color=blue!50, draw=none ] (0,0) rectangle (2,2);
\draw[-{Latex[length=4.5mm, width=2.5mm]}, >=stealth, vecb,thick] (0,0)--(0,2) node[above]{$\operatorname{L}\ba$};
\draw[-{Latex[length=4.5mm, width=2.5mm]}, >=stealth, vecb,thick] (0,0)--(2,0) node[right]{$\operatorname{L}\bb$};
\end{scope};


\pgfmathparse{0.75}		\let\Xd\pgfmathresult
\pgfmathparse{1.25}		\let\Yd\pgfmathresult
\coordinate (D) at (\Xd,\Yd);

\pgfmathparse{0}		\let\Xe\pgfmathresult
\pgfmathparse{0}		\let\Ye\pgfmathresult
\coordinate (E) at (\Xe,\Ye);

\pgfmathparse{0.25}		\let\Xf\pgfmathresult
\pgfmathparse{1.75}		\let\Yf\pgfmathresult
\coordinate (F) at (\Xf,\Yf);


\tkzMarkAngle[size=1, mark = none, arrows=<-,line width=1pt, mkcolor=blue ](D,E,F);



\pgfmathparse{1.25}		\let\Xd\pgfmathresult
\pgfmathparse{0.75}		\let\Yd\pgfmathresult
\coordinate (D) at (\Xd,\Yd);

\pgfmathparse{0}		\let\Xe\pgfmathresult
\pgfmathparse{0}		\let\Ye\pgfmathresult
\coordinate (E) at (\Xe,\Ye);

\pgfmathparse{1.75}		\let\Xf\pgfmathresult
\pgfmathparse{0.25}		\let\Yf\pgfmathresult
\coordinate (F) at (\Xf,\Yf);


\tkzMarkAngle[size=1, mark = none, arrows=->,line width=1pt, mkcolor=blue ](F,E,D);




\pgfmathparse{-1}		\let\Xg\pgfmathresult
\pgfmathparse{-1}		\let\Yg\pgfmathresult
\coordinate (G) at (\Xg,\Yg);

\pgfmathparse{6}		\let\Xi\pgfmathresult
\pgfmathparse{6}		\let\Yi\pgfmathresult
\coordinate (I) at (\Xi,\Yi);



\draw[dashed] (G)--(I) node[near end, below right]{$x_1=x_2$};







%\tkzMarkAngle[size=1, mark = none, arrows=->,line width=1.5pt, mkcolor=red ](B,A,E);



\end{tikzpicture}



    \end{center}

\end{frame}


\begin{frame}
\frametitle{Определитель отражения}
    

Оператор $\LL : \begin{pmatrix}
  a_1 \\
  a_2 \\
\end{pmatrix} \to 
\begin{pmatrix}
  a_2 \\
  a_1 \\
\end{pmatrix}$ отражает относительно $x_1= x_2$.


\pause



    Площадь параллелограмма не изменяется. 


    Меняется направление поворота от первого вектора ко второму. 

    \pause

    \[
    \det \LL = \frac{S(\LL \ba, \LL \bb )}{S(\ba, \bb )} = -1
    \]

\end{frame}






\begin{frame}
    \frametitle{Определитель растягивания компонент}


    Рассмотрим оператор $\LL : \begin{pmatrix}
      a_1 \\
      a_2 \\
    \end{pmatrix} \to 
    \begin{pmatrix}
      2a_1 \\
      -3a_2 \\
    \end{pmatrix}$.
    

    \pause

    Одна сторона растягивается в два раза, вторая — в три раза.


    Меняется направление поворота от первого вектора ко второму. 

    \pause

    \[
    \det \LL = \frac{S(\LL \ba, \LL \bb )}{S(\ba, \bb )} = (-1)\cdot 2\cdot 3 = -6
    \]

\end{frame}




\begin{frame}
    \frametitle{Определитель поворота}


    Оператор $\Rot: \R^2 \to \R^2$ вращает плоскость.

    \pause

  \begin{center}


      \begin{tikzpicture}[
      scale=1.5,
      MyPoints/.style={draw=blue,fill=white,thick},
      Segments/.style={draw=blue!50!red!70,thick},
      MyCircles/.style={green!50!blue!50,thin}, 
      every node/.style={scale=1}
      ]
      %\grid;
      \clip (-7,-2.5) rectangle (6.5, 5.5);



      %{\verb!->!new, arrowhead = 2mm, line width=4pt}
      %, arrowhead = 3mm
      %, arrowhead = 0.2

      % Feel free to change here coordinates of points A and B
      \pgfmathparse{0}		\let\Xa\pgfmathresult
      \pgfmathparse{0}		\let\Ya\pgfmathresult
      \coordinate (A) at (\Xa,\Ya);




      \pgfmathparse{2}		\let\Xb\pgfmathresult
      \pgfmathparse{0.5}		\let\Yb\pgfmathresult
      \coordinate (B) at (\Xb,\Yb);

      \pgfmathparse{0.5}		\let\Xd\pgfmathresult
      \pgfmathparse{1.5}		\let\Yd\pgfmathresult
      \coordinate (D) at (\Xd,\Yd);

      \pgfmathparse{130}		\let\angle\pgfmathresult;
      \pgfmathparse{sqrt(10)}		\let\rad\pgfmathresult;


      \pgfmathparse{\Xb*cos(\angle)  - \Yb*sin(\angle)}		\let\Xe\pgfmathresult
      \pgfmathparse{\Xb*sin(\angle)  + \Yb*cos(\angle)}		\let\Ye\pgfmathresult
      \coordinate (E) at (\Xe,\Ye);

      \pgfmathparse{\Xd*cos(\angle)  - \Yd*sin(\angle)}		\let\Xf\pgfmathresult
      \pgfmathparse{\Xd*sin(\angle)  + \Yd*cos(\angle)}		\let\Yf\pgfmathresult
      \coordinate (F) at (\Xf,\Yf);

      \tkzMarkAngle[size=1, mark = none, arrows=->,line width=1pt, mkcolor=blue ](D,A,F);


      \begin{scope}[cm={\Xb,\Yb,\Xd,\Yd,(0,0)}]
      \draw[pattern=north west lines, pattern color=blue!50, draw=none ] (0,0) rectangle (2,2);
      \draw[-{Latex[length=4.5mm, width=2.5mm]}, >=stealth, vecb,thick] (0,0)--(0,2) node[above left]{$\bb$};
      \draw[-{Latex[length=4.5mm, width=2.5mm]}, >=stealth, vecb,thick] (0,0)--(2,0) node[below right]{$\ba$};
      \end{scope};

      \begin{scope}[cm={\Xe,\Ye,\Xf,\Yf,(0,0)}]
      \draw[pattern=north west lines, pattern color=blue!50, draw=none ] (0,0) rectangle (2,2);
      \draw[-{Latex[length=4.5mm, width=2.5mm]}, >=stealth, vecb,thick] (0,0)--(0,2) node[below right]{$\operatorname{R}\bb$};
      \draw[-{Latex[length=4.5mm, width=2.5mm]}, >=stealth, vecb,thick] (0,0)--(2,0) node[above right]{$\operatorname{R}\ba$};
      \end{scope};














      %\tkzMarkAngle[size=1, mark = none, arrows=->,line width=1.5pt, mkcolor=red ](B,A,E);



      \end{tikzpicture}


  \end{center}


\end{frame}



\begin{frame}
    \frametitle{Определитель поворота}


    Оператор $\Rot: \R^2 \to \R^2$ вращает плоскость.

    \pause

    При вращении не изменяется площадь параллелограмма.

    При вращении не изменяется направление поворота от первого вектора ко второму.

    \pause

    \[
    \det \Rot = \frac{S(\Rot \ba, \Rot \bb )}{S(\ba, \bb )} = 1     
    \]

\end{frame}





\begin{frame}
    \frametitle{Определитель проекции}


    Оператор $\HH: \R^2 \to \R^2$ проецирует векторы на прямую $\ell$.   \pause

    \begin{center}

        \begin{tikzpicture}[
        scale=2,
        MyPoints/.style={draw=blue,fill=white,thick},
        Segments/.style={draw=blue!50!red!70,thick},
        MyCircles/.style={green!50!blue!50,thin}, 
        every node/.style={scale=1.2}
        ]
        %\draw[color=gray,step=1.0,dotted] (-2.1,-5.1) grid (7.6,2.1);
        \clip (-1.5,-5.5) rectangle (6.5,2.5);

        \pgfmathparse{3}		\let\Xa\pgfmathresult
        \pgfmathparse{-3}		\let\Ya\pgfmathresult
        \coordinate (A) at (\Xa,\Ya);

        \pgfmathparse{6}		\let\Xb\pgfmathresult
        \pgfmathparse{0}		\let\Yb\pgfmathresult
        \coordinate (B) at (\Xb,\Yb);

        \pgfmathparse{-1}		\let\Xc\pgfmathresult
        \pgfmathparse{1}		\let\Yc\pgfmathresult
        \coordinate (C) at (\Xc,\Yc);

        \pgfmathparse{4}		\let\Xd\pgfmathresult
        \pgfmathparse{-4}		\let\Yd\pgfmathresult
        \coordinate (D) at (\Xd,\Yd);

        \pgfmathparse{0}		\let\Xe\pgfmathresult
        \pgfmathparse{0}		\let\Ye\pgfmathresult
        \coordinate (E) at (\Xe,\Ye);

        \pgfmathparse{3.5}		\let\Xf\pgfmathresult
        \pgfmathparse{-0.5}		\let\Yf\pgfmathresult
        \coordinate (F) at (\Xf,\Yf);

        \pgfmathparse{4}		\let\Xg\pgfmathresult
        \pgfmathparse{0}		\let\Yg\pgfmathresult
        \coordinate (G) at (\Xg,\Yg);

        \pgfmathparse{4.5}		\let\Xh\pgfmathresult
        \pgfmathparse{1.5}		\let\Yh\pgfmathresult
        \coordinate (H) at (\Xh,\Yh);



        \begin{scope}[cm={1.5,0.5,1.5,-0.5,(0,0)}]
        \draw[pattern=north west lines, pattern color=blue!50, draw=none ] (0,0) rectangle (2,2);
        \draw[-{Latex[length=4.5mm, width=2.5mm]}, >=stealth, vecb,thick] (0,0)--(0,2) node[below]{$\ba$};
        \draw[-{Latex[length=4.5mm, width=2.5mm]}, >=stealth, vecb,thick] (0,0)--(2,0) node[above]{$\bb$};
        \end{scope};


        \draw[ black,dashed] (C)--(D) node[below right]{$\ell$};

        \draw[dashed] (A)--(B);

        \draw[vecb,line width=0.5mm] (E)--(A) ;



        \node [above right] at (0, 5) {$\operatorname{H} \bv = \bb$}; 



        \tkzMarkRightAngle[size=0.3](B,A,C);


        \end{tikzpicture}





    \end{center}



\end{frame}




\begin{frame}
    \frametitle{Определитель проекции}


    Оператор $\HH: \R^2 \to \R^2$ проецирует векторы на прямую $\ell$.  



    \pause

    При проекции любой параллелограмм «складывается» в отрезок нулевой площади.


    \[
    \det \HH = \frac{S(\HH \ba, \HH \bb )}{S(\ba, \bb )} = 0     
    \]

\end{frame}




\begin{frame}
    \frametitle{Чем прекрасна ориентированная площадь?}

    \begin{block}{Утверждение}
        Ориентированная площадь $S(\ba, \bb)$ линейна по каждому аргументу:
        \[
        S(\lambda \ba, \bb) = \lambda S(\ba, \bb), \quad S(\ba + \bb, \bc) = S(\ba, \bc) + S(\bb, \bc)    
        \]
    \end{block}
    \pause
  \begin{center}
      \begin{tikzpicture}[
      scale=1.5,
      MyPoints/.style={draw=blue,fill=white,thick},
      Segments/.style={draw=blue!50!red!70,thick},
      MyCircles/.style={green!50!blue!50,thin}, 
      every node/.style={scale=1}
      ]
      %\grid;
      \clip (-1,-2.5) rectangle (6.5, 4.5);





      %{\verb!->!new, arrowhead = 2mm, line width=4pt}
      %, arrowhead = 3mm
      %, arrowhead = 0.2

      % Feel free to change here coordinates of points A and B
      \pgfmathparse{0}		\let\Xa\pgfmathresult
      \pgfmathparse{0}		\let\Ya\pgfmathresult
      \coordinate (A) at (\Xa,\Ya);

      \pgfmathparse{4}		\let\Xb\pgfmathresult
      \pgfmathparse{3}		\let\Yb\pgfmathresult
      \coordinate (B) at (\Xb,\Yb);

      \pgfmathparse{4}		\let\Xc\pgfmathresult
      \pgfmathparse{0}		\let\Yc\pgfmathresult
      \coordinate (C) at (\Xc,\Yc);




      \pgfmathparse{4}		\let\Xg\pgfmathresult
      \pgfmathparse{0}		\let\Yg\pgfmathresult
      \coordinate (G) at (\Xg,\Yg);

      \pgfmathparse{4}		\let\Xh\pgfmathresult
      \pgfmathparse{0}		\let\Yh\pgfmathresult
      \coordinate (H) at (\Xh,\Yh);

      \pgfmathparse{4}		\let\Xi\pgfmathresult
      \pgfmathparse{0}		\let\Yi\pgfmathresult
      \coordinate (I) at (\Xi,\Yi);



      \begin{scope}[cm={0.5,-1,1.5,0,(0,0)}]
      \draw[pattern=north west lines, pattern color=blue!50, draw=none ] (0,0) rectangle (2,2);
      \draw[-{Latex[length=4.5mm, width=2.5mm]}, >=stealth, vecb,thick] (0,0)--(0,2);
      \draw[{Latex[length=4.5mm, width=2.5mm]}-, >=stealth, vecb,thick] (0,0)--(2,0) node[midway, left]{$\ba$};
      \draw[{Latex[length=4.5mm, width=2.5mm]}-, >=stealth, vecb,thick] (0,2)--(2,2) ;
      \draw[-{Latex[length=4.5mm, width=2.5mm]}, >=stealth, vecb,thick] (2,0)--(2,2) node[midway, below]{$\bc$};

      \end{scope};

      \begin{scope}[cm={1.5,0,1,2,(0,0)}]
      \draw[pattern=north west lines, pattern color=blue!50, draw=none ] (0,0) rectangle (2,2);
      \draw[-{Latex[length=4.5mm, width=1.5mm]}, >=stealth, vecb,thick] (0,0)--(0,2) node[midway, above left]{$\bb$};
      \draw[-{Latex[length=4.5mm, width=1.5mm]}, >=stealth, vecb,thick] (2,0)--(2,2) ;
      \draw[-{Latex[length=4.5mm, width=2.5mm]}, >=stealth, vecb,thick] (0,2)--(2,2) node[midway, above ]{$\bc$};
      \end{scope};


      \draw[-{Latex[length=4.5mm, width=1.5mm]}, >=stealth, vecb,thick] (1,-2)--(2,4);


      \draw[-{Latex[length=4.5mm, width=1.5mm]}, >=stealth, vecb,thick] (4,-2)--(5,4) node[midway, right ]{$\ba+\bb$};














      %\tkzMarkAngle[size=1, mark = none, arrows=->,line width=1.5pt, mkcolor=red ](B,A,E);



      \end{tikzpicture}


  \end{center}



    

\end{frame}



\begin{frame}
    \frametitle{Корректность идеи определителя}

    Величина $\det \LL = \frac{S(\LL\ba, \LL\bb)}{S(\ba, \bb)}$ не зависит от выбора $\ba$ и $\bb$!


    \pause
    \begin{block}{Идея доказательства}
        Обозначим $\be_1 = \begin{pmatrix}
            1 \\
            0 \\
        \end{pmatrix}$, $\be_2 = \begin{pmatrix}
            0 \\
            1 \\
        \end{pmatrix}$.
        \pause

        Возьмём $\ba = 5\be_1 + 7\be_2$. Найдём $S(\LL\ba, \LL\be_2)/{S(\ba, \be_2)}$:
        \pause
        \[
        \frac{S(\LL (5\be_1 + 7\be_2), \LL \be_2)}{S(5\be_1 + 7\be_2,\be_2)} =
         \frac{S(\LL 5\be_1 , \LL \be_2) + S(\LL 7\be_2 , \LL \be_2)}{S(5\be_1,\be_2) + S(7\be_2,\be_2)} =
        \]
        \pause
        \[
         =\frac{5S(\LL \be_1 , \LL \be_2) + 0}{5S(\be_1,\be_2) + 0} =
         \frac{S(\LL \be_1 , \LL \be_2)}{S(\be_1,\be_2)}
        \]
    \end{block}

    

\end{frame}



\begin{frame}
    \frametitle{Ещё один взгляд на определитель}
    
    Обозначим $\be_1 = \begin{pmatrix}
        1 \\
        0 \\
    \end{pmatrix}$, $\be_2 = \begin{pmatrix}
        0 \\
        1 \\
    \end{pmatrix}$.

    \pause

    \begin{block}{Определение}
        Преобразуем параллелограмм, образованный векторами $\be_1$ и $\be_2$, с помощью оператора $\LL$. 
        
        Определитель линейного оператора $\LL:\R^2 \to \R^2$ равен ориентированной 
        площади полученного параллелограмма.
        \[
        \det \LL = S(\LL\be_1, \LL\be_2)    
        \]
        
    \end{block}
    
\end{frame}


\begin{frame}
    \frametitle{Определитель в пространстве}

    Идея: заменим ориентированную площадь параллелограмма $S(\ba, \bb)$ 
    на ориентированный объём параллелепипеда $S(\ba, \bb, \bc)$.
    
    \pause

\begin{block}{Определение}
    Возьмём любые три вектора $\ba$, $\bb$ и $\bc$, для которых $S(\ba, \bb, \bc)\neq 0$.

    \alert{Определитель} оператора $\LL:\R^3 \to \R^3$ показывает во сколько раз изменяется
    ориентированный объём
    \[
    \det \LL = \frac{S(\LL\ba, \LL\ba, \LL\bc)}{S(\ba, \bb, \bc)}    
    \]
\end{block} 


\end{frame}



\begin{frame}
    \frametitle{А что такое ориентированный объём?}

\pause 

Обозначим $\be_1 = \begin{pmatrix}
    1 \\
    0 \\
    0 \\
\end{pmatrix}$, $\be_2 = \begin{pmatrix}
    0 \\
    1 \\
    0
\end{pmatrix}$, $\be_3 = \begin{pmatrix}
    0 \\
    0 \\
    1 \\
\end{pmatrix}$.

% Скажем, что $S(\be_1, \be_2, \be_3) = 1$.

% Сопоставим большой, указательный и средний палец этим векторам.

\pause

\begin{block}{Определение}
    Рассмотрим параллелепипед, образованный $\ba$, $\bb$ и $\bc$.

    \pause

    С помощью поворота: 

    Cовместим вектор $\be_1$ с вектором $\ba$;

    Затем вектор $\be_2$ «положим» в плоскость $\ba$, $\bb$.

    \pause

    \alert{Ориентированный объём} $S(\ba, \bb, \bc)$ объявим отрицательным,
    если векторы $\be_3$ и $\bc$ смотрят в разные полупространства.
    
\end{block}


\end{frame}


\begin{frame}
    \frametitle{Определитель в пространстве}


    \begin{center}
    \begin{tikzpicture}[
    scale=1.4,
    MyPoints/.style={draw=blue,fill=white,thick},
    Segments/.style={draw=blue!50!red!70,thick},
    MyCircles/.style={green!50!blue!50,thin}, 
    every node/.style={scale=1.2}
    ]
    %\draw[color=gray,step=1.0,dotted] (-5.5,-6.5) grid (5.5,6.5); 
    \clip (-5.5,-6.9) rectangle (9.1,6.5);

    %{\verb!->!new, arrowhead = 2mm, line width=4pt}
    %, arrowhead = 3mm
    %, arrowhead = 0.2

    % Feel free to change here coordinates of points A and B
    \pgfmathparse{0}		\let\Xa\pgfmathresult
    \pgfmathparse{0}		\let\Ya\pgfmathresult
    \coordinate (A) at (\Xa,\Ya);

    \pgfmathparse{0}		\let\Xb\pgfmathresult
    \pgfmathparse{3}		\let\Yb\pgfmathresult
    \coordinate (B) at (\Xb,\Yb);

    \pgfmathparse{2}		\let\Xc\pgfmathresult
    \pgfmathparse{2}		\let\Yc\pgfmathresult
    \coordinate (C) at (\Xc,\Yc);

    \pgfmathparse{4}		\let\Xd\pgfmathresult
    \pgfmathparse{0}		\let\Yd\pgfmathresult
    \coordinate (D) at (\Xd,\Yd);

    \pgfmathparse{2}		\let\Xe\pgfmathresult
    \pgfmathparse{5}		\let\Ye\pgfmathresult
    \coordinate (E) at (\Xe,\Ye);

    \pgfmathparse{6}		\let\Xf\pgfmathresult
    \pgfmathparse{5}		\let\Yf\pgfmathresult
    \coordinate (F) at (\Xf,\Yf);

    \pgfmathparse{4}		\let\Xg\pgfmathresult
    \pgfmathparse{3}		\let\Yg\pgfmathresult
    \coordinate (G) at (\Xg,\Yg);

    \pgfmathparse{6}		\let\Xg\pgfmathresult
    \pgfmathparse{2}		\let\Yg\pgfmathresult
    \coordinate (H) at (\Xg,\Yg);


    \draw[-{Latex[length=4.5mm, width=2.5mm]}, >=stealth, vecb,  thick] (A)--(B) node[midway, left]{$\ba$};

    \draw[-{Latex[length=4.5mm, width=2.5mm]}, >=stealth, vecb,  thick] (A)--(C) node[midway, above]{$\bb$};

    \draw[-{Latex[length=4.5mm, width=2.5mm]}, >=stealth, vecb,  thick] (A)--(D) node[midway, below]{$\bc$};

    \draw[vecb, dashed] (C)--(E);
    \draw[vecb, dashed] (B)--(E);
    \draw[vecb, dashed] (E)--(F);
    \draw[vecb, dashed] (G)--(F);
    \draw[vecb, dashed] (G)--(B);
    \draw[vecb, dashed] (G)--(D);
    \draw[vecb, dashed] (D)--(H);
    \draw[vecb, dashed] (F)--(H);
    \draw[vecb, dashed] (C)--(H);

    \begin{scope}[cm={0.1,-0.9,-1,0,(0,0)}]
    \draw[-{Latex[length=4.5mm, width=2.5mm]}, >=stealth,   thick] (0,0)--(0,3) node[midway, above]{$\operatorname{L} \ba$};
    \draw[-{Latex[length=4.5mm, width=2.5mm]}, >=stealth,   thick] (0,0)--(2,2) node[midway, above left ]{$\operatorname{L} \bb$};
    \draw[-{Latex[length=4.5mm, width=2.5mm]}, >=stealth,   thick] (0,0)--(4,0) node[midway, right]{$\operatorname{L} \bc$};
    \draw[vecb, dashed] (2,2)--(2,5);
    \draw[vecb, dashed] (0,3)--(2,5);
    \draw[vecb, dashed] (2,5)--(6,5);
    \draw[vecb, dashed] (4,3)--(6,5);
    \draw[vecb, dashed] (4,3)--(0,3);
    \draw[vecb, dashed] (4,3)--(4,0);
    \draw[vecb, dashed] (4,0)--(6,2);
    \draw[vecb, dashed] (6,5)--(6,2);
    \draw[vecb, dashed] (2,2)--(6,2);
    \end{scope}



    %\node [right,darkgray] at (-5,-6) {$\operatorname{LF}$ }; 

    %\node [right,darkgray] at (5,5.5) {$\operatorname{F}$ }; 



    %\draw[-{Latex[length=4.5mm, width=2.5mm]}, >=stealth, thick] (A)--(B) node[left]{$\ba$};
    %
    %\draw[-{Latex[length=4.5mm, width=2.5mm]}, >=stealth, thick] (A)--(C) node[above]{$\bb$};
    %
    %\draw[-{Latex[length=4.5mm, width=2.5mm]}, >=stealth, thick] (A)--(D) node[above]{$\bc$};

    \node [above right,darkgray] at (1.3,-3.5) {$\det \operatorname{L} = \dfrac{S(\LL\ba,\LL\bb,\LL\bc)}{S(\ba,\bb,\bc)} $};





    \end{tikzpicture}

\end{center}        
    

\end{frame}




\begin{frame}
    \frametitle{Определитель растягивания компонент}


    Рассмотрим оператор $\LL : \begin{pmatrix}
      a_1 \\
      a_2 \\
      a_3 \\
    \end{pmatrix} \to 
    \begin{pmatrix}
      2a_1 \\
      3a_2 \\
      -5a_3 \\
    \end{pmatrix}$.
    

    \pause

    Одна сторона растягивается в два раза, вторая — в три раза, третья — в пять.


    Первые два вектора не изменяют направления при преобразовании.

    
    Третий вектор меняет полупространство, в котором он лежит относительно первых двух. 

    \pause

    \[
    \det \LL = \frac{S(\LL \ba, \LL \bb, \LL \bc )}{S(\ba, \bb, \bc )} = (-1)\cdot 2\cdot 3\cdot 5  = - 30
    \]

\end{frame}



\begin{frame}
    \frametitle{Определитель проекции}


    Оператор $\HH : \R^3 \to \R^3$ проецирует векторы на плоскость $\alpha$. 

    \pause

    Любой параллелепипед «схлопывается» в плоскую фигуру нулевого объёма.

    \pause

    \[
    \det \HH = \frac{S(\HH \ba, \HH \bb, \HH \bc )}{S(\ba, \bb, \bc )} = 0
    \]

\end{frame}




% !TEX root = ../linal_lecture_03.tex

\begin{frame} % название фрагмента

\videotitle{Свойства определителя}

\end{frame}



\begin{frame}{Краткий план:}
  \begin{itemize}[<+->]
    \item Ориентированный объём в $\R^n$;
    \item Свойства определителя;
    \item Явная формула для определителя.
  \end{itemize}

\end{frame}




\begin{frame}
    \frametitle{Формализация ориентированного объёма}

    %Преобразование $\LL\R^n\to\R^n$, желаемые свойства $\det \LL$:

    Вектор $\be_i$ содержит на $i$-м месте единицу, а на остальных — нули.
    \pause

    \begin{enumerate}
        \item Верный гипер-объём базового гипер-кубика:
        \[
        S(\be_1, \be_2, \ldots, \be_n) = 1   \pause 
        \]
        \item Линейность по каждому аргументу:
        \[
            S(\red{\ba + \bb}, \bv_2, \bv_3, \ldots, \bv_n) =  S(\red{\ba}, \bv_2, \bv_3, \ldots, \bv_n) + 
        \]
        \[
            + S(\red{\bb}, \bv_2, \bv_3, \ldots, \bv_n)          
        \]
        \[
            S(\red{\lambda}\bv_1, \bv_2, \bv_3, \ldots, \bv_n) = \red{ \lambda } S(\bv_1, \bv_2, \bv_3, \ldots, \bv_n) \pause
        \]
        \item Антисимметричность: 
        \[
          S(\red{\bv_1}, \red{\bv_2}, \bv_3, \ldots, \bv_n) = - S(\red{\bv_2}, \red{\bv_1}, \bv_3, \ldots, \bv_n)  
        \]
    \end{enumerate}
\end{frame}



\begin{frame}
    \frametitle{Определитель во всей $n$-мерности}

    \begin{block}{Определение}
        Возьмём векторы $\bv_1$, \ldots, $\bv_n$, для которых $S(\bv_1, \ldots, \bv_n)\neq 0$.

        \alert{Определитель} оператора $\LL:\R^n \to \R^n$ показывает во сколько раз изменяется
        ориентированный гипер-объём произвольного параллелепипеда:
        \[
            \det \LL = \frac{S(\LL\bv_1, \ldots, \LL\bv_n)}{S(\bv_1, \ldots, \bv_n)}    
        \]
    \end{block}  
    
    \pause

\begin{block}{Определение}

    \alert{Определитель} оператора $\LL:\R^n \to \R^n$ показывает во сколько раз изменяется
    ориентированный гипер-объём базового гипер-кубика:
    \[
        \det \LL = S(\LL\be_1, \ldots, \LL\be_n)
    \]
\end{block}  


\end{frame}



\begin{frame}
    \frametitle{Определитель матрицы}

    \begin{block}{Определение}
        \alert{Определителем матрицы} называется определитель соответствующего линейного оператора. 
    \end{block}
    
    \pause

    В матрице $\LL$ $i$-й столбец равен $\LL\be_i$, поэтому 
    \[
    \det \LL = S(\col_1 \LL, \col_2 \LL, \ldots, \col_n \LL)    
    \]

    \pause
    \begin{block}{Утверждение}
        Определитель матрицы можно считать по строкам: 
    \[
    \det \LL = S(\row_1 \LL, \row_2 \LL, \ldots, \row_n \LL)    
    \]    
    \end{block}

    Определитель обозначают $\det \LL$ или $|\LL|$.

\end{frame}





\begin{frame}
    \frametitle{Быстрые признаки равенства нулю}

    \begin{enumerate}
        \item Если среди векторов есть два одинаковых, то 
        гипер-объём параллелепипеда равен нулю. 
        \[
            S(\ba, \ba, \bv_3, \ldots, \bv_n) = 0
        \]
        \pause
        \item Если среди векторов есть один нулевой, то 
        гипер-объём параллелепипеда равен нулю. 
        \[
            S(\bzero, \bv_2, \bv_3, \ldots, \bv_n) = 0
        \]
        \end{enumerate}

\end{frame}

\begin{frame}
\frametitle{Принцип Кавальери}

        «Cкашивание» параллелепипеда вбок не изменяет гипер-объём:
        \[
            S(\ba, \bb, \bv_3, \ldots, \bv_n) = S(\ba + \bb, \bb, \bv_3, \ldots, \bv_n)
        \]
    \pause
    \begin{center}
        \begin{tikzpicture}[
        scale=1.7,
        MyPoints/.style={draw=blue,fill=white,thick},
        Segments/.style={draw=blue!50!red!70,thick},
        MyCircles/.style={green!50!blue!50,thin}, 
        every node/.style={scale=1.2}
        ]
        \clip (-.5,-1.5) rectangle (7.5,4.5);


        %%\draw[->, >=stealth] (-1,0)--(6.5,0) node[right]{$x_1$};
        %\draw[-{Latex[length=4.5mm, width=2.5mm]}, >=stealth] (0,-1)--(0,5) node[above left]{$x_2$};
        %
        %\draw[-{Latex[length=4.5mm, width=2.5mm]}, >=stealth] (-1,0)--(6.5,0) 
        %node[right]{$x_1$};

        % Feel free to change here coordinates of points A and B
        \pgfmathparse{0}		\let\Xa\pgfmathresult
        \pgfmathparse{0}		\let\Ya\pgfmathresult
        \coordinate (A) at (\Xa,\Ya);

        \pgfmathparse{1}		\let\Xb\pgfmathresult
        \pgfmathparse{4}		\let\Yb\pgfmathresult
        \coordinate (B) at (\Xb,\Yb);

        \pgfmathparse{3}		\let\Xc\pgfmathresult
        \pgfmathparse{0}		\let\Yc\pgfmathresult
        \coordinate (C) at (\Xc,\Yc);

        \pgfmathparse{4}		\let\Xd\pgfmathresult
        \pgfmathparse{4}		\let\Yd\pgfmathresult
        \coordinate (D) at (\Xd,\Yd);


        \pgfmathparse{7}		\let\Xe\pgfmathresult
        \pgfmathparse{4}		\let\Ye\pgfmathresult
        \coordinate (E) at (\Xe,\Ye);



        \draw[-{Latex[length=4.5mm, width=2.5mm]}, >=stealth, vecb,thick] (A)--(B) node[midway,left]{$\ba$};

        \draw[-{Latex[length=4.5mm, width=2.5mm]}, >=stealth, vecb,thick] (C)--(D);


        \draw[-{Latex[length=4.5mm, width=2.5mm]}, >=stealth, veca,thick] (A)--(C) node[midway,below]{$\bb$};

        \draw[-{Latex[length=4.5mm, width=2.5mm]}, >=stealth, veca,thick] (B)--(D) ;

        \draw[-{Latex[length=4.5mm, width=2.5mm]}, >=stealth, veca,thick] (D)--(E) ;


        \draw[-{Latex[length=4.5mm, width=1.5mm]}, >=stealth, vecc,thick] (A)--(D) node[midway,above, sloped]{$\ba+\bb$};

        \draw[-{Latex[length=4.5mm, width=1.5mm]}, >=stealth, vecc,thick] (C)--(E) ;

        % \node [above right,darkgray] at (1,-1.5) {$S(\ba,\bb)=S(\ba+\bb,\bb)$};


        \end{tikzpicture}


    \end{center}


    
\end{frame}



\begin{frame}
    \frametitle{Принцип Кавальери на матрице}
    
    Единственным ненулевым элементом столбца можно «скосить» всю строку:

    \[
    \begin{vmatrix}
        0 & 3 & -2 \\
        \red{4} & 2 & 7 \\
        0 & 1 & -5 \\
    \end{vmatrix} =  \pause 
\begin{vmatrix}
    0 & 3 & -2 \\
    \red{4} & \blue{0} & 7 \\
    0 & 1 & -5 \\
\end{vmatrix}  = \pause 
\begin{vmatrix}
    0 & 3 & -2 \\
    \red{4} & \blue{0} & \blue{0} \\
    0 & 1 & -5 \\
\end{vmatrix}  
    \]

    \pause
    Единственным ненулевым элементом строки можно «скосить» весь столбец:
    
    \[
    \begin{vmatrix}
        3 & 3 & -2 \\
        \red{4} & 0 & 0 \\
        2 & 1 & -5 \\
    \end{vmatrix} = \pause 
\begin{vmatrix}
    \blue{0} & 3 & -2 \\
\red{4} & 0 & 0 \\
2 & 1 & -5 \\
\end{vmatrix}  = \pause 
\begin{vmatrix}
    \blue{0} & 3 & -2 \\
\red{4} & 0 & 0 \\
\blue{0} & 1 & -5 \\
\end{vmatrix}  
    \]
    
\end{frame}


\begin{frame}
    \frametitle{Определитель и ранг}

    \begin{block}{Утверждение}
        Для матрицы $\LL$ размера $n\times n$ четыре свойства эквиваленты:

        \begin{enumerate}
            \item Определитель равен нулю, $\det \LL = 0$.
            \pause
            \item Столбцы матрицы линейно зависимы.
            \pause
            \item Строки матрицы линейно зависимы.
            \pause
            \item Ранг матрицы меньше числа столбцов, $\rank \LL < n$.
        \end{enumerate}
    \end{block}


    

\end{frame}

\begin{frame}
    \frametitle{Определитель композиции}
    \begin{block}{Утверждение}
        Определитель композиции $A$ и $B$ равен произведению определителей:
        \[
          \det (AB) = \det A \det B  
        \]
    \end{block}
    \pause
    \begin{block}{Следствие}

        \[
        \det A \det A^{-1} = \det (A \cdot A^{-1}) = \det \Id = 1
        \]
    \end{block}

\end{frame}





\begin{frame}
    \frametitle{Спокойствие, только спокойствие!}

    \begin{block}{Утверждение}
        Свойства нормировки, линейности по аргументам и антисимметричности однозначно определяют функцию гипер-объёма $S(\bv_1, \ldots, \bv_n)$.
    \end{block}

    \pause

\begin{block}{Утверждение}
    Отношение гипер-объёмов $\det \LL = \frac{S(\LL\bv_1, \ldots, \LL\bv_n)}{S(\bv_1, \ldots, \bv_n)}$ не зависит от выбора $\bv_1$, \ldots, $\bv_n$.
\end{block}


\end{frame}




\begin{frame}
    \frametitle{Формула с перестановками}

    \begin{block}{Определение}
        \alert{Перестановкой} называют последовательность из $n$ чисел, 
        в которой каждое число от $1$ до $n$ встречается ровно один раз.
    \end{block}

    \pause
    Примеры: $(12345)$, $(32145)$, $(21354)$.

    \pause
    \begin{block}{Определение}
    Перестановку называют \alert{чётной}, если требуется чётное число смен местами двух чисел,
    чтобы привести перестановку к $(1234\ldots n)$.

    Если $\sigma$ — чётная перестановка, то пишут $\sign \sigma  = 1$,
    для нечётной пишут $\sign\sigma = -1$.
\end{block}
\pause
Примеры: 

$\sign(12345) = 1$, $\sign(32145)=-1$, $\sign(21354)=1$.


\end{frame}



\begin{frame}
\frametitle{Расстановка ладей!}

    Рассмотрим квадратную матрицу. 

    Перестановку $\sigma$ будем трактовать как инструкцию, какой элемент взять из очередной строки матрицы.
    
    \[
    (3124) \sim \begin{pmatrix}
    . & . & * & . \\
    * & . & . & . \\
    . & * & . & . \\
    . & . & . & * \\
    \end{pmatrix}    
    \]

    \pause

    С помощью $p(\sigma)$ обозначим произведение этих элементов. 

    Например, $p(3124) = a_{13} \cdot a_{21} \cdot a_{32} \cdot a_{44}$.


\end{frame}


\begin{frame}
\frametitle{Явная формула}

\begin{block}{Утверждение}
Трём свойствам определителя (нормировке, линейности, антисимметричности) удовлетворяет единственная функция
\[
    \det \LL = \sum_{\sigma} \sign (\sigma) \cdot p(\sigma).
\]

Перестановку $\sigma$ трактуем как инструкцию, какой элемент взять из очередной строки матрицы.

С помощью $p(\sigma)$ обозначено произведение этих элементов. 

\end{block}


\end{frame}


\begin{frame}
    \frametitle{Иллюстрация для $2\times 2$}

    \[
    \det \begin{pmatrix}
        a & b \\
        c & d \\
    \end{pmatrix} = \pause
    +\underset{\red{\sign(12)=1}}{\begin{pmatrix}
        a &  \\
         & d \\
    \end{pmatrix}} - 
    \underset{\textcolor{blue}{\sign(21)=-1}}{\begin{pmatrix}
         & b \\
        c &  \\
    \end{pmatrix}} = \pause  ad - bc
    \]

\end{frame}



\begin{frame}
    \frametitle{Иллюстрация для $3\times 3$}

    \[
    \det \begin{pmatrix}
        a & b & c \\
        d & e & f \\
        g & h & i \\
    \end{pmatrix} = \pause
    \]
\[
= +\underset{\red{\sign(123)=1}}{\begin{pmatrix}
    a &  &  \\
     & e &  \\
     &  & i \\
\end{pmatrix}} +
\underset{\red{\sign(312)=1}}{\begin{pmatrix}
     &  & c \\
    d &  &  \\
     & h &  \\
\end{pmatrix}} +
\underset{\red{\sign(231)=1}}{\begin{pmatrix}
     & b &  \\
     &  & f \\
    g &  &  \\
\end{pmatrix}}
\]
\[
- \underset{\textcolor{blue}{\sign(321)=-1}}{\begin{pmatrix}
     &  & c \\
     & e &  \\
    g &  &  \\
\end{pmatrix}} 
- \underset{\textcolor{blue}{\sign(213)=-1}}{\begin{pmatrix}
     & b & \\
    d &  &  \\
     &  & i \\
\end{pmatrix}} 
- \underset{\textcolor{blue}{\sign(132)=-1}}{\begin{pmatrix}
    a &  &  \\
     &  & f \\
     & h &  \\
\end{pmatrix}}=
\]
\pause
\[
= aei + cdh + bfg - ceg - bdi - afh
\]




\end{frame}





\begin{frame}

\lecturetitle{Вычисление определителя}

\todo{Это видеофрагмент с доской, слайдов здесь нет :)}

\end{frame}
    

% !TEX root = ../linal_lecture_03.tex

\begin{frame} % название фрагмента

\videotitle{Определитель и транспонирование}

\end{frame}



\begin{frame}{Краткий план:}
  \begin{itemize}[<+->]
    \item Транспонирование матрицы;
    \item Определитель и транспонирование;
    \item Разложение определителя по строке.
  \end{itemize}

\end{frame}




\begin{frame}
    \frametitle{Конструктивный подход к транспонированию}

    Определение транспонирования оператора основано на свойстве 
    \[
        \langle \LL \ba, \bb \rangle = \langle \ba, \LL^T\bb\rangle.
    \]

    \pause

    Возьмём, к примеру, $\ba = \be_2$ и $\bb = \be_3$:
    \[
        \langle \col_2 \LL, \be_3 \rangle = \langle \be_2, \col_3 \LL^T \rangle
    \]
    \pause 
    \[
        \LL_{32} = \LL^T_{23}        
    \]
    \pause

    Транспонирование меняет местами строки и столбцы матрицы!


\end{frame}


\begin{frame}
    \frametitle{Транспонирование матрицы}

    Пример:
    \[
    \LL = \begin{pmatrix}
        1 & 2 & 3 & 4 \\
        5 & 6 & 7 & 8 \\
        9 & 10 & 11 & 12 \\
    \end{pmatrix}    
    \]
    \pause
    \[
    \LL^T = \begin{pmatrix}
        1 & 5 & 9 \\
        2 & 6 & 10 \\
        3 & 7 & 11 \\
        4 & 8 & 12 \\
    \end{pmatrix}    
    \]
    \pause
    Заметим, что $\LL^{TT}=\LL$.
\end{frame}


\begin{frame}
    \frametitle{Транспонирование и определитель}

    Явная формула определителя:
    \[    
    \det \LL = \sum_{\sigma} \sign(\sigma) p(\sigma) 
    \]
    \pause
    %
    Перестановка диктует, какой элемент выбрать в каждой строке:
    \[
    (3124) \sim \begin{pmatrix}
    . & . & * & . \\
    * & . & . & . \\
    . & * & . & . \\
    . & . & . & * \\
    \end{pmatrix}
    \]
    \pause
    \begin{block}{Утверждение}
        Если в матрице выбран один элемент в каждой строке и в каждом столбце, то при транспонировании это свойство сохраняется.
    \end{block}

\end{frame}

\begin{frame}
\frametitle{Транспонирование и определитель}

\begin{block}{Утверждение}
    Чётности перестановок, кодирующих координаты элементов по строкам и по столбцам, одинаковые.
\end{block}
    %
    \[
        \begin{pmatrix}
    . & . & \textcolor{red}{a} & . \\
    \textcolor{red}{b} & . & . & . \\
    . & c & . & . \\
    . & . & . & d \\
    \end{pmatrix}
    \]   
    %
    \[
        (\col_1 \leftrightarrow \col_3) \sim (\row_1 \leftrightarrow \row_2)
    \]
    \pause
    \[
        \sign(3124) = \sign(2314)
    \]
\end{frame}



\begin{frame}
    \frametitle{Транспонирование и определитель}

Перестановка $\sigma$ выбирает элемент в каждой строке:
\[    
\det \LL = \sum_{\sigma} \sign(\sigma) p(\sigma) 
\]

Перестановка $\sigma$ выбирает элемент в каждом столбце:
\[    
\det \LL^T = \sum_{\sigma} \sign(\sigma) p^T(\sigma)
\]

\pause

\begin{block}{Утверждение}
\[
    \det \LL = \det \LL^T
\]
\end{block}

\end{frame}


\begin{frame}
    \frametitle{Разложение по столбцу на примере}
    Возьмём аддитивность:
    \[
\begin{vmatrix}
    -1 & \red{2} & 3 \\
    4 & \red{5} & 6 \\
    7 & \red{8} & 9 \\
\end{vmatrix}  =       
\begin{vmatrix}
    -1 & \red{2} & 3 \\
    4 & 0 & 6 \\
    7 & 0 & 9 \\
\end{vmatrix} +
\begin{vmatrix}
    -1 & 0 & 3 \\
    4 & \red{5} & 6 \\
    7 & 0 & 9 \\
\end{vmatrix} +
\begin{vmatrix}
    -1 & 0 & 3 \\
    4 & 0 & 6 \\
    7 & \red{8} & 9 \\
\end{vmatrix} \pause
    \]
    Добавим немного принципа Кавальери:
\[
    \begin{vmatrix}
        -1 & \red{2} & 3 \\
        4 & \red{5} & 6 \\
        7 & \red{8} & 9 \\
    \end{vmatrix}  =       
    \begin{vmatrix}
        \blue{0} & \red{2} & \blue{0} \\
        4 & 0 & 6 \\
        7 & 0 & 9 \\
    \end{vmatrix} +
    \begin{vmatrix}
        -1 & 0 & 3 \\
\blue{0} & \red{5} & \blue{0} \\
        7 & 0 & 9 \\
    \end{vmatrix} +
    \begin{vmatrix}
        -1 & 0 & 3 \\
        4 & 0 & 6 \\
\blue{0} & \red{8} & \blue{0} \\
    \end{vmatrix} \pause
\]
Взболтаем и переставим столбцы:
\[
    \begin{vmatrix}
        -1 & \red{2} & 3 \\
        4 & \red{5} & 6 \\
        7 & \red{8} & 9 \\
    \end{vmatrix}  =       
    (-1)^1 \begin{vmatrix}
        \red{2} & 0   & 0 \\
        0 & 4  & 6 \\
        0 & 7  & 9 \\
    \end{vmatrix} +
    (-1)^2 \begin{vmatrix}
        \red{5} & 0 & 0 \\
       0 & -1 & 3 \\
        0 & 7 & 9 \\
    \end{vmatrix} +
    (-1)^3\begin{vmatrix}
        \red{8} & 0 & 0 \\
        0 & -1 & 3 \\
        0 & 4 & 6 \\
    \end{vmatrix} 
\]
\end{frame}


\begin{frame}
\frametitle{Разложение по столбцу на примере}
Взболтаем и переставим столбцы:
\[
    \begin{vmatrix}
        -1 & \red{2} & 3 \\
        4 & \red{5} & 6 \\
        7 & \red{8} & 9 \\
    \end{vmatrix}  =       
    (-1)^1 \begin{vmatrix}
        \red{2} & 0   & 0 \\
        0 & 4  & 6 \\
        0 & 7  & 9 \\
    \end{vmatrix} +
    (-1)^2 \begin{vmatrix}
        \red{5} & 0 & 0 \\
       0 & -1 & 3 \\
        0 & 7 & 9 \\
    \end{vmatrix} +
    (-1)^3\begin{vmatrix}
        \red{8} & 0 & 0 \\
        0 & -1 & 3 \\
        0 & 4 & 6 \\
    \end{vmatrix} \pause
\]
Снизим размерность:
\[
    \begin{vmatrix}
        -1 & \red{2} & 3 \\
        4 & \red{5} & 6 \\
        7 & \red{8} & 9 \\
    \end{vmatrix}  =       
    (-1)^1 \red{2} \begin{vmatrix}
         4  & 6 \\
         7  & 9 \\
    \end{vmatrix} +
    (-1)^2 \red{5}\begin{vmatrix}
        -1 & 3 \\
         7 & 9 \\
    \end{vmatrix} +
    (-1)^3 \red{8} \begin{vmatrix}
         -1 & 3 \\
         4 & 6 \\
    \end{vmatrix} 
\]




\end{frame}


\begin{frame}
    \frametitle{Разложение по столбцу}

    Выберем любой столбец и «пробежимся» вдоль него!

    \[
        \begin{vmatrix}
            * & \red{a_{12}} & * \\
            * & \red{a_{22}} & * \\
            * & \red{a_{32}} & * \\
        \end{vmatrix}  =      
    \]
    \[ 
       =(-1)^{1+2} \red{a_{12}} \det A_{12}^{\cross} +
        (-1)^{2+2} \red{a_{22}} \det A_{22}^{\cross} +
        (-1)^{3+2} \red{a_{32}} \det A_{32}^{\cross}
    \]
    Матрица $A_{ij}^{\cross}$ получается из исходной $A$ вычеркиванием строки $i$ и
    столбца $j$.
    \pause
    \begin{block}{Утверждение}       
    \[
    \det \LL = \sum_{i=1}^{n} (-1)^{i+j} a_{ij} \det A_{ij}^{\cross},
    \]
    \end{block}

    
\end{frame}


\begin{frame}
    \frametitle{Разложение по строке}

    Можно раскладывать и по строке $i$:
\begin{block}{Утверждение}
\[
\det A = \sum_{j=1}^{n} (-1)^{i+j} a_{ij} \det A_{ij}^{\cross},
\]
\end{block}

    \begin{block}{Определение}
        \alert{Алгебраическим дополнением} элемента $a_{ij}$ матрицы $A$ называют величину
        \[
        A_{ij} = (-1)^{i+j} \det A_{ij}^{\cross},    
        \]
    \end{block}

    Матрица $A_{ij}^{\cross}$ получается из исходной $A$ вычеркиванием строки $i$ и
    столбца $j$.

\end{frame}


\begin{frame}

\lecturetitle{Метод Гаусса}

\todo{Это видеофрагмент с доской, слайдов здесь нет :)}

\end{frame}



\begin{frame}
\lecturetitle{Метод Крамера}
\todo{Это видеофрагмент с доской, слайдов здесь нет :)}
\end{frame}



\begin{frame}
\lecturetitle{Метод Крамера и нахождение обратной матрицы}
\todo{Это видеофрагмент с доской, слайдов здесь нет :)}
\end{frame}
    



% !TEX root = ../linal_lecture_03.tex

\begin{frame} % название фрагмента

\videotitle{Поиск обратной матрицы: итоги}

\end{frame}



\begin{frame}{Краткий план:}
  \begin{itemize}[<+->]
    \item Метод Гаусса;
    \item Метод Крамера;
    \item Критерий наличия обратной матрицы.
  \end{itemize}

\end{frame}




\begin{frame}
    \frametitle{Как поймать двух зайцев?}

    Если нужно решить две системы уравнений, $A \bx = \bb$ и $A \by = \bc$, то можно решать их одновременно!

    \pause

    Вместо двух систем
    \[
            \left(
            \begin{array}{ccc|c}
            1 & 3 & 3  & 10 \\
            -1 & 1 & 1 & 2 \\
            2 & 7 & 1  & 17 \\
            \end{array}
            \right), \quad
            \left(
            \begin{array}{ccc|c}
            1 & 3 & 3  & 10 \\
            -1 & 1 & 1 & 6 \\
            2 & 7 & 1  & 6 \\
            \end{array}
            \right)    
    \]
    \pause

    решаем систему \alert{с двойной правой частью}:
\[
        \left(
        \begin{array}{ccc|cc}
        1 & 3 & 3  & 10 & 10 \\
        -1 & 1 & 1 & 2 & 6 \\
        2 & 7 & 1  & 17 & 6 \\
        \end{array}
        \right).
\]


\end{frame}




\begin{frame}
    \frametitle{Как поймать двух зайцев?}

    Решаем систему \alert{с двойной правой частью}:
\[
        \left(
        \begin{array}{ccc|cc}
        1 & 3 & 3  & 10 & 10 \\
        -1 & 1 & 1 & 2 & 6 \\
        2 & 7 & 1  & 17 & 6 \\
        \end{array}
        \right) 
\]
\pause
Приводим левую часть к виду единичной матрицы используя гауссовские преобразования:
\begin{itemize}
    \item перестановку строк;
    \item домножение строки на ненулевое число;
    \item прибавление одной строки к другой с любым весом.
\end{itemize}
\pause
\[
\left(
\begin{array}{ccc|cc}
1 & 0 & 0  & 1 & -2 \\
0 & 1 & 0 & 2 & 1 \\
0 & 0 & 1  & 1 & 3 \\
\end{array}
\right)\pause \quad \text{Видим ответ: }  \bx = \begin{pmatrix}
    1 \\
    2 \\
    1 \\
\end{pmatrix}, \by = \begin{pmatrix}
    -2 \\
    1 \\
    3 \\
\end{pmatrix}
\]


\end{frame}









\begin{frame}
    \frametitle{Метод Гаусса}
Для нахождения обратной матрицы для $A$ нужно решить систему $A \cdot B = \Id$. 

Для примера
\[
A = \begin{pmatrix}
    3 & 2 & 4 \\
    4 & 3 & 6 \\
    6 & 4 & 9 \\
\end{pmatrix}
\]
\pause
Приписываем справа единичную матрицу $\Id$:
\[
\left(
\begin{array}{ccc|ccc}
3 & 2 & 4  & 1 & 0 & 0 \\
4 & 3 & 6 & 0 & 1 & 0 \\
6 & 4 & 9  & 0 & 0 & 1\\
\end{array}
\right)
\]
\end{frame}

\begin{frame}
\frametitle{Метод Гаусса}

Приписываем справа единичную матрицу $\Id$:
\[
\left(
\begin{array}{ccc|ccc}
3 & 2 & 4  & 1 & 0 & 0 \\
4 & 3 & 6 & 0 & 1 & 0 \\
6 & 4 & 9  & 0 & 0 & 1\\
\end{array}
\right)
\]

Если $\det A\neq 0$, то с помощью гауссовских преобразований получаем единичную слева:
\[
\left(
\begin{array}{ccc|ccc}
1 & 0 & 0  & 3 & -2 & 0 \\
0 & 1 & 0 & 0 & 3 & -2 \\
0 & 0 & 1  & -2 & 0 & 1\\
\end{array}
\right)
\]
\pause
Решение системы, матрицу $A^{-1}$, читаем справа:
\[
A^{-1} =   \begin{pmatrix}
 3 & -2 & 0 \\
 0 & 3 & -2 \\
 -2 & 0 & 1\\
\end{pmatrix} 
\]

\end{frame}



\begin{frame}
    \frametitle{Метод Крамера}

    Рассмотрим систему $A\bx = \bb$ из $n$ уравнений с $n$ неизвестными.

    \pause

    \begin{block}{Утверждение}
        Если $\det A \neq 0$, то решение системы единственно и 
        \[
        x_i = \frac{\det A_i}{\det A},    
        \]
        где матрица $A_i$ получена из матрицы $A$ заменой $i$-го столбца на правую часть $\bb$.
    \end{block}


\end{frame}


\begin{frame}
    \frametitle{Метод Крамера для обратной матрицы}

    Матрица $A$ имеет размер $n\times n$.

    \pause

    \begin{block}{Утверждение}
        Если $\det A \neq 0$, то матрица $B=A^{-1}$ существует и
        \[
        b_{ij} = \frac{(-1)^{i+j}\det (A_{ji}^{\cross})}{\det A},    
        \]
        где матрица $A_{ji}^{\cross}$ получена из матрицы $A$ вычёркиванием строки $j$ и столбца $i$.
    \end{block}
    \pause

    Напомним, что $(-1)^{i+j}\det (A_{ji}^{\cross})$ называется \alert{алгебраическим дополнением} к $a_{ji}$.

\end{frame}


\begin{frame}
        \frametitle{Метод Крамера для обратной матрицы}
    
        Хотим обратить матрицу $A$.

        \begin{enumerate}
            \item Находим определитель $\det A$. Если $\det A = 0$, то матрица $A$ не обратимая. \pause
            \item Для каждого элемента $A$ находим алгебраическое дополнение $C_{ij} = (-1)^{i+j}\det (A_{ij}^{\cross})$.
            Помещаем все дополнения в матрицу $C$. \pause
            \item Транспонируем матрицу $С$ и получаем \alert{присоединённую матрицу} $\adj A = C^T$.  \pause
            \item Делим матрицу $\adj A$ на $\det A$ и получаем $A^{-1} = \adj A/\det A$.
        \end{enumerate}
        
    
\end{frame}


\begin{frame}
    \frametitle{Вырожденные и невырожденные матрицы}

    Матрица $A$ размера $n\times n$ называется \alert{вырожденной}, если:

    \begin{enumerate}
        \item $\det A = 0$; \pause
        \item Система $A \bx = \bzero$ имеет бесконечное количество решений; \pause
        \item $\rank A < n$; \pause
        \item $A^{-1}$ не существует; \pause
    \end{enumerate}

Матрица $A$ размера $n\times n$ называется \alert{невырожденной}, если:

\begin{enumerate}
    \item $\det A \neq 0$; \pause
    \item Система $A \bx = \bb$ имеет единственное решение; \pause
    \item $\rank A = n$; \pause
    \item $A^{-1}$ существует;
\end{enumerate}


\end{frame}


% !TEX root = ../linal_lecture_03.tex

\begin{frame} % название фрагмента

\videotitle{LU-разложение}

\end{frame}



\begin{frame}{Краткий план:}
  \begin{itemize}[<+->]
    \item Гауссовские преобразования: другой взгляд;
    \item LU-разложение;
    \item Применение LU-разложения.
  \end{itemize}

\end{frame}



\begin{frame}
    \frametitle{Треугольные матрицы}

    \begin{block}{Определение}
        Квадратная матрица называется \alert{верхнетреугольной},
        если ниже диагонали у неё стоят нулевые числа, например,
        \[
            \mathrm{U} = 
        \begin{pmatrix}
            4 & 5 & -1 & 2 \\
            0 & 0 & 2 & 6 \\
            0 & 0 & 3 & 1 \\
            0 & 0 & 0 & -1 \\
        \end{pmatrix}.    
        \]
    \end{block}
\pause
При перемножении верхнетреугольных матриц получается верхнетреугольная.
\pause
Определитель треугольной матрицы равен произведению диагональных элементов.

\end{frame}


\begin{frame}
\frametitle{Треугольные матрицы}

    \begin{block}{Определение}
        Квадратная матрица называется \alert{нижнетреугольной},
        если ниже диагонали у неё стоят нулевые числа, например,
        \[
            \LL =
        \begin{pmatrix}
            4 & 0 & 0 & 0 \\
            2 & 1 & 0 & 0 \\
            3 & 3 & 0 & 0 \\
            1 & 2 & 2 & -1 \\
        \end{pmatrix}.    
        \]
    \end{block}
    \pause
    При перемножении нижнетреугольных матриц получается нижнетреугольная.
    \pause
    Определитель треугольной матрицы равен произведению диагональных элементов.

\end{frame}




\begin{frame}
    \frametitle{Гауссовские преобразования}

    Рассмотрим систему уравнений в матричном виде:
    \[
    \left(
    \begin{array}{ccc|c}
    1 & 3 & 3  & 10 \\
    -1 & 1 & 1 & 2 \\
    2 & 7 & 1  & 17 \\
    \end{array}
    \right)
    \]
    \pause

    Гауссовские преобразования уравнений системы:
    
    \begin{enumerate}
        \item Домножение строки на ненулевое число;
        \item Перестановка двух строк местами;
        \item Прибавление к данной строке другой строки, домноженной на произвольное $\lambda$.
    \end{enumerate}
    \pause
    Каждое из этих действий можно закодировать умножением на матрицу!

\end{frame}

\begin{frame}
    \frametitle{Домножение строки как матрица}


    Домножим вторую строка матрицы $A$ на $7$:
    \[
    \begin{pmatrix}
        1 & 0 & 0 \\
        0 & \red{7} & 0 \\
        0 & 0 & 1 \\
    \end{pmatrix}   \cdot 
    \begin{pmatrix}
        a & b & c \\
        d & e & f \\
        g & h & i \\
    \end{pmatrix} =\pause
\begin{pmatrix}
    a & b & c \\
    \red{7d} & \red{7e} & \red{7f} \\
    g & h & i \\
\end{pmatrix}   
    \]


Левая матрица задаёт веса строк для правой матрицы!


\end{frame}



\begin{frame}
    \frametitle{Перестановка строк как матрица}

 
Переставим первую и вторую строки матрицы $A$:

    \[
    \begin{pmatrix}
        0 & \red{1} & 0 \\
        \red{1} & 0 & 0 \\
        0 & 0 & 1 \\
    \end{pmatrix}   \cdot 
    \begin{pmatrix}
        a & b & c \\
        d & e & f \\
        g & h & i \\
    \end{pmatrix} =\pause
\begin{pmatrix}
    \red{d} & \red{e} & \red{f} \\
    \red{a} & \red{b} & \red{c} \\
    g & h & i \\
\end{pmatrix}   
    \]

    Левая матрица задаёт веса строк для правой матрицы!


\end{frame}


\begin{frame}
    \frametitle{Прибавление строки как матрица}

    Из второй строки вычтем первую строку с весом $4$:

    \[
    \begin{pmatrix}
        1 & 0 & 0 \\
        \red{-4} & 1 & 0 \\
        0 & 0 & 1 \\
    \end{pmatrix}   \cdot 
    \begin{pmatrix}
        a & b & c \\
        d & e & f \\
        g & h & i \\
    \end{pmatrix} =\pause
\begin{pmatrix}
    a & b & c \\
    d -\red{4a} & e - \red{4b} & f-\red{4c} \\
    g & h & i \\
\end{pmatrix}   
    \]
    
    %\vspace{10pt}
    Левая матрица задаёт веса строк для правой матрицы!
 \pause
 Прибавлению строки к другой строке ниже соответствует нижнетреугольная матрица.

\end{frame}



\begin{frame}
    \frametitle{Вычитание — антоним прибавления}

    

    \[
    \begin{pmatrix}
        1 & 0 & 0 \\
        \red{-4} & 1 & 0 \\
        0 & 0 & 1 \\
    \end{pmatrix}   \cdot 
\begin{pmatrix}
    1 & 0 & 0 \\
    \red{4} & 1 & 0 \\
    0 & 0 & 1 \\
\end{pmatrix} =
\begin{pmatrix}
    1 & 0 & 0 \\
    0 & 1 & 0 \\
    0 & 0 & 1 \\
\end{pmatrix}   
    \]
    
\pause
Левая матрица отвечает за вычитание первой строки из второй с весом $4$.

\pause
Правая матрица отвечает за прибавление первой строки ко второй с весом $4$.


\end{frame}





\begin{frame}
    \frametitle{Переход к треугольной матрице}

    С помощью метода Гаусса мы приводим квадратную матрицу к ступенчатому верхнетреугольному виду:

    \[
    A \overset{g_1}{\to} A_1  \overset{g_2}{\to} A_2 \overset{g_3}{\to} \ldots A_{k-1}\overset{g_k}{\to} \mathrm{U} 
    \]
    \pause

    В матричном виде:
    \[
    \mathrm{U} = G_k \cdot G_{k-1} \cdot \ldots  \cdot  G_2 \cdot G_{1} A
    \]    

\end{frame}


\begin{frame}
    \frametitle{Уменьшим список разрешённых действий!}

    \begin{block}{Алгоритм приведения к ступенчатому виду}
    \begin{enumerate}
        \item Выберем первое уравнение так, чтобы в нём была переменная $x_1$.
        \item Вычитаем первое уравнение из остальных так, чтобы в них пропала переменная $x_1$.
        \item Зафиксируем первое уравнение и работаем с остальными. 
    \end{enumerate}
    \end{block}
    \pause
    Выводы:
    \begin{itemize}
        \item Можно обойтись без домножения строк на число. \pause
        \item Все перестановки строк можно сделать в начале. 
\end{itemize}

    

\end{frame}


\begin{frame}
    \frametitle{$LU$-разложение}

    С помощью метода Гаусса мы приводим квадратную матрицу к ступенчатому верхнетреугольному виду:

    \[
    A \overset{p}{\to} A_1  \overset{\ell_1}{\to} A_2 \overset{\ell_2}{\to} \ldots A_{k-1}\overset{\ell_k}{\to} \mathrm{U} 
    \]
    \pause
    В матричном виде: 
    \[
    \mathrm{U} = \LL_k \cdot \LL_{k-1} \cdot \ldots  \cdot  \LL_2 \cdot \LL_{1} \cdot \mathrm{P} \cdot A.\pause
    \]
    Отменим действия $\LL_i$.
    \[
        \LL_1^{-1} \cdot \LL_2^{-1} \cdot \ldots \cdot \LL_k^{-1} \cdot  \LL_k^{-1} \cdot \mathrm{U} =  \mathrm{P} \cdot A.\pause
    \]
    Гауссовские преобразования эквивалентны разложению
    \[
    \LL\cdot  \mathrm{U} = \mathrm{P} \cdot A.     
    \]

\end{frame}

\begin{frame}
    \frametitle{Зачем нужно $LU$-разложение?}

    Если $LU$-разложение матрицы $A$ получено, то можно очень быстро
    \begin{itemize}
        \item найти определитель матрицы $A$;
        \item решить любую систему $A\bx = \bb$.
    \end{itemize}

    \pause
    \[
    \begin{pmatrix}
        1 & 0 & 0 \\
        2 & 1 & 0 \\
        -3 & 5 & 1 \\
    \end{pmatrix} \cdot 
    \begin{pmatrix}
        -2 & 1 & 3 \\
        0 & -1 & 2 \\
        0 &  0 & 1 \\
    \end{pmatrix} = 
    \begin{pmatrix}
-2 & 1 & 3 \\
-4 & 1 & 8 \\
6 &  -8 & 2 \\        
    \end{pmatrix}
    \]

    \pause
    \[
    \det A = 1\cdot 1\cdot 1\cdot (-2) \cdot (-1) \cdot 1 = 2
    \]    

\end{frame}



\begin{frame}
    \frametitle{Резюме}

    \begin{itemize}[<+->]
    \item Определитель измеряет изменение площади и объёма.
    \item Свойства определителя.
    \item Обращение матрицы методом Гаусса.
    \item Обращение матрицы методом Крамера.
    \item Метод Гаусса как $LU$-разложение.
    \item Бонус: комплексные числа.
    \end{itemize}
    \pause
    \alert{Следующая лекция:} спектральное разложение и диагонализация.
        


\end{frame}


    

\begin{frame}
\lecturetitle{Комплексные числа}
\todo{бонусное видео! Это видеофрагмент с доской, слайдов здесь нет :)}
\end{frame}
    

\end{document}
