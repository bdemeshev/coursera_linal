% !TEX root = ../linal_lecture_03.tex

\begin{frame} % название фрагмента

\videotitle{Поиск обратной матрицы: итоги}

\end{frame}



\begin{frame}{Краткий план:}
  \begin{itemize}[<+->]
    \item Метод Гаусса;
    \item Метод Крамера;
    \item Критерий наличия обратной матрицы.
  \end{itemize}

\end{frame}




\begin{frame}
    \frametitle{Как поймать двух зайцев?}

    Если нужно решить две системы уравнений, $A \bx = \bb$ и $A \by = \bc$, то можно решать их одновременно!

    \pause

    Вместо двух систем
    \[
            \left(
            \begin{array}{ccc|c}
            1 & 3 & 3  & 10 \\
            -1 & 1 & 1 & 2 \\
            2 & 7 & 1  & 17 \\
            \end{array}
            \right), \quad
            \left(
            \begin{array}{ccc|c}
            1 & 3 & 3  & 10 \\
            -1 & 1 & 1 & 6 \\
            2 & 7 & 1  & 6 \\
            \end{array}
            \right)    
    \]
    \pause

    решаем систему \alert{с двойной правой частью}:
\[
        \left(
        \begin{array}{ccc|cc}
        1 & 3 & 3  & 10 & 10 \\
        -1 & 1 & 1 & 2 & 6 \\
        2 & 7 & 1  & 17 & 6 \\
        \end{array}
        \right).
\]


\end{frame}




\begin{frame}
    \frametitle{Как поймать двух зайцев?}

    Решаем систему \alert{с двойной правой частью}:
\[
        \left(
        \begin{array}{ccc|cc}
        1 & 3 & 3  & 10 & 10 \\
        -1 & 1 & 1 & 2 & 6 \\
        2 & 7 & 1  & 17 & 6 \\
        \end{array}
        \right) 
\]
\pause
Приводим левую часть к виду единичной матрицы используя гауссовские преобразования:
\begin{itemize}
    \item перестановку строк;
    \item домножение строки на ненулевое число;
    \item прибавление одной строки к другой с любым весом.
\end{itemize}
\pause
\[
\left(
\begin{array}{ccc|cc}
1 & 0 & 0  & 1 & -2 \\
0 & 1 & 0 & 2 & 1 \\
0 & 0 & 1  & 1 & 3 \\
\end{array}
\right)\pause \quad \text{Видим ответ: }  \bx = \begin{pmatrix}
    1 \\
    2 \\
    1 \\
\end{pmatrix}, \by = \begin{pmatrix}
    -2 \\
    1 \\
    3 \\
\end{pmatrix}
\]


\end{frame}









\begin{frame}
    \frametitle{Метод Гаусса}
Для нахождения обратной матрицы для $A$ нужно решить систему $A \cdot B = \Id$. 

Для примера
\[
A = \begin{pmatrix}
    3 & 2 & 4 \\
    4 & 3 & 6 \\
    6 & 4 & 9 \\
\end{pmatrix}
\]
\pause
Приписываем справа единичную матрицу $\Id$:
\[
\left(
\begin{array}{ccc|ccc}
3 & 2 & 4  & 1 & 0 & 0 \\
4 & 3 & 6 & 0 & 1 & 0 \\
6 & 4 & 9  & 0 & 0 & 1\\
\end{array}
\right)
\]
\end{frame}

\begin{frame}
\frametitle{Метод Гаусса}

Приписываем справа единичную матрицу $\Id$:
\[
\left(
\begin{array}{ccc|ccc}
3 & 2 & 4  & 1 & 0 & 0 \\
4 & 3 & 6 & 0 & 1 & 0 \\
6 & 4 & 9  & 0 & 0 & 1\\
\end{array}
\right)
\]

Если $\det A\neq 0$, то с помощью гауссовских преобразований получаем единичную слева:
\[
\left(
\begin{array}{ccc|ccc}
1 & 0 & 0  & 3 & -2 & 0 \\
0 & 1 & 0 & 0 & 3 & -2 \\
0 & 0 & 1  & -2 & 0 & 1\\
\end{array}
\right)
\]
\pause
Решение системы, матрицу $A^{-1}$, читаем справа:
\[
A^{-1} =   \begin{pmatrix}
 3 & -2 & 0 \\
 0 & 3 & -2 \\
 -2 & 0 & 1\\
\end{pmatrix} 
\]

\end{frame}



\begin{frame}
    \frametitle{Метод Крамера}

    Рассмотрим систему $A\bx = \bb$ из $n$ уравнений с $n$ неизвестными.

    \pause

    \begin{block}{Утверждение}
        Если $\det A \neq 0$, то решение системы единственно и 
        \[
        x_i = \frac{\det A_i}{\det A},    
        \]
        где матрица $A_i$ получена из матрицы $A$ заменой $i$-го столбца на правую часть $\bb$.
    \end{block}


\end{frame}


\begin{frame}
    \frametitle{Метод Крамера для обратной матрицы}

    Матрица $A$ имеет размер $n\times n$.

    \pause

    \begin{block}{Утверждение}
        Если $\det A \neq 0$, то матрица $B=A^{-1}$ существует и
        \[
        b_{ij} = \frac{(-1)^{i+j}\det (A_{ji}^{\cross})}{\det A},    
        \]
        где матрица $A_{ji}^{\cross}$ получена из матрицы $A$ вычёркиванием строки $j$ и столбца $i$.
    \end{block}
    \pause

    Напомним, что $(-1)^{i+j}\det (A_{ji}^{\cross})$ называется \alert{алгебраическим дополнением} к $a_{ji}$.

\end{frame}


\begin{frame}
        \frametitle{Метод Крамера для обратной матрицы}
    
        Хотим обратить матрицу $A$.

        \begin{enumerate}
            \item Находим определитель $\det A$. Если $\det A = 0$, то матрица $A$ не обратимая. \pause
            \item Для каждого элемента $A$ находим алгебраическое дополнение $C_{ij} = (-1)^{i+j}\det (A_{ij}^{\cross})$.
            Помещаем все дополнения в матрицу $C$. \pause
            \item Транспонируем матрицу $С$ и получаем \alert{присоединённую матрицу} $\adj A = C^T$.  \pause
            \item Делим матрицу $\adj A$ на $\det A$ и получаем $A^{-1} = \adj A/\det A$.
        \end{enumerate}
        
    
\end{frame}


\begin{frame}
    \frametitle{Вырожденные и невырожденные матрицы}

    Матрица $A$ размера $n\times n$ называется \alert{вырожденной}, если:

    \begin{enumerate}
        \item $\det A = 0$; \pause
        \item Система $A \bx = \bzero$ имеет бесконечное количество решений; \pause
        \item $\rank A < n$; \pause
        \item $A^{-1}$ не существует; \pause
    \end{enumerate}

Матрица $A$ размера $n\times n$ называется \alert{невырожденной}, если:

\begin{enumerate}
    \item $\det A \neq 0$; \pause
    \item Система $A \bx = \bb$ имеет единственное решение; \pause
    \item $\rank A = n$; \pause
    \item $A^{-1}$ существует;
\end{enumerate}


\end{frame}