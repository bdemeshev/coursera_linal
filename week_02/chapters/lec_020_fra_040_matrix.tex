% !TEX root = ../linal_lecture_02.tex

\begin{frame} % название фрагмента

\videotitle{Матрица линейного оператора}

\end{frame}



\begin{frame}{Краткий план:}
  \begin{itemize}[<+->]
    \item Матрица линейного оператора;
    \item Примеры;
    \item Обобщение на векторное пространство.
  \end{itemize}

\end{frame}


\begin{frame}{Как записать линейный оператор?}

Любой вектор $\bv$ представим в виде:
\[
\bv =  \begin{pmatrix}
    v_1 \\
    v_2 \\
    \vdots \\
    v_n 
\end{pmatrix} =  
v_1  \begin{pmatrix}
    1 \\
    0 \\
    \vdots \\
    0 
\end{pmatrix} + v_2  \begin{pmatrix}
    0 \\
    1 \\
    \vdots \\
    0 
\end{pmatrix} + \ldots +
v_n  \begin{pmatrix}
    0 \\
    0 \\
    \vdots \\
    1 
\end{pmatrix}
\]

\pause
По свойству линейности
\[
\LL \bv =  \LL \begin{pmatrix}
    v_1 \\
    v_2 \\
    \vdots \\
    v_n 
\end{pmatrix} =  
v_1  \LL \begin{pmatrix}
    1 \\
    0 \\
    \vdots \\
    0 
\end{pmatrix} + v_2  \LL \begin{pmatrix}
    0 \\
    1 \\
    \vdots \\
    0 
\end{pmatrix} + \ldots +
v_n  \LL \begin{pmatrix}
    0 \\
    0 \\
    \vdots \\
    1 
\end{pmatrix}
\]

\pause
Достаточно понять, что оператор $\LL$ делает с векторами, содержащими 
одну единичку и нули на остальных местах.

\end{frame}


\begin{frame}{Запишем оператор по столбцам!}


Обозначим $\be_i$ — вектор, у которого на $i$-м месте стоит $1$, а на остальных местах — $0$.
\[
\be_i = \begin{pmatrix}
    0 \\
    \vdots \\
    0 \\
    1 \\
    0  \\
    \vdots \\
    0 \\
\end{pmatrix}    
\]

\pause 
\begin{block}{Определение}
\alert{Матрицей линейного оператора} $\LL: \R^n \to \R^k$ назовём прямоугольную табличку чисел, в которой $i$-ый столбец равен $\LL \be_i$. 
\end{block}
\end{frame}


\begin{frame}{Растягивание компонент}

$\LL : \begin{pmatrix}
  a_1 \\
  a_2 \\
\end{pmatrix} \to 
\begin{pmatrix}
  2a_1 \\
  -3a_2 \\
\end{pmatrix}$

\pause

Действие оператора $\LL$ на базисных векторах $\be_1$ и $\be_2$:

$\LL : \begin{pmatrix}
  1 \\
  0 \\
\end{pmatrix} \to 
\begin{pmatrix}
  2 \\
  0 \\
\end{pmatrix}, \quad
\LL : \begin{pmatrix}
  0 \\
  1 \\
\end{pmatrix} \to 
\begin{pmatrix}
  0 \\
  -3 \\
\end{pmatrix}$

\pause

Матрица оператора, $\LL = 
\begin{pmatrix}
  2 & 0  \\
  0 & -3 \\
\end{pmatrix}.$



\end{frame}



\begin{frame}{Перестановка компонент вектора}

$\LL : \begin{pmatrix}
  a_1 \\
  a_2 \\
  a_3 \\
\end{pmatrix} \to 
\begin{pmatrix}
  a_3 \\
  a_1 \\
  a_2 \\
\end{pmatrix}$

\pause

Действие оператора $\LL$ на базисных векторах $\be_1$, $\be_2$ и $\be_3$:

$\LL : \begin{pmatrix}
  1 \\
  0 \\
  0 \\
\end{pmatrix} \to 
\begin{pmatrix}
  0 \\
  1 \\
  0 \\
\end{pmatrix}, \;
\LL : \begin{pmatrix}
  0 \\
  1 \\
  0 \\
\end{pmatrix} \to 
\begin{pmatrix}
  0 \\
  0 \\
  1 \\
\end{pmatrix} \;
\LL : \begin{pmatrix}
  0 \\
  0 \\
  1 \\
\end{pmatrix} \to 
\begin{pmatrix}
  1 \\
  0 \\
  0 \\
\end{pmatrix}$

\pause

Матрица оператора, $\LL = 
\begin{pmatrix}
  0 & 0 & 1  \\
  1 & 0 & 0 \\
  0 & 1 & 0 \\
\end{pmatrix}.$



\end{frame}


\begin{frame}{Поворот плоскости}


Оператор $\Rot :\R^2 \to \R^2$ поворачивает плоскость на $30^{\circ}$ против часовой стрелки.

$\Rot : \begin{pmatrix}
  a_1 \\
  a_2 \\
\end{pmatrix} \to 
\begin{pmatrix}
  a_1\cos 30^{\circ} - a_2 \sin 30^{\circ}  \\
  a_1\sin 30^{\circ} + a_2 \cos 30^{\circ}  \\
\end{pmatrix}$

\pause

Действие оператора $\Rot$ на базисных векторах $\be_1$ и $\be_2$:

$\Rot : \begin{pmatrix}
  1 \\
  0 \\
\end{pmatrix} \to 
\begin{pmatrix}
\cos 30^{\circ}  \\
\sin 30^{\circ}  \\
\end{pmatrix}, \quad
\Rot : \begin{pmatrix}
  0 \\
  1 \\
\end{pmatrix} \to 
\begin{pmatrix}
 -\sin 30^{\circ}  \\
 \cos 30^{\circ} \\
\end{pmatrix}$

\pause

Матрица оператора, $\Rot = 
\begin{pmatrix}
  \cos 30^{\circ} & -\sin 30^{\circ}  \\
  \sin 30^{\circ} & \cos 30^{\circ} \\
\end{pmatrix}.$



\end{frame}
    


\begin{frame}{Оператор бездельника!}


$\Id : \begin{pmatrix}
  a_1 \\
  a_2 \\
\end{pmatrix} \to 
\begin{pmatrix}
  a_1 \\
  a_2 \\
\end{pmatrix}$

\pause

Действие оператора $\Id$ на базисных векторах $\be_1$ и $\be_2$:

$\Id : \begin{pmatrix}
  1 \\
  0 \\
\end{pmatrix} \to 
\begin{pmatrix}
1  \\
0  \\
\end{pmatrix}, \quad
\Id : \begin{pmatrix}
  0 \\
  1 \\
\end{pmatrix} \to 
\begin{pmatrix}
0 \\
1 \\
\end{pmatrix}$

\pause

Матрица оператора, \alert{единичная матрица}, $\Id = 
\begin{pmatrix}
  1 & 0  \\
  0 & 1 \\
\end{pmatrix}.$



\end{frame}
    



\begin{frame}{Дописывание нуля}


$\LL : \begin{pmatrix}
  a_1 \\
  a_2 \\
\end{pmatrix} \to 
\begin{pmatrix}
  a_1 \\
  0 \\
  a_2 \\
\end{pmatrix}$

\pause

Действие оператора $\LL$ на базисных векторах $\be_1$ и $\be_2$:

$\LL : \begin{pmatrix}
  1 \\
  0 \\
\end{pmatrix} \to 
\begin{pmatrix}
1  \\
0 \\
0  \\
\end{pmatrix}, \quad
\LL : \begin{pmatrix}
  0 \\
  1 \\
\end{pmatrix} \to 
\begin{pmatrix}
0 \\
0 \\
1 \\
\end{pmatrix}$

\pause

Матрица оператора, $\LL = 
\begin{pmatrix}
  1 & 0  \\
  0 & 0 \\
  0 & 1 \\
\end{pmatrix}.$


Матрица размера $3\times 2$ соответствует оператору $\LL: \R^2 \to \R^3$.


\end{frame}
    



\begin{frame}{Удаление компоненты вектора}


$\LL : \begin{pmatrix}
  a_1 \\
  a_2 \\
  a_3 \\
\end{pmatrix} \to 
\begin{pmatrix}
  a_1 \\
  a_3 \\
\end{pmatrix}$

\pause

Действие оператора $\LL$ на базисных векторах $\be_1$, $\be_2$ и $\be_3$:

$\LL : \begin{pmatrix}
  1 \\
  0 \\
  0 \\
\end{pmatrix} \to 
\begin{pmatrix}
1  \\
0  \\
\end{pmatrix}, \;
\LL : \begin{pmatrix}
  0 \\
  1 \\
  0 \\
\end{pmatrix} \to 
\begin{pmatrix}
0 \\
0 \\
\end{pmatrix}, \;
\LL : \begin{pmatrix}
  0 \\
  0 \\
  1 \\
\end{pmatrix} \to 
\begin{pmatrix}
0 \\
1 \\
\end{pmatrix}$

\pause

Матрица оператора, $\LL = 
\begin{pmatrix}
  1 & 0 & 0 \\
  0 & 0 & 1 \\
\end{pmatrix}.$


Матрица размера $2\times 3$ соответствует оператору $\LL: \R^3 \to \R^2$.


\end{frame}
    







\begin{frame}{Нумерация элементов матрицы}

\alert{Сначала строки, потом столбцы!}
\pause

\[
A = \begin{pmatrix} 
    5 & \textcolor{red}{-2} & 8 \\
    7 & 1 & 9 \\
\end{pmatrix}    
\]

Матрица $A$ имеет размер $2 \times 3$ и $a_{12} = -2$.

\pause
Элемент матрицы $A$, лежащий в строке $i$ в столбце $j$, 
обозначают $a_{ij}$.

Матрица имеет размер $n\times k$, если в ней $n$ строк
и $k$ столбцов.

\end{frame}



\begin{frame}{Абстрактное определение}

% Мы не делали разницу между оператором $\LL$ и его матрицей.

%\pause 
% В тех редких ситуациях, когда эту разницу нужно подчеркнуть, 
% потребуются аккуратные обозначения. 

% \pause
Пусть оператор $\LL$ действует из пространства $V$ с базисом 
$\be = \{\be_1, \be_2, \ldots, \be_n\}$ в пространство $W$ с базисом
$\bff = \{\bff_1, \bff_2, \ldots, \bff_k\}$.

\pause
\begin{block}{Определение}
\alert{Матрицей} $\LL_{\be\bff}$ \alert{линейного оператора} $\LL$ называется табличка
чисел, определяемая по следующему алгоритму:

\begin{enumerate}
     \item Находим вектор $\LL \be_j \in W$.
     \pause
    \item Раскладываем этот вектор по базису $\bff$:
     \[
         \LL \be_j = a_{1j} \bff_1 + a_{2j} \bff_2 + \ldots + a_{kj} \bff_k
     \]
     \pause
     \item Помещаем числа $a_{1j}, a_{2j}, \ldots, a_{kj}$ в столбец $j$ таблички.
     \pause
     \item Повторяем шаги 1, 2 и 3 для всех столбцов.
\end{enumerate}

\end{block}

\end{frame}