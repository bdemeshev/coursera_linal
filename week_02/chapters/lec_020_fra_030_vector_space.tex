% !TEX root = ../linal_lecture_02.tex

\begin{frame} % название фрагмента

\videotitle{Векторное пространство}

\end{frame}



\begin{frame}{Краткий план:}
  \begin{itemize}[<+->]
    \item Векторное пространство;
    \item Базис векторного пространства;
    \item Размерность векторного пространства.
  \end{itemize}

\end{frame}


\begin{frame}{Векторное пространство}

\begin{block}{Определение} 
Множество $V$ произвольных объектов называется \alert{конечномерным векторным пространством}, если:

\begin{itemize}[<+->]
\item множество $V$ можно взаимно однозначно сопоставить пространству $\R^n$;
\item определено сложение двух объектов $\ba$ и $\bb$ из $V$, 
и оно соответствует сложению столбцов из $\R^n$;
\item определено умножение объекта $\ba$ из $V$ на число $\lambda\in \R$, 
и оно соответствует умножению столбца $\R^n$ на $\lambda$.
\end{itemize}
\end{block}

Элементы векторного пространства называют \alert{векторами}. 
\pause
Векторное пространство также называют \alert{линейным}.

\end{frame}



\begin{frame}{Многочлены}
Множество $V$ всех многочленов от $t$ степени не выше трёх:
\[
V  = \{ at^3 + bt^2 + ct + d \mid a, b, c, d \in \R \}
\]

\pause
Взаимно однозначное сопоставление: 
\[5t^3 + 6t^2 - 3t + 2 \leftrightarrow \begin{pmatrix} 
    5 \\
    6 \\
    -3 \\
    2 \\
\end{pmatrix}.
\]
\pause


Сложение двух многочленов и умножение многочлена на число соответствуют операциям над столбцами чисел.
\end{frame}




\begin{frame}{Пример векторного пространства}

Множество $V$ всех функций $f(t)$ равных нулю вне двух данных точек:
\[
V  = \{ f \mid f(t)=0 \text{ для всех } t\neq \pm 1 \}
\]

\pause
Взаимно однозначное сопоставление: 
\[f \leftrightarrow \begin{pmatrix}
    f(-1) \\
    f(1) \\
\end{pmatrix}.
\]
\pause


Сложение двух таких функций и умножение на число соответствуют операциям над столбцами чисел.
\end{frame}
    

\begin{frame}{Типичный элемент $V$}
	
	\begin{center}


\begin{tikzpicture}[
scale=2,
MyPoints/.style={draw=blue,fill=blue,thick},
Segments/.style={draw=blue!50!red!70,thick},
MyCircles/.style={green!50!blue!50,thin}, 
every node/.style={scale=1}
]
%\grid;


\clip (-5.5,-2.5) rectangle (6.5,5);




%\draw[->, >=stealth] (-1,0)--(6.5,0) node[right]{$x_1$};
\draw[-{Latex[length=4.5mm, width=2.5mm]}, >=stealth] (0,-1)--(0,4) node[above right]{$f(t)$};

\draw[-{Latex[length=4.5mm, width=2.5mm]}, >=stealth] (-5,0)--(6,0) 
node[below ]{$t$};


%{\verb!->!new, arrowhead = 2mm, line width=4pt}
%, arrowhead = 3mm
%, arrowhead = 0.2

% Feel free to change here coordinates of points A and B
\pgfmathparse{-5}		\let\Xa\pgfmathresult
\pgfmathparse{0}		\let\Ya\pgfmathresult
\coordinate (A) at (\Xa,\Ya);

\pgfmathparse{5.55}		\let\Xb\pgfmathresult
\pgfmathparse{0}		\let\Yb\pgfmathresult
\coordinate (B) at (\Xb,\Yb);

\draw[vecb, line width=0.5mm] (A)--(B);


\fill[MyPoints]  (-2,2) circle (0.8mm);
\fill[MyPoints]  (2,3) circle (0.8mm);


%\path (0,.5) pic[red] {cross=1pt};
\draw (-2,0) node[cross=3mm,blue,line width=1mm] {};
\draw (2,0) node[cross=3mm,blue,line width=1mm] {};


\end{tikzpicture}

\end{center}

\end{frame}



\begin{frame}{Аналогия с $\R^n$}

% Сопоставление абстрактного множества $V$ и множества столбцов $\R^n$ дарит нам:
% \pause
\begin{block}{Определение} 
Вектор $\bc$ называется \alert{линейной комбинацией} векторов $\bv_1$, $\bv_2$, \ldots, $\bv_k$, 
если его можно представить в виде их суммы с некоторыми действительными весами $\alpha_i$:
\[
  \bc = \alpha_1 \bv_1 + \alpha_2 \bv_2 + \ldots + \alpha_k \bv_k
\]
\end{block}

\pause
\begin{block}{Определение} 
Множество векторов $M$, содержащее все возможные линейные комбинации векторов $\bv_1$, 
$\bv_2$, \ldots, $\bv_k$, называется их \alert{линейной оболочкой},
\[
  M = \Span\{ \bv_1, \bv_2, \ldots, \bv_k \}
\]
\end{block}
    
\pause 
Полностью аналогично определяются линейно зависимые и независимые наборы векторов. 

\end{frame}


\begin{frame}{Базис и размерность пространства}

\begin{block}{Определение}
\alert{Базисом векторного пространства} $V$ называется любой набор $\{\be_1, \be_2, \ldots, \be_n\}$,
такой что 
\begin{itemize}
    \item $V = \Span \{\be_1, \be_2, \ldots, \be_n\}$;
    \item векторы $\{\be_1, \be_2, \ldots, \be_n\}$ линейно независимы. 
\end{itemize}
\end{block}

\pause
\begin{block}{Определение}
Число  векторов в базисе, $n$, называют \alert{размерностью пространства} $V$, $\dim V = n$.
\end{block}

\end{frame}


\begin{frame}{Продолжаем аналогию}



Пространство $V$ взаимнооднозначно сопоставлено с $\R^n$ и при этом сложение в $V$ 
соответствует сложению в $\R^n$, 
а умножение на число в $V$ соответствует умножению на число в $\R^n$.
 


\begin{block}{Утверждение}
Линейная независимость в $V$ соответствует линейной независимости в $\R^n$.

Базис в $V$ соответствует базису в $\R^n$.

Размерность $V$ равна размерности $\R^n$, $\dim V = \dim \R^n = n$. 
\end{block}


\end{frame}



\begin{frame}{Формальности}

Мы слишком привыкли к свойствам чисел!

\pause

\vspace{40pt}

\begin{block}{Эквивалентное определение}
Множество $V$ называется \alert{векторным пространством}, если выполнено восемь свойств\ldots
\end{block}

\end{frame}

\begin{frame}{Восемь аксиом: сложение}
\begin{enumerate}
    \item При сложении можно расставлять скобки как хочешь (\alert{ассоциативность}):
    \[
    \ba + (\bb + \bc) = (\ba + \bb) + \bc    
    \]
    \item При сложении можно путать лево и право (\alert{коммутативность}):
    \[
    \ba + \bb = \bb + \ba    
    \]
    \item  Существует \alert{нулевой} вектор $\bzero$:
    \[
    \ba + \bzero = \ba    
    \]
    \item Для любого вектора $\ba$ найдется \alert{противоположный} вектор $-\ba$:
    \[
    \ba + (-\ba) = \bzero    
    \]
\end{enumerate}

\end{frame}


\begin{frame}{Восемь аксиом: умножение}
\begin{enumerate}[resume]
    \item[5.] Умножение вектора на число \alert{совместимо} с умножением чисел:
    \[
    \lambda_1 (\lambda_2\ba) = (\lambda_1 \lambda_2 ) \ba
    \]
    \item[6.]  Умножение на \alert{единицу} не меняет вектор:
    \[
    1\cdot \ba= \ba
    \]
    \item[7.] Раскрывать скобки вокруг векторов можно (\alert{дистрибутивность умножения}):
    \[
    \lambda (\ba + \bb) = \lambda \ba + \lambda \bb
    \]
    \item[8.] Раскрывать скобки вокруг чисел можно (\alert{дистрибутивность умножения}):
    \[
    (\lambda_1 + \lambda_2) \ba = \lambda_1 \ba + \lambda_2 \ba
    \]
\end{enumerate}

\end{frame}