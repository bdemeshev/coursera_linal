% !TEX root = ../linal_lecture_02.tex

\begin{frame} % название фрагмента

\videotitle{Линейная комбинация и независимость}

\end{frame}



\begin{frame}{Краткий план:}
  \begin{itemize}[<+->]
    \item Линейная комбинация векторов;
    \item Зависимые и независимые наборы векторов.
  \end{itemize}

\end{frame}


\begin{frame}{Линейная комбинация}

\begin{block}{Определение} 
Вектор $\bc$ называется \alert{линейной комбинацией} векторов $\bv_1$, $\bv_2$, \ldots, $\bv_k$, 
если его можно представить в виде их суммы с некоторыми действительными весами $\alpha_i$:
\[
  \bc = \alpha_1 \bv_1 + \alpha_2 \bv_2 + \ldots + \alpha_k \bv_k
\]
\end{block}

\pause
Пример. Вектор $\begin{pmatrix}
  4 \\
  5 \\
\end{pmatrix}$ — это линейная комбинация векторов $\begin{pmatrix}
  1 \\
  0 \\
\end{pmatrix}$ и $\begin{pmatrix}
  1 \\
  1 \\
\end{pmatrix}$:

\[
\begin{pmatrix}
  4 \\
  5 \\
\end{pmatrix} = -1 \begin{pmatrix}
  1 \\
  0 \\
\end{pmatrix} + 5 \begin{pmatrix}
  1 \\
  1 \\
\end{pmatrix}  
\]


\end{frame}



\begin{frame}{Линейная комбинация: геометрия}


\begin{center}

\begin{tikzpicture}[
  scale=1.6,
  MyPoints/.style={draw=blue,fill=white,thick},
  Segments/.style={draw=blue!50!red!70,thick},
  MyCircles/.style={green!50!blue!50,thin}, 
  every node/.style={scale=1.2}
  ]
  %\grid;
  \clip (-.5,-.5) rectangle (7.5,6.5);


  %%\draw[->, >=stealth] (-1,0)--(6.5,0) node[right]{$x_1$};
  %\draw[-{Latex[length=4.5mm, width=2.5mm]}, >=stealth] (0,-1)--(0,5) node[above left]{$x_2$};
  %
  %\draw[-{Latex[length=4.5mm, width=2.5mm]}, >=stealth] (-1,0)--(6.5,0) 
  %node[right]{$x_1$};

  % Feel free to change here coordinates of points A and B
  \pgfmathparse{0}		\let\Xa\pgfmathresult
  \pgfmathparse{0}		\let\Ya\pgfmathresult
  \coordinate (A) at (\Xa,\Ya);

  \pgfmathparse{1}		\let\Xb\pgfmathresult
  \pgfmathparse{3}		\let\Yb\pgfmathresult
  \coordinate (B) at (\Xb,\Yb);

  \pgfmathparse{3}		\let\Xc\pgfmathresult
  \pgfmathparse{1}		\let\Yc\pgfmathresult
  \coordinate (C) at (\Xc,\Yc);

  \pgfmathparse{7}		\let\Xd\pgfmathresult
  \pgfmathparse{5}		\let\Yd\pgfmathresult
  \coordinate (D) at (\Xd,\Yd);

  \pgfmathparse{6}		\let\Xe\pgfmathresult
  \pgfmathparse{2}		\let\Ye\pgfmathresult
  \coordinate (E) at (\Xe,\Ye);


  % Let I be the midpoint of [AB]
  \pgfmathparse{(\Xb+\Xa)/2} \let\XI\pgfmathresult
  \pgfmathparse{(\Yb+\Ya)/2} \let\YI\pgfmathresult
  \coordinate (I) at (\XI,\YI);	


  \draw[-{Latex[length=4.5mm, width=2mm]}, >=stealth, vecb,thick] (A)--(B) node[midway,left]{$\bb$};

  \draw[vecb,dashed] (E)--(D);


  \draw[-{Latex[length=4.5mm, width=2mm]}, >=stealth, veca,thick] (A)--(C) node[midway,below]{$\ba$};

  \draw[ veca,dashed] (C)--(E) node[midway,below]{$\ba$};


  \draw[veca,dashed] (B)--(D);


  \draw[-{Latex[length=4.5mm, width=2.5mm]}, >=stealth, vecc,thick] (A)--(D) node[midway,above]{$\bc$};


  \node [above right] at (2, 5) {$\bc = 2 \cdot \ba + 1 \cdot \bb $}; 


  \end{tikzpicture}
    
\end{center}


\end{frame}



\begin{frame}{Любой вектор — линейная комбинация}

Любой вектор $\bv \in \R^2$ — линейная комбинация векторов $\begin{pmatrix}
    1 \\
    0 \\
  \end{pmatrix}$ и $\begin{pmatrix}
    0 \\
    1 \\
  \end{pmatrix}$:

\[
\begin{pmatrix}
  v_1 \\
  v_2 \\
\end{pmatrix} = 
v_1 \begin{pmatrix}
    1 \\
    0 \\
  \end{pmatrix} + 
  v_2 \begin{pmatrix}
    0 \\
    1 \\
  \end{pmatrix}
\]

\pause
Аналогично, любой вектор  $\bv \in \R^3$ представим в виде:

\[
\begin{pmatrix}
v_1 \\
v_2 \\
v_3 \\
\end{pmatrix} = 
v_1 \begin{pmatrix}
  1 \\
  0 \\
  0 \\
\end{pmatrix} + 
v_2 \begin{pmatrix}
  0 \\
  1 \\
  0 \\
\end{pmatrix} +
v_3 \begin{pmatrix}
  0 \\
  0 \\
  1 \\
\end{pmatrix} 
\]


  

\end{frame}



\begin{frame}
\frametitle{Линейная зависимость}


\begin{block}{Определение}
Набор $A$ из двух и более векторов называется 
\alert{линейно зависимым}, если хотя бы один вектор является линейной комбинацией остальных.


Набор $A = \{\bzero\}$ из одного нулевого вектора также называется \alert{линейно зависимым}.
\end{block}


\end{frame}




\begin{frame}{Линейная зависимость: геометрия}

\begin{center}
\begin{tikzpicture}[
scale=1.5,
MyPoints/.style={draw=black,fill=black,thick},
Segments/.style={draw=blue!50!red!70,thick},
MyCircles/.style={green!50!blue!50,thin}, 
every node/.style={scale=1}
]

%\grid;

\clip (-1.5,-1.5) rectangle (5.5,5.5);

\begin{scope}[cm={1,1,1.5,0,(0,0)}]
\draw[draw=blue!30, dashed] (-1.2,-4.2) grid[step=1] (3.5,7);
\end{scope}

%{\verb!->!new, arrowhead = 2mm, line width=4pt}
%, arrowhead = 3mm
%, arrowhead = 0.2

% Feel free to change here coordinates of points A and B
\pgfmathparse{0}		\let\Xa\pgfmathresult
\pgfmathparse{0}		\let\Ya\pgfmathresult
\coordinate (A) at (\Xa,\Ya);

\pgfmathparse{2}		\let\Xb\pgfmathresult
\pgfmathparse{0.5}		\let\Yb\pgfmathresult
\coordinate (B) at (\Xb,\Yb);

\pgfmathparse{2}		\let\Xd\pgfmathresult
\pgfmathparse{4}		\let\Yd\pgfmathresult
\coordinate (D) at (\Xd,\Yd);

\pgfmathparse{4}		\let\Xc\pgfmathresult
\pgfmathparse{0}		\let\Yc\pgfmathresult
\coordinate (C) at (\Xc,\Yc);


\pgfmathparse{1}		\let\Xe\pgfmathresult
\pgfmathparse{1}		\let\Ye\pgfmathresult
\coordinate (E) at (\Xe,\Ye);

\pgfmathparse{2.5}		\let\Xf\pgfmathresult
\pgfmathparse{0}		\let\Yf\pgfmathresult
\coordinate (F) at (\Xf,\Yf);

\pgfmathparse{4}		\let\Xg\pgfmathresult
\pgfmathparse{1}		\let\Yg\pgfmathresult
\coordinate (G) at (\Xg,\Yg);




\draw[-{Latex[length=4.5mm, width=2.5mm]}, >=stealth, thick] (A)--(D) node[above left]{$\bd$};

\draw[-{Latex[length=4.5mm, width=2.5mm]}, >=stealth, vecb, thick] (A)--(E) node[right]{$\ba$};

\draw[-{Latex[length=4.5mm, width=2.5mm]}, >=stealth, vecb, thick] (A)--(F) node[below]{$\bb$};

\draw[-{Latex[length=4.5mm, width=2.5mm]}, >=stealth, veca, thick] (A)--(G) node[right]{$\bc$};


\draw[black, dashed] (B)--(D);

\fill[MyPoints]  (0,0) circle (0.8mm);

%\node [right,darkgray] at (-0.5,-2) {$\{\ba, \bb, \bc\}$ — линейно зависимы}; 

%\node [right,darkgray] at (-0.5,-3) {$\{\ba, \bb, \bd\}$ — независимы}; 



\end{tikzpicture}

\end{center}

Набор $\{\ba, \bb, \bc \}$ — линейно зависим.

Набор $\{\ba, \bb, \bd \}$ — линейно независим.

\end{frame}  









\begin{frame}
\frametitle{Линейная зависимость: примеры}



Набор $A = \left\{ \begin{pmatrix}
      0 \\
      2 \\
    \end{pmatrix}, \begin{pmatrix}
      3 \\
      4 \\
    \end{pmatrix} \right\}$ — линейно независимый.

\pause

Набор $B = \left\{ \begin{pmatrix}
      0 \\
      2 \\
      0 \\
    \end{pmatrix}, \begin{pmatrix}
      3 \\
      4 \\
      0 \\
    \end{pmatrix},
    \begin{pmatrix}
      1 \\
      0 \\
      0 \\
    \end{pmatrix} \right\}$ — линейно зависимый:

    \[
      \begin{pmatrix}
        3 \\
        4 \\
        0 \\
      \end{pmatrix} = 2
    \begin{pmatrix}
      0 \\
      2 \\
      0 \\
    \end{pmatrix} + 3
    \begin{pmatrix}
      1 \\
      0 \\
      0 \\
    \end{pmatrix}  
    \]
  

\end{frame}


\begin{frame}
  \frametitle{Линейная зависимость: дубль два}

\begin{block}{Эквивалентное определение} Набор векторов $A = \{ \bv_1, \bv_2, \ldots, \bv_k\}$ называется \alert{линейно зависимым},
  если можно найти такие веса $\alpha_1$, $\alpha_2$, \ldots, $\alpha_k$, что
  \[
  \alpha_1 \bv_1 + \alpha_2 \bv_2 + \ldots + \alpha_k \bv_k = \bzero,  
  \]
  и при этом хотя бы одно из чисел $\alpha_i$ отлично от $0$. 
\end{block}

\pause

\begin{block}{Доказательство эквивалентности}
Вектор с ненулевым коэффициентом $\alpha_i$ перед ним можно выразить через остальные. 
\pause

Если вектор $\bv_2$ выражен через $\bv_1$ и $\bv_3$, $\bv_2 = \alpha_1 \bv_1 + \alpha_3 \bv_3$, 
то искомая нулевая линейная комбинация имеет вид: $\alpha_1 \bv_1 +(-1)\bv_2 + \alpha_3 \bv_3=\bzero$.
\end{block}

\end{frame}