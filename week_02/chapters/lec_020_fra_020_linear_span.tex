% !TEX root = ../linal_lecture_02.tex

\begin{frame} % название фрагмента

\videotitle{Линейная оболочка}

\end{frame}



\begin{frame}{Краткий план:}
  \begin{itemize}[<+->]
    \item Линейная комбинация векторов;
    \item Зависимые и независимые наборы векторов.
  \end{itemize}

\end{frame}


\begin{frame}{Линейная комбинация}

\begin{block}{Определение} 
Вектор $\bv$ называется \alert{линейной комбинацией} векторов $\bx_1$, $\bx_2$, \ldots, $\bx_k$, 
если его можно представить в виде их суммы с некоторыми действительными весами $\alpha_i$:
\[
  \bv = \alpha_1 \bx_1 + \alpha_2 \bx_2 + \ldots + \alpha_k \bx_k
\]
\end{block}

\pause
Пример. Вектор $\begin{pmatrix}
  4 \\
  5 \\
\end{pmatrix}$ — это линейная комбинация векторов $\begin{pmatrix}
  1 \\
  0 \\
\end{pmatrix}$ и $\begin{pmatrix}
  1 \\
  1 \\
\end{pmatrix}$:

\[
\begin{pmatrix}
  4 \\
  5 \\
\end{pmatrix} = -1 \begin{pmatrix}
  1 \\
  0 \\
\end{pmatrix} + 5 \begin{pmatrix}
  1 \\
  1 \\
\end{pmatrix}  
\]


\end{frame}



\begin{frame}{Любой вектор — линейная комбинация}

Любой вектор $\bv \in \R^2$ — линейная комбинация векторов $\begin{pmatrix}
    1 \\
    0 \\
  \end{pmatrix}$ и $\begin{pmatrix}
    0 \\
    1 \\
  \end{pmatrix}$:

\[
\begin{pmatrix}
  v_1 \\
  v_2 \\
\end{pmatrix} = 
v_1 \begin{pmatrix}
    1 \\
    0 \\
  \end{pmatrix} + 
  v_2 \begin{pmatrix}
    0 \\
    1 \\
  \end{pmatrix}
\]

\pause
Аналогично, любой вектор  $\bv \in \R^3$ представим в виде:

\[
\begin{pmatrix}
v_1 \\
v_2 \\
v_3 \\
\end{pmatrix} = 
v_1 \begin{pmatrix}
  1 \\
  0 \\
  0 \\
\end{pmatrix} + 
v_2 \begin{pmatrix}
  0 \\
  1 \\
  0 \\
\end{pmatrix} +
v_3 \begin{pmatrix}
  0 \\
  0 \\
  1 \\
\end{pmatrix} 
\]


  

\end{frame}



\begin{frame}
\frametitle{Линейная зависимость}


\begin{block}{Определение}
Набор $A$ из двух и более векторов называется 
\alert{линейно зависимым}, если хотя бы один вектор является линейной комбинацией остальных.


Набор $A = \{\bzero\}$ из одного нулевого вектора также называется \alert{линейно зависимым}.
\end{block}


\end{frame}


\begin{frame}
\frametitle{Линейная зависимость: примеры}



Набор $A = \left\{ \begin{pmatrix}
      0 \\
      2 \\
    \end{pmatrix}, \begin{pmatrix}
      3 \\
      4 \\
    \end{pmatrix} \right\}$ — линейно независимый.

\pause

Набор $A = \left\{ \begin{pmatrix}
      0 \\
      2 \\
      0 \\
    \end{pmatrix}, \begin{pmatrix}
      3 \\
      4 \\
      0 \\
    \end{pmatrix},
    \begin{pmatrix}
      1 \\
      0 \\
      0 \\
    \end{pmatrix} \right\}$ — линейно зависимый:

    \[
      \begin{pmatrix}
        3 \\
        4 \\
        0 \\
      \end{pmatrix} = 2
    \begin{pmatrix}
      0 \\
      2 \\
      0 \\
    \end{pmatrix} + 3
    \begin{pmatrix}
      1 \\
      0 \\
      0 \\
    \end{pmatrix}  
    \]
  

\end{frame}


\begin{frame}
  \frametitle{Линейная зависимость: дубль два}

\begin{block}{Эквивалентное пределение} Набор векторов $A = \{ \bv_1, \bv_2, \ldots, \bv_k\}$ называется \alert{линейно зависимым},
  если можно найти такие веса $\alpha_1$, $\alpha_2$, \ldots, $\alpha_k$, что
  \[
  \alpha_1 \bv_1 + \alpha_2 \bv_2 + \ldots + \alpha_k \bv_k = \bzero,  
  \]
  и при этом хотя бы одно из чисел $\alpha_i$ отлично от $0$. 
\end{block}

\pause

\begin{block}{Доказательство эквивалентности}
Вектор с ненулевым коэффициентом $\alpha_i$ перед ним можно выразить через остальные. 
\pause

Если вектор $\bv_2$ выражен через $\bv_1$ и $\bv_3$, $\bv_2 = \alpha_1 \bv_1 + \alpha_3 \bv_3$, 
то искомая нулевая линейная комбинация имеет вид: $\alpha_1 \bv_1 +(-1)\bv_2 + \alpha_3 \bv_3=\bzero$.
\end{block}

\end{frame}