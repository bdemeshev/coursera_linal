% !TEX root = ../linal_lecture_02.tex

\begin{frame} % название фрагмента

\videotitle{Линейная оболочка}

\end{frame}



\begin{frame}{Краткий план:}
  \begin{itemize}[<+->]
    \item Линейная оболочка векторов;
    \item Базис линейной оболочки векторов;
    \item Размерность линейной оболочки векторов.
  \end{itemize}

\end{frame}


\begin{frame}{Линейная оболочка}

\begin{block}{Определение} 
Множество векторов $M$, содержащее все возможные линейные комбинации векторов $\bv_1$, 
$\bv_2$, \ldots, $\bv_k$, называется их \alert{линейной оболочкой},
\[
  M = \Span\{ \bv_1, \bv_2, \ldots, \bv_k \}
\]
\end{block}

\end{frame}



\begin{frame}{Линейная оболочка векторов: картинка}

\begin{center}

  \begin{tikzpicture}[
    scale=1.5,
    MyPoints/.style={draw=black,fill=black,thick},
    Segments/.style={draw=blue!50!red!70,thick},
    MyCircles/.style={green!50!blue!50,thin}, 
    every node/.style={scale=1}
    ]

    %\grid;

    \clip (-1.5,-1.5) rectangle (5.5,5.5);

    \begin{scope}[cm={1,1,1.5,0,(0,0)}]
    \draw[draw=blue!30, dashed] (-1.2,-4.2) grid[step=1] (3.5,7);
    \end{scope}

    %{\verb!->!new, arrowhead = 2mm, line width=4pt}
    %, arrowhead = 3mm
    %, arrowhead = 0.2

    % Feel free to change here coordinates of points A and B
    \pgfmathparse{0}		\let\Xa\pgfmathresult
    \pgfmathparse{0}		\let\Ya\pgfmathresult
    \coordinate (A) at (\Xa,\Ya);

    \pgfmathparse{2}		\let\Xb\pgfmathresult
    \pgfmathparse{0.5}		\let\Yb\pgfmathresult
    \coordinate (B) at (\Xb,\Yb);

    \pgfmathparse{2}		\let\Xd\pgfmathresult
    \pgfmathparse{4}		\let\Yd\pgfmathresult
    \coordinate (D) at (\Xd,\Yd);

    \pgfmathparse{4}		\let\Xc\pgfmathresult
    \pgfmathparse{0}		\let\Yc\pgfmathresult
    \coordinate (C) at (\Xc,\Yc);


    \pgfmathparse{1}		\let\Xe\pgfmathresult
    \pgfmathparse{1}		\let\Ye\pgfmathresult
    \coordinate (E) at (\Xe,\Ye);

    \pgfmathparse{2.5}		\let\Xf\pgfmathresult
    \pgfmathparse{0}		\let\Yf\pgfmathresult
    \coordinate (F) at (\Xf,\Yf);

    \pgfmathparse{4}		\let\Xg\pgfmathresult
    \pgfmathparse{1}		\let\Yg\pgfmathresult
    \coordinate (G) at (\Xg,\Yg);




    \draw[-{Latex[length=4.5mm, width=2.5mm]}, >=stealth, thick] (A)--(D) node[above left]{$\bd$};

    \draw[-{Latex[length=4.5mm, width=2.5mm]}, >=stealth, vecb, thick] (A)--(E) node[right]{$\ba$};

    \draw[-{Latex[length=4.5mm, width=2.5mm]}, >=stealth, vecb, thick] (A)--(F) node[below]{$\bb$};

    \draw[-{Latex[length=4.5mm, width=2.5mm]}, >=stealth, veca, thick] (A)--(G) node[right]{$\bc$};


    \draw[black, dashed] (B)--(D);

    \fill[MyPoints]  (0,0) circle (0.8mm);

    %\node [right,darkgray] at (0.5,-2) {$\bc \in \operatorname{Lin} (\ba, \bb) $ }; 

    %\node [right,darkgray] at (0.5,-3) {$\bd \notin \operatorname{Lin} (\ba, \bb) $ }; 



    \end{tikzpicture}
  \end{center}
  
Вектор $\bc$ лежит в плоскости $\Span\{ \ba, \bb \}$.

Вектор $\bd$ не лежит в плоскости $\Span\{ \ba, \bb \}$.


\end{frame}



\begin{frame}
\frametitle{Базис линейной оболочки}


\begin{block}{Определение}
Набор векторов $A = \{\bv_1, \bv_2, \ldots, \bv_d \}$ называется
\alert{базисом линейной оболочки} $\Span\{\bx_1, \bx_2, \ldots, \bx_k \}$,
если:
\begin{itemize}
  \item $\Span\{\bv_1, \bv_2, \ldots, \bv_d \} =  \Span\{\bx_1, \bx_2, \ldots, \bx_k \}$;
  \item Набор векторов $A$ линейно независим.
\end{itemize}
\end{block}

\end{frame}





\begin{frame}{Базис линейной оболочки: картинка}

  \begin{center}
  
    \begin{tikzpicture}[
      scale=1.5,
      MyPoints/.style={draw=black,fill=black,thick},
      Segments/.style={draw=blue!50!red!70,thick},
      MyCircles/.style={green!50!blue!50,thin}, 
      every node/.style={scale=1}
      ]
  
      %\grid;
  
      \clip (-1.5,-1.5) rectangle (5.5,5.5);
  
      \begin{scope}[cm={1,1,1.5,0,(0,0)}]
      \draw[draw=blue!30, dashed] (-1.2,-4.2) grid[step=1] (3.5,7);
      \end{scope}
  
      %{\verb!->!new, arrowhead = 2mm, line width=4pt}
      %, arrowhead = 3mm
      %, arrowhead = 0.2
  
      % Feel free to change here coordinates of points A and B
      \pgfmathparse{0}		\let\Xa\pgfmathresult
      \pgfmathparse{0}		\let\Ya\pgfmathresult
      \coordinate (A) at (\Xa,\Ya);
  
      \pgfmathparse{2}		\let\Xb\pgfmathresult
      \pgfmathparse{0.5}		\let\Yb\pgfmathresult
      \coordinate (B) at (\Xb,\Yb);
  
      \pgfmathparse{2}		\let\Xd\pgfmathresult
      \pgfmathparse{4}		\let\Yd\pgfmathresult
      \coordinate (D) at (\Xd,\Yd);
  
      \pgfmathparse{4}		\let\Xc\pgfmathresult
      \pgfmathparse{0}		\let\Yc\pgfmathresult
      \coordinate (C) at (\Xc,\Yc);
  
  
      \pgfmathparse{1}		\let\Xe\pgfmathresult
      \pgfmathparse{1}		\let\Ye\pgfmathresult
      \coordinate (E) at (\Xe,\Ye);
  
      \pgfmathparse{2.5}		\let\Xf\pgfmathresult
      \pgfmathparse{0}		\let\Yf\pgfmathresult
      \coordinate (F) at (\Xf,\Yf);
  
      \pgfmathparse{4}		\let\Xg\pgfmathresult
      \pgfmathparse{1}		\let\Yg\pgfmathresult
      \coordinate (G) at (\Xg,\Yg);
  
  
  
  
      \draw[-{Latex[length=4.5mm, width=2.5mm]}, >=stealth, thick] (A)--(D) node[above left]{$\bd$};
  
      \draw[-{Latex[length=4.5mm, width=2.5mm]}, >=stealth, vecb, thick] (A)--(E) node[right]{$\ba$};
  
      \draw[-{Latex[length=4.5mm, width=2.5mm]}, >=stealth, vecb, thick] (A)--(F) node[below]{$\bb$};
  
      \draw[-{Latex[length=4.5mm, width=2.5mm]}, >=stealth, veca, thick] (A)--(G) node[right]{$\bc$};
  
  
      \draw[black, dashed] (B)--(D);
  
      \fill[MyPoints]  (0,0) circle (0.8mm);
  
      %\node [right,darkgray] at (0.5,-2) {$\bc \in \operatorname{Lin} (\ba, \bb) $ }; 
  
      %\node [right,darkgray] at (0.5,-3) {$\bd \notin \operatorname{Lin} (\ba, \bb) $ }; 
  
  
  
      \end{tikzpicture}
    \end{center}
    
  Для линейной оболочки $\Span\{ \ba, \bb, \bc\}$ базисами будут $A_1 = \{ \ba, \bb\}$, 
  $A_2= \{ \bb, 2\bc\}$, $A_3 = \{ 3\ba, 5\bc\}$.
  

  \end{frame}
  






\begin{frame}
\frametitle{Базис оболочки: примеры}

Рассмотрим линейную оболочку $M = \Span\left\{
\begin{pmatrix}
  1 \\
  1 \\
\end{pmatrix}, \, 
\begin{pmatrix}
  3 \\
  0 \\
\end{pmatrix},  \,
\begin{pmatrix}
  0 \\
  4 \\
\end{pmatrix}  
\right\}$

\pause

Набор $A = \left\{ \begin{pmatrix}
      0 \\
      2 \\
    \end{pmatrix}, \begin{pmatrix}
      3 \\
      4 \\
    \end{pmatrix} \right\}$ — базис для $M$.

\pause

Набор $B = \left\{ \begin{pmatrix}
  1 \\
  0 \\
\end{pmatrix}, \begin{pmatrix}
  7 \\
  -4 \\
\end{pmatrix} \right\}$ — базис для $M$.

\end{frame}



\begin{frame}{Зачем нужен базис?}

\begin{block}{Утверждение}
Если $\{\bv_1, \bv_2, \ldots, \bv_d\}$ — базис линейной оболочки $M$, 
то любой вектор $\bx\in M$ \alert{единственным} образом представим в виде 
\[
\bx  = \alpha_1 \bv_1 + \ldots + \alpha_d \bv_d
\]
\end{block}

\end{frame}


\begin{frame}{Зачем нужен базис?}


\begin{block}{Доказательство}
Линейная комбинация базиса совпадает с $M$, значит любой вектор из $M$ представим
как линейная комбинация элементов базиса.
\pause

Если бы для некоторого $\bx$ нашлось два различных представления
\[
  \alpha_1 \bv_1 + \ldots + \alpha_d \bv_d = \alpha_1' \bv_1 + \ldots + \alpha_d' \bv_d,
\]
то была бы зависимость между элементами базиса, что невозможно. 
\end{block}

\end{frame}


\begin{frame}{Свойства базиса линейной оболочки}

\begin{block}{Утверждение}
Если набор векторов $A =\{ \bv_1, \bv_2, \ldots, \bv_k\}$ линейно независим, то он сам
является базисом своей линейной оболочки $\Span  \{ \bv_1, \bv_2, \ldots, \bv_k\}$.
\end{block}


\pause
\begin{block}{Утверждение}
Если наборы векторов $A$ и $B$ — являются базисами для линейной оболочки $M$,
то наборы $A$ и $B$ содержат одинаковое количество векторов. 
\end{block}
  
\pause
\begin{block}{Утверждение}
Если набор $A$ содержит $k$ векторов, 
то базис линейной оболочки $\Span A$ содержит $k$ элементов или меньше. 
\end{block}
  


\end{frame}


\begin{frame}{Размерность линейной оболочки}

\begin{block}{Определение}
  Если базис линейной оболочки $M$ содержит $d$ элементов, то 
  число $d$ называется \alert{размерностью линейной оболочки} $M$.
  \[
  d = \dim M  
  \] 
\end{block}
    

\end{frame}



\begin{frame}{Размерность линейной оболочки: картинка}

\begin{center}

  \begin{tikzpicture}[
    scale=1.5,
    MyPoints/.style={draw=black,fill=black,thick},
    Segments/.style={draw=blue!50!red!70,thick},
    MyCircles/.style={green!50!blue!50,thin}, 
    every node/.style={scale=1}
    ]

    %\grid;

    \clip (-1.5,-1.5) rectangle (5.5,5.5);

    \begin{scope}[cm={1,1,1.5,0,(0,0)}]
    \draw[draw=blue!30, dashed] (-1.2,-4.2) grid[step=1] (3.5,7);
    \end{scope}

    %{\verb!->!new, arrowhead = 2mm, line width=4pt}
    %, arrowhead = 3mm
    %, arrowhead = 0.2

    % Feel free to change here coordinates of points A and B
    \pgfmathparse{0}		\let\Xa\pgfmathresult
    \pgfmathparse{0}		\let\Ya\pgfmathresult
    \coordinate (A) at (\Xa,\Ya);

    \pgfmathparse{2}		\let\Xb\pgfmathresult
    \pgfmathparse{0.5}		\let\Yb\pgfmathresult
    \coordinate (B) at (\Xb,\Yb);

    \pgfmathparse{2}		\let\Xd\pgfmathresult
    \pgfmathparse{4}		\let\Yd\pgfmathresult
    \coordinate (D) at (\Xd,\Yd);

    \pgfmathparse{4}		\let\Xc\pgfmathresult
    \pgfmathparse{0}		\let\Yc\pgfmathresult
    \coordinate (C) at (\Xc,\Yc);


    \pgfmathparse{1}		\let\Xe\pgfmathresult
    \pgfmathparse{1}		\let\Ye\pgfmathresult
    \coordinate (E) at (\Xe,\Ye);

    \pgfmathparse{2.5}		\let\Xf\pgfmathresult
    \pgfmathparse{0}		\let\Yf\pgfmathresult
    \coordinate (F) at (\Xf,\Yf);

    \pgfmathparse{4}		\let\Xg\pgfmathresult
    \pgfmathparse{1}		\let\Yg\pgfmathresult
    \coordinate (G) at (\Xg,\Yg);




    \draw[-{Latex[length=4.5mm, width=2.5mm]}, >=stealth, thick] (A)--(D) node[above left]{$\bd$};

    \draw[-{Latex[length=4.5mm, width=2.5mm]}, >=stealth, vecb, thick] (A)--(E) node[right]{$\ba$};

    \draw[-{Latex[length=4.5mm, width=2.5mm]}, >=stealth, vecb, thick] (A)--(F) node[below]{$\bb$};

    \draw[-{Latex[length=4.5mm, width=2.5mm]}, >=stealth, veca, thick] (A)--(G) node[right]{$\bc$};


    \draw[black, dashed] (B)--(D);

    \fill[MyPoints]  (0,0) circle (0.8mm);

    %\node [right,darkgray] at (0.5,-2) {$\bc \in \operatorname{Lin} (\ba, \bb) $ }; 

    %\node [right,darkgray] at (0.5,-3) {$\bd \notin \operatorname{Lin} (\ba, \bb) $ }; 



    \end{tikzpicture}
  \end{center}

\pause
Размерность $\Span\{ \ba, \bb, \bc\}$ равна 2.

\pause
Размерность $\Span\{ \ba, \bb, \bd\}$ равна 3.
\end{frame}


\begin{frame}{Пространство $\R^n$}
\begin{block}{Определение} 
\alert{Пространство $\R^n$} — множество всех возможных векторов из $n$ чисел. 
 \[
 \R^n = \left\{ \begin{pmatrix}
 x_1 \\
 x_2 \\
 \vdots \\
 x_n \\
 \end{pmatrix} \middle| x_1 \in \R, \ldots, x_n \in \R
   \right\}  
 \]
\end{block}
\pause
\[
 \R^n = \Span\left\{
\begin{pmatrix}
1 \\
0 \\
\vdots \\
0 \\
\end{pmatrix}, \,
\begin{pmatrix}
0 \\
1 \\
\vdots \\
0 \\
\end{pmatrix}, \, \ldots, \,
\begin{pmatrix}
0 \\
0\\
\vdots \\
1 \\
\end{pmatrix}
\right\}
\]

\pause
Размерность $\R^n$ равна $n$. 


\end{frame}