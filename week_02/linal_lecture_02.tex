\documentclass[14 pt,xcolor=dvipsnames]{beamer}

% !TEX root = linal_lecture_01.tex

\usepackage{epsdice}

\usepackage[absolute,overlay]{textpos}

\usepackage[orientation=portrait,size=custom,width=25.4,height=19.05]{beamerposter}

%25,4 см 19,05 см размеры слайда в powerpoint

\usetheme{metropolis}
\metroset{
  %progressbar=none,
  numbering=none,
  subsectionpage=progressbar,
  block=fill
}

%\usecolortheme{seahorse}

\usepackage{fontspec}
\usepackage{polyglossia}
\setmainlanguage{russian}


\usepackage{fontawesome5} % removed [fixed]
\setmainfont[Ligatures=TeX]{Myriad Pro}
\setsansfont{Myriad Pro}

% why do we need \newfontfamily:
% http://tex.stackexchange.com/questions/91507/
\newfontfamily{\cyrillicfonttt}{Myriad Pro}
\newfontfamily{\cyrillicfont}{Myriad Pro}
%\newfontfamily{\cyrillicfontbs}{Myriad Pro}
\newfontfamily{\cyrillicfontsf}{Myriad Pro}


% https://tex.stackexchange.com/questions/175860/why-does-unicode-math-break-the-kerning-of-accents-in-combination-with-amssymb
% "You shouldn't be using amssymb together with unicode-math"
\usepackage{amsmath,amsxtra,amsthm} % amssymb

%\usepackage{bm}

\usepackage{fdsymbol} % \nperp

\usepackage{unicode-math}


\usepackage{centernot}

\usepackage{graphicx}
\graphicspath{{img/}}

\usepackage{wrapfig}
\usepackage{animate}
\usepackage{tikz}
%\usetikzlibrary{shapes.geometric,patterns,positioning,matrix,calc,arrows,shapes,fit,decorations,decorations.pathmorphing}
\usepackage{pifont}
\usepackage{comment}
\usepackage[font=small,labelfont=bf]{caption}
\captionsetup[figure]{labelformat=empty}
\includecomment{techno}

\usefonttheme[onlymath]{serif}


%Расположение

\setbeamersize{text margin left=15 mm,text margin right=5mm} 
\setlength{\leftmargini}{38 pt}

%\usepackage{showframe}
%\usepackage{enumitem}
%\setlist{leftmargin=5.5mm}


%Цвета от дирекции

\definecolor{dirblack}{RGB}{58, 58, 58}
\definecolor{dirwhite}{RGB}{245, 245, 245}
\definecolor{dirred}{RGB}{149, 55, 53}
\definecolor{dirblue}{RGB}{0, 90, 171}
\definecolor{dirorange}{RGB}{235, 143, 76}
\definecolor{dirlightblue}{RGB}{75, 172, 198}
\definecolor{dirgreen}{RGB}{155, 187, 89}
\definecolor{dircomment}{RGB}{128, 100, 162}

\setbeamercolor{title separator}{bg=dirlightblue!50, fg=dirblue}

%Цвета блоков

% Голубой блок!
\setbeamercolor{block title}{bg=dirblue!30,fg=dirblack}
\setbeamercolor{block title example}{bg=dirlightblue!50,fg=dirblack}
\setbeamercolor{block body example}{bg=dirlightblue!20,fg=dirblack}

\AtBeginEnvironment{exampleblock}{\setbeamercolor{itemize item}{fg=dirblack}}
%\setbeamertemplate{blocks}[rounded][shadow]

% Набор команд для удобства верстки

\newcommand{\RR}{\mathbb{R}}
\newcommand{\ZZ}{\mathbb{Z}}
\newcommand{\la}{\lambda}

% Набор команд для структуризации

%\newcommand{\quest}{\faQuestionCircleO}
%\faPencilSquareO \faPuzzlePiece \faQuestionCircleO  \faIcon*[regular]{file} {\textcolor{dirblue}
%\newcommand{\quest}{\textcolor{dirblue}{\boxed{\textbf{?}}}
\newcommand{\task}{\faIcon{tasks}}
\newcommand{\exmpl}{\faPuzzlePiece}
\newcommand{\dfn}{\faIcon{pen-square}}
\newcommand{\quest}{\textcolor{dirblue}{\faQuestionCircle[regular]}}
\newcommand{\acc}[1]{\textcolor{dirred}{#1}}
\newcommand{\accm}[1]{\textcolor{dirred}{#1}}
\newcommand{\acct}[1]{\textcolor{dirblue}{#1}}
\newcommand{\acctm}[1]{\textcolor{dirblue}{#1}}
\newcommand{\accex}[1]{\textcolor{dirblack}{\bf #1}}
\newcommand{\accexm}[1]{\textcolor{dirblack}{ \mathbf{#1}}}
\newcommand{\acclp}[1]{\textcolor{dirorange}{\it #1}}
\newcommand{\todo}[1]{\textcolor{dircomment}{\bf #1}}
\newcommand{\graylink}[1]{{\fontsize{11}{12}\selectfont \textcolor{gray}{#1}}}
\newcommand{\figcaption}[1]{{\fontsize{18}{20}\selectfont #1}}


\newcommand{\videotitle}[1]{
    {\fontsize{33}{30}\selectfont \textcolor{dirblue}{\textbf{#1}} }

    %\todo{название видеофрагмента}
}

\newcommand{\lecturetitle}[1]{
  {\fontsize{33}{30}\selectfont \textcolor{dirblue}{\textbf{#1}} }

    %\todo{название лекции}
}





\newcommand{\spcbig}{\vspace{-10 pt}}
\newcommand{\spcsmall}{\vspace{-5 pt}}

%\usepackage{listings}
%\lstset{
%xleftmargin=0 pt,
%  basicstyle=\small, 
%  language=Python,
  %tabsize = 2,
%  backgroundcolor=\color{mc!20!white}
%}



%\newcommand{\mypart}[1]{\begin{frame}[standout]{\huge #1}\end{frame}}

\setbeamercolor{background canvas}{bg=}

% frame title setup
\setbeamercolor{frametitle}{bg=,fg=dirblue}
\setbeamertemplate{frametitle}[default][left]

\addtobeamertemplate{frametitle}{\hspace*{-0.5 cm}}{\vspace*{0.25cm}}


%Шрифты
\setbeamerfont{frametitle}{family=\rmfamily,series=\bfseries,size={\fontsize{33}{30}}}
\setbeamerfont{framesubtitle}{family=\rmfamily,series=\bfseries,size={\fontsize{26}{20}}}


% удобнее знать номер слайда, чтобы вносить правки!  

\setbeamercolor{footline}{fg=dircomment}
\setbeamerfont{footline}{series=\bfseries, size={\fontsize{12}{14}}}
%\setbeamertemplate{footline}[page number]

\defbeamertemplate{footline}{custom footline}
{%
  \hspace*{\fill}%
  \usebeamercolor[fg]{page number in head/foot}%
  \usebeamerfont{page number in head/foot}%
  page: \insertpagenumber\,/\,\insertpresentationendpage%
  \hspace{20pt}%
  slide: \insertframenumber\,/\,\inserttotalframenumber%
  %\hspace*{\fill}
  \vskip2pt%
}
\setbeamertemplate{footline}[custom footline]

\usepackage{physics}
\newcommand{\R}{\mathbb{R}}
\newcommand{\Rot}{\mathrm{R}}
\newcommand{\HH}{\mathrm{H}}
\newcommand{\Id}{\mathrm{I}}


\usepackage[outline]{contour}


\usepackage{pgfplots}
\pgfplotsset{compat=newest}

\usepackage{tikz}
\usetikzlibrary{calc}
\usetikzlibrary{quotes,angles}
\usetikzlibrary{arrows}
\usetikzlibrary{arrows.meta}
\usetikzlibrary{positioning,intersections,decorations.markings}
\usetikzlibrary{patterns}

\usepackage{tkz-euclide} 

\newcommand{\grid}{\draw[color=gray,step=1.0,dotted] (-2.1,-2.1) grid (9.6,6.1)}

\newcommand{\ba}{\symbf{a}}
\newcommand{\be}{\symbf{e}}
\newcommand{\bb}{\symbf{b}}
\newcommand{\bc}{\symbf{c}}
\newcommand{\bd}{\symbf{d}}
\newcommand{\bx}{\symbf{x}}
\newcommand{\bv}{\symbf{v}}
\newcommand{\bzero}{\symbf{0}}


\DeclareMathOperator{\Lin}{Span}

\DeclareMathOperator{\Span}{Span}
\DeclareMathOperator{\LL}{L}

%\tikzset{>=latex}

\colorlet{veca}{red}
\colorlet{vecb}{blue}
\colorlet{vecc}{olive}







\begin{document}


\begin{frame} % название лекции


\lecturetitle{Матричная запись}

\end{frame}


% !TEX root = ../linal_lecture_02.tex

\begin{frame} % название фрагмента

\videotitle{Линейная комбинация и независимость}

\end{frame}



\begin{frame}{Краткий план:}
  \begin{itemize}[<+->]
    \item Линейная комбинация векторов;
    \item Зависимые и независимые наборы векторов.
  \end{itemize}

\end{frame}


\begin{frame}{Линейная комбинация}

\begin{block}{Определение} 
Вектор $\bc$ называется \alert{линейной комбинацией} векторов $\bv_1$, $\bv_2$, \ldots, $\bv_k$, 
если его можно представить в виде их суммы с некоторыми действительными весами $\alpha_i$:
\[
  \bc = \alpha_1 \bv_1 + \alpha_2 \bv_2 + \ldots + \alpha_k \bv_k
\]
\end{block}

\pause
Пример. Вектор $\begin{pmatrix}
  4 \\
  5 \\
\end{pmatrix}$ — это линейная комбинация векторов $\begin{pmatrix}
  1 \\
  0 \\
\end{pmatrix}$ и $\begin{pmatrix}
  1 \\
  1 \\
\end{pmatrix}$:

\[
\begin{pmatrix}
  4 \\
  5 \\
\end{pmatrix} = -1 \begin{pmatrix}
  1 \\
  0 \\
\end{pmatrix} + 5 \begin{pmatrix}
  1 \\
  1 \\
\end{pmatrix}  
\]


\end{frame}



\begin{frame}{Линейная комбинация: геометрия}


\begin{center}

\begin{tikzpicture}[
  scale=1.6,
  MyPoints/.style={draw=blue,fill=white,thick},
  Segments/.style={draw=blue!50!red!70,thick},
  MyCircles/.style={green!50!blue!50,thin}, 
  every node/.style={scale=1.2}
  ]
  %\grid;
  \clip (-.5,-.5) rectangle (7.5,6.5);


  %%\draw[->, >=stealth] (-1,0)--(6.5,0) node[right]{$x_1$};
  %\draw[-{Latex[length=4.5mm, width=2.5mm]}, >=stealth] (0,-1)--(0,5) node[above left]{$x_2$};
  %
  %\draw[-{Latex[length=4.5mm, width=2.5mm]}, >=stealth] (-1,0)--(6.5,0) 
  %node[right]{$x_1$};

  % Feel free to change here coordinates of points A and B
  \pgfmathparse{0}		\let\Xa\pgfmathresult
  \pgfmathparse{0}		\let\Ya\pgfmathresult
  \coordinate (A) at (\Xa,\Ya);

  \pgfmathparse{1}		\let\Xb\pgfmathresult
  \pgfmathparse{3}		\let\Yb\pgfmathresult
  \coordinate (B) at (\Xb,\Yb);

  \pgfmathparse{3}		\let\Xc\pgfmathresult
  \pgfmathparse{1}		\let\Yc\pgfmathresult
  \coordinate (C) at (\Xc,\Yc);

  \pgfmathparse{7}		\let\Xd\pgfmathresult
  \pgfmathparse{5}		\let\Yd\pgfmathresult
  \coordinate (D) at (\Xd,\Yd);

  \pgfmathparse{6}		\let\Xe\pgfmathresult
  \pgfmathparse{2}		\let\Ye\pgfmathresult
  \coordinate (E) at (\Xe,\Ye);


  % Let I be the midpoint of [AB]
  \pgfmathparse{(\Xb+\Xa)/2} \let\XI\pgfmathresult
  \pgfmathparse{(\Yb+\Ya)/2} \let\YI\pgfmathresult
  \coordinate (I) at (\XI,\YI);	


  \draw[-{Latex[length=4.5mm, width=2mm]}, >=stealth, vecb,thick] (A)--(B) node[midway,left]{$\bb$};

  \draw[vecb,dashed] (E)--(D);


  \draw[-{Latex[length=4.5mm, width=2mm]}, >=stealth, veca,thick] (A)--(C) node[midway,below]{$\ba$};

  \draw[ veca,dashed] (C)--(E) node[midway,below]{$\ba$};


  \draw[veca,dashed] (B)--(D);


  \draw[-{Latex[length=4.5mm, width=2.5mm]}, >=stealth, vecc,thick] (A)--(D) node[midway,above]{$\bc$};


  \node [above right] at (2, 5) {$\bc = 2 \cdot \ba + 1 \cdot \bb $}; 


  \end{tikzpicture}
    
\end{center}


\end{frame}



\begin{frame}{Любой вектор — линейная комбинация}

Любой вектор $\bv \in \R^2$ — линейная комбинация векторов $\begin{pmatrix}
    1 \\
    0 \\
  \end{pmatrix}$ и $\begin{pmatrix}
    0 \\
    1 \\
  \end{pmatrix}$:

\[
\begin{pmatrix}
  v_1 \\
  v_2 \\
\end{pmatrix} = 
v_1 \begin{pmatrix}
    1 \\
    0 \\
  \end{pmatrix} + 
  v_2 \begin{pmatrix}
    0 \\
    1 \\
  \end{pmatrix}
\]

\pause
Аналогично, любой вектор  $\bv \in \R^3$ представим в виде:

\[
\begin{pmatrix}
v_1 \\
v_2 \\
v_3 \\
\end{pmatrix} = 
v_1 \begin{pmatrix}
  1 \\
  0 \\
  0 \\
\end{pmatrix} + 
v_2 \begin{pmatrix}
  0 \\
  1 \\
  0 \\
\end{pmatrix} +
v_3 \begin{pmatrix}
  0 \\
  0 \\
  1 \\
\end{pmatrix} 
\]


  

\end{frame}



\begin{frame}
\frametitle{Линейная зависимость}


\begin{block}{Определение}
Набор $A$ из двух и более векторов называется 
\alert{линейно зависимым}, если хотя бы один вектор является линейной комбинацией остальных.


Набор $A = \{\bzero\}$ из одного нулевого вектора также называется \alert{линейно зависимым}.
\end{block}


\end{frame}




\begin{frame}{Линейная зависимость: геометрия}

\begin{center}
\begin{tikzpicture}[
scale=1.5,
MyPoints/.style={draw=black,fill=black,thick},
Segments/.style={draw=blue!50!red!70,thick},
MyCircles/.style={green!50!blue!50,thin}, 
every node/.style={scale=1}
]

%\grid;

\clip (-1.5,-1.5) rectangle (5.5,5.5);

\begin{scope}[cm={1,1,1.5,0,(0,0)}]
\draw[draw=blue!30, dashed] (-1.2,-4.2) grid[step=1] (3.5,7);
\end{scope}

%{\verb!->!new, arrowhead = 2mm, line width=4pt}
%, arrowhead = 3mm
%, arrowhead = 0.2

% Feel free to change here coordinates of points A and B
\pgfmathparse{0}		\let\Xa\pgfmathresult
\pgfmathparse{0}		\let\Ya\pgfmathresult
\coordinate (A) at (\Xa,\Ya);

\pgfmathparse{2}		\let\Xb\pgfmathresult
\pgfmathparse{0.5}		\let\Yb\pgfmathresult
\coordinate (B) at (\Xb,\Yb);

\pgfmathparse{2}		\let\Xd\pgfmathresult
\pgfmathparse{4}		\let\Yd\pgfmathresult
\coordinate (D) at (\Xd,\Yd);

\pgfmathparse{4}		\let\Xc\pgfmathresult
\pgfmathparse{0}		\let\Yc\pgfmathresult
\coordinate (C) at (\Xc,\Yc);


\pgfmathparse{1}		\let\Xe\pgfmathresult
\pgfmathparse{1}		\let\Ye\pgfmathresult
\coordinate (E) at (\Xe,\Ye);

\pgfmathparse{2.5}		\let\Xf\pgfmathresult
\pgfmathparse{0}		\let\Yf\pgfmathresult
\coordinate (F) at (\Xf,\Yf);

\pgfmathparse{4}		\let\Xg\pgfmathresult
\pgfmathparse{1}		\let\Yg\pgfmathresult
\coordinate (G) at (\Xg,\Yg);




\draw[-{Latex[length=4.5mm, width=2.5mm]}, >=stealth, thick] (A)--(D) node[above left]{$\bd$};

\draw[-{Latex[length=4.5mm, width=2.5mm]}, >=stealth, vecb, thick] (A)--(E) node[right]{$\ba$};

\draw[-{Latex[length=4.5mm, width=2.5mm]}, >=stealth, vecb, thick] (A)--(F) node[below]{$\bb$};

\draw[-{Latex[length=4.5mm, width=2.5mm]}, >=stealth, veca, thick] (A)--(G) node[right]{$\bc$};


\draw[black, dashed] (B)--(D);

\fill[MyPoints]  (0,0) circle (0.8mm);

%\node [right,darkgray] at (-0.5,-2) {$\{\ba, \bb, \bc\}$ — линейно зависимы}; 

%\node [right,darkgray] at (-0.5,-3) {$\{\ba, \bb, \bd\}$ — независимы}; 



\end{tikzpicture}

\end{center}

Набор $\{\ba, \bb, \bc \}$ — линейно зависим.

Набор $\{\ba, \bb, \bd \}$ — линейно независим.

\end{frame}  









\begin{frame}
\frametitle{Линейная зависимость: примеры}



Набор $A = \left\{ \begin{pmatrix}
      0 \\
      2 \\
    \end{pmatrix}, \begin{pmatrix}
      3 \\
      4 \\
    \end{pmatrix} \right\}$ — линейно независимый.

\pause

Набор $B = \left\{ \begin{pmatrix}
      0 \\
      2 \\
      0 \\
    \end{pmatrix}, \begin{pmatrix}
      3 \\
      4 \\
      0 \\
    \end{pmatrix},
    \begin{pmatrix}
      1 \\
      0 \\
      0 \\
    \end{pmatrix} \right\}$ — линейно зависимый:

    \[
      \begin{pmatrix}
        3 \\
        4 \\
        0 \\
      \end{pmatrix} = 2
    \begin{pmatrix}
      0 \\
      2 \\
      0 \\
    \end{pmatrix} + 3
    \begin{pmatrix}
      1 \\
      0 \\
      0 \\
    \end{pmatrix}  
    \]
  

\end{frame}


\begin{frame}
  \frametitle{Линейная зависимость: дубль два}

\begin{block}{Эквивалентное определение} Набор векторов $A = \{ \bv_1, \bv_2, \ldots, \bv_k\}$ называется \alert{линейно зависимым},
  если можно найти такие веса $\alpha_1$, $\alpha_2$, \ldots, $\alpha_k$, что
  \[
  \alpha_1 \bv_1 + \alpha_2 \bv_2 + \ldots + \alpha_k \bv_k = \bzero,  
  \]
  и при этом хотя бы одно из чисел $\alpha_i$ отлично от $0$. 
\end{block}

\pause

\begin{block}{Доказательство эквивалентности}
Вектор с ненулевым коэффициентом $\alpha_i$ перед ним можно выразить через остальные. 
\pause

Если вектор $\bv_2$ выражен через $\bv_1$ и $\bv_3$, $\bv_2 = \alpha_1 \bv_1 + \alpha_3 \bv_3$, 
то искомая нулевая линейная комбинация имеет вид: $\alpha_1 \bv_1 +(-1)\bv_2 + \alpha_3 \bv_3=\bzero$.
\end{block}

\end{frame}

% !TEX root = ../linal_lecture_02.tex

\begin{frame} % название фрагмента

\videotitle{Линейная оболочка}

\end{frame}



\begin{frame}{Краткий план:}
  \begin{itemize}[<+->]
    \item Линейная оболочка векторов;
    \item Базис линейной оболочки векторов;
    \item Размерность линейной оболочки векторов.
  \end{itemize}

\end{frame}


\begin{frame}{Линейная оболочка}

\begin{block}{Определение} 
Множество векторов $V$, содержащее все возможные линейные комбинации векторов $\bv_1$, 
$\bv_2$, \ldots, $\bv_k$, называется их \alert{линейной оболочкой},
\[
  V = \Span\{ \bv_1, \bv_2, \ldots, \bv_k \}
\]
\end{block}

\end{frame}



\begin{frame}{Линейная оболочка векторов: картинка}




\begin{center}

  \begin{tikzpicture}[
    scale=1.5,
    MyPoints/.style={draw=black,fill=black,thick},
    Segments/.style={draw=blue!50!red!70,thick},
    MyCircles/.style={green!50!blue!50,thin}, 
    every node/.style={scale=1}
    ]

    %\grid;

    \clip (-1.5,-1.5) rectangle (5.5,5.5);

    \begin{scope}[cm={1,1,1.5,0,(0,0)}]
    \draw[draw=blue!30, dashed] (-1.2,-4.2) grid[step=1] (3.5,7);
    \end{scope}

    %{\verb!->!new, arrowhead = 2mm, line width=4pt}
    %, arrowhead = 3mm
    %, arrowhead = 0.2

    % Feel free to change here coordinates of points A and B
    \pgfmathparse{0}		\let\Xa\pgfmathresult
    \pgfmathparse{0}		\let\Ya\pgfmathresult
    \coordinate (A) at (\Xa,\Ya);

    \pgfmathparse{2}		\let\Xb\pgfmathresult
    \pgfmathparse{0.5}		\let\Yb\pgfmathresult
    \coordinate (B) at (\Xb,\Yb);

    \pgfmathparse{2}		\let\Xd\pgfmathresult
    \pgfmathparse{4}		\let\Yd\pgfmathresult
    \coordinate (D) at (\Xd,\Yd);

    \pgfmathparse{4}		\let\Xc\pgfmathresult
    \pgfmathparse{0}		\let\Yc\pgfmathresult
    \coordinate (C) at (\Xc,\Yc);


    \pgfmathparse{1}		\let\Xe\pgfmathresult
    \pgfmathparse{1}		\let\Ye\pgfmathresult
    \coordinate (E) at (\Xe,\Ye);

    \pgfmathparse{2.5}		\let\Xf\pgfmathresult
    \pgfmathparse{0}		\let\Yf\pgfmathresult
    \coordinate (F) at (\Xf,\Yf);

    \pgfmathparse{4}		\let\Xg\pgfmathresult
    \pgfmathparse{1}		\let\Yg\pgfmathresult
    \coordinate (G) at (\Xg,\Yg);




    \draw[-{Latex[length=4.5mm, width=2.5mm]}, >=stealth, thick] (A)--(D) node[above left]{$\bd$};

    \draw[-{Latex[length=4.5mm, width=2.5mm]}, >=stealth, vecb, thick] (A)--(E) node[right]{$\ba$};

    \draw[-{Latex[length=4.5mm, width=2.5mm]}, >=stealth, vecb, thick] (A)--(F) node[below]{$\bb$};

    \draw[-{Latex[length=4.5mm, width=2.5mm]}, >=stealth, veca, thick] (A)--(G) node[right]{$\bc$};


    \draw[black, dashed] (B)--(D);

    \fill[MyPoints]  (0,0) circle (0.8mm);

    %\node [right,darkgray] at (0.5,-2) {$\bc \in \operatorname{Lin} (\ba, \bb) $ }; 

    %\node [right,darkgray] at (0.5,-3) {$\bd \notin \operatorname{Lin} (\ba, \bb) $ }; 



    \end{tikzpicture}
  \end{center}
  
Вектор $\bc$ лежит в плоскости $\Span\{ \ba, \bb \}$.

Вектор $\bd$ не лежит в плоскости $\Span\{ \ba, \bb \}$.


\end{frame}



\begin{frame}
\frametitle{Линейная зависимость}


\begin{block}{Определение}
Набор $A$ из двух и более векторов называется 
\alert{линейно зависимым}, если хотя бы один вектор является линейной комбинацией остальных.


Набор $A = \{\bzero\}$ из одного нулевого вектора также называется \alert{линейно зависимым}.
\end{block}


\end{frame}


\begin{frame}
\frametitle{Линейная зависимость: примеры}



Набор $A = \left\{ \begin{pmatrix}
      0 \\
      2 \\
    \end{pmatrix}, \begin{pmatrix}
      3 \\
      4 \\
    \end{pmatrix} \right\}$ — линейно независимый.

\pause

Набор $A = \left\{ \begin{pmatrix}
      0 \\
      2 \\
      0 \\
    \end{pmatrix}, \begin{pmatrix}
      3 \\
      4 \\
      0 \\
    \end{pmatrix},
    \begin{pmatrix}
      1 \\
      0 \\
      0 \\
    \end{pmatrix} \right\}$ — линейно зависимый:

    \[
      \begin{pmatrix}
        3 \\
        4 \\
        0 \\
      \end{pmatrix} = 2
    \begin{pmatrix}
      0 \\
      2 \\
      0 \\
    \end{pmatrix} + 3
    \begin{pmatrix}
      1 \\
      0 \\
      0 \\
    \end{pmatrix}  
    \]
  

\end{frame}


\begin{frame}
  \frametitle{Линейная зависимость: дубль два}

\begin{block}{Эквивалентное пределение} Набор векторов $A = \{ \bv_1, \bv_2, \ldots, \bv_k\}$ называется \alert{линейно зависимым},
  если можно найти такие веса $\alpha_1$, $\alpha_2$, \ldots, $\alpha_k$, что
  \[
  \alpha_1 \bv_1 + \alpha_2 \bv_2 + \ldots + \alpha_k \bv_k = \bzero,  
  \]
  и при этом хотя бы одно из чисел $\alpha_i$ отлично от $0$. 
\end{block}

\pause

\begin{block}{Доказательство эквивалентности}
Вектор с ненулевым коэффициентом $\alpha_i$ перед ним можно выразить через остальные. 
\pause

Если вектор $\bv_2$ выражен через $\bv_1$ и $\bv_3$, $\bv_2 = \alpha_1 \bv_1 + \alpha_3 \bv_3$, 
то искомая нулевая линейная комбинация имеет вид: $\alpha_1 \bv_1 +(-1)\bv_2 + \alpha_3 \bv_3=\bzero$.
\end{block}

\end{frame}

% !TEX root = ../linal_lecture_02.tex

\begin{frame} % название фрагмента

\videotitle{Векторное пространство}

\end{frame}



\begin{frame}{Краткий план:}
  \begin{itemize}[<+->]
    \item Векторное пространство;
    \item Базис векторного пространства;
    \item Размерность векторного пространства.
  \end{itemize}

\end{frame}


\begin{frame}{Векторное пространство}

\begin{block}{Определение} 
Множество $V$ произвольных объектов называется \alert{конечномерным векторным пространством}, если:

\begin{itemize}[<+->]
\item множество $V$ можно взаимно однозначно сопоставить пространству $\R^n$;
\item определено сложение двух объектов $\ba$ и $\bb$ из $V$, 
и оно соответствует сложению столбцов из $\R^n$;
\item определено умножение объекта $\ba$ из $V$ на число $\lambda\in \R$, 
и оно соответствует умножению столбца $\R^n$ на $\lambda$.
\end{itemize}
\end{block}

Элементы векторного пространства называют \alert{векторами}. 
\pause
Векторное пространство также называют \alert{линейным}.

\end{frame}



\begin{frame}{Многочлены}
Множество $V$ всех многочленов от $t$ степени не выше трёх:
\[
V  = \{ at^3 + bt^2 + ct + d \mid a, b, c, d \in \R^n \}
\]

\pause
Взаимно однозначное сопоставление: 
\[5t^3 + 6t^2 - 3t + 2 \leftrightarrow \begin{pmatrix} 
    5 \\
    6 \\
    -3 \\
    2 \\
\end{pmatrix}.
\]
\pause


Сложение двух многочленов и умножение многочлена на число соответствуют операциям над столбцами чисел.
\end{frame}




\begin{frame}{Пример векторного пространства}

Множество $V$ всех функций $f(t)$ равных нулю вне двух данных точек:
\[
V  = \{ f \mid f(t)=0 \text{ для всех } t\neq \pm 1 \}
\]

\pause
Взаимно однозначное сопоставление: 
\[f \leftrightarrow \begin{pmatrix}
    f(-1) \\
    f(1) \\
\end{pmatrix}.
\]
\pause


Сложение двух таких функций и умножение на число соответствуют операциям над столбцами чисел.
\end{frame}
    

\begin{frame}{Типичный элемент $V$}
	
	\begin{center}


\begin{tikzpicture}[
scale=2,
MyPoints/.style={draw=blue,fill=blue,thick},
Segments/.style={draw=blue!50!red!70,thick},
MyCircles/.style={green!50!blue!50,thin}, 
every node/.style={scale=1}
]
%\grid;


\clip (-5.5,-2.5) rectangle (6.5,5);




%\draw[->, >=stealth] (-1,0)--(6.5,0) node[right]{$x_1$};
\draw[-{Latex[length=4.5mm, width=2.5mm]}, >=stealth] (0,-1)--(0,4) node[above right]{$f(t)$};

\draw[-{Latex[length=4.5mm, width=2.5mm]}, >=stealth] (-5,0)--(6,0) 
node[below ]{$t$};


%{\verb!->!new, arrowhead = 2mm, line width=4pt}
%, arrowhead = 3mm
%, arrowhead = 0.2

% Feel free to change here coordinates of points A and B
\pgfmathparse{-5}		\let\Xa\pgfmathresult
\pgfmathparse{0}		\let\Ya\pgfmathresult
\coordinate (A) at (\Xa,\Ya);

\pgfmathparse{5.55}		\let\Xb\pgfmathresult
\pgfmathparse{0}		\let\Yb\pgfmathresult
\coordinate (B) at (\Xb,\Yb);

\draw[vecb, line width=0.5mm] (A)--(B);


\fill[MyPoints]  (-2,2) circle (0.8mm);
\fill[MyPoints]  (2,3) circle (0.8mm);


%\path (0,.5) pic[red] {cross=1pt};
\draw (-2,0) node[cross=3mm,blue,line width=1mm] {};
\draw (2,0) node[cross=3mm,blue,line width=1mm] {};


\end{tikzpicture}

\end{center}

\end{frame}



\begin{frame}{Аналогия с $\R^n$}

% Сопоставление абстрактного множества $V$ и множества столбцов $\R^n$ дарит нам:
% \pause
\begin{block}{Определение} 
Вектор $\bc$ называется \alert{линейной комбинацией} векторов $\bv_1$, $\bv_2$, \ldots, $\bv_k$, 
если его можно представить в виде их суммы с некоторыми действительными весами $\alpha_i$:
\[
  \bc = \alpha_1 \bv_1 + \alpha_2 \bv_2 + \ldots + \alpha_k \bv_k
\]
\end{block}

\pause
\begin{block}{Определение} 
Множество векторов $M$, содержащее все возможные линейные комбинации векторов $\bv_1$, 
$\bv_2$, \ldots, $\bv_k$, называется их \alert{линейной оболочкой},
\[
  M = \Span\{ \bv_1, \bv_2, \ldots, \bv_k \}
\]
\end{block}
    
\pause 
Полностью аналогично определяются линейно зависимые и независимые наборы векторов. 

\end{frame}


\begin{frame}{Базис и размерность пространства}

\begin{block}{Определение}
\alert{Базисом векторного пространства} $V$ называется любой набор $\{\be_1, \be_2, \ldots, \be_n\}$,
такой что 
\begin{itemize}
    \item $V = \Span \{\be_1, \be_2, \ldots, \be_n\}$;
    \item векторы $\{\be_1, \be_2, \ldots, \be_n\}$ линейно независимы. 
\end{itemize}
\end{block}

\pause
\begin{block}{Определение}
Число  векторов в базисе, $n$, называют \alert{размерностью пространства} $V$, $\dim V = n$.
\end{block}

\end{frame}


\begin{frame}{Продолжаем аналогию}



Пространство $V$ взаимнооднозначно сопоставлено с $\R^n$ и при этом сложение в $V$ 
соответствует сложению в $\R^n$, 
а умножение на число в $V$ соответствует умножению на число в $\R^n$.
 


\begin{block}{Утверждение}
Линейная независимость в $V$ соответствует линейной независимости в $\R^n$.

Базис в $V$ соответствует базису в $\R^n$.

Размерность $V$ равна размерности $\R^n$, $\dim V = \dim \R^n = n$. 
\end{block}


\end{frame}



\begin{frame}{Формальности}

Мы слишком привыкли к свойствам чисел!

\pause

\vspace{40pt}

\begin{block}{Эквивалентное определение}
Множество $V$ называется \alert{векторным пространством}, если выполнено восемь свойств\ldots
\end{block}

\end{frame}

\begin{frame}{Восемь аксиом: сложение}
\begin{enumerate}
    \item При сложении можно расставлять скобки как хочешь (\alert{ассоциативность}):
    \[
    \ba + (\bb + \bc) = (\ba + \bb) + \bc    
    \]
    \item При сложении можно путать лево и право (\alert{коммутативность}):
    \[
    \ba + \bb = \bb + \ba    
    \]
    \item  Существует \alert{нулевой} вектор $\bzero$:
    \[
    \ba + \bzero = \ba    
    \]
    \item Для любого вектора $\ba$ найдется \alert{противоположный} вектор $-\ba$:
    \[
    \ba + (-\ba) = \bzero    
    \]
\end{enumerate}

\end{frame}


\begin{frame}{Восемь аксиом: умножение}
\begin{enumerate}[resume]
    \item[5.] Умножение вектора на число \alert{совместимо} с умножением чисел:
    \[
    \lambda_1 (\lambda_2\ba) = (\lambda_1 \lambda_2 ) \ba
    \]
    \item[6.]  Умножение на \alert{единицу} не меняет вектор:
    \[
    1\cdot \ba= \ba
    \]
    \item[7.] Раскрывать скобки вокруг векторов можно (\alert{дистрибутивность умножения}):
    \[
    \lambda (\ba + \bb) = \lambda \ba + \lambda \bb
    \]
    \item[8.] Раскрывать скобки вокруг чисел можно (\alert{дистрибутивность умножения}):
    \[
    (\lambda_1 + \lambda_2) \ba = \lambda_1 \ba + \lambda_2 \ba
    \]
\end{enumerate}

\end{frame}

%% !TEX root = ../linal_lecture_02.tex

\begin{frame} % название фрагмента

\videotitle{Матрица линейного оператора}

\end{frame}



\begin{frame}{Краткий план:}
  \begin{itemize}[<+->]
    \item Матрица линейного оператора;
    \item Примеры;
    \item Обобщение на векторное пространство.
  \end{itemize}

\end{frame}


\begin{frame}{Как записать линейный оператор?}

Любой вектор $\bv$ представим в виде:
\[
\bv =  \begin{pmatrix}
    v_1 \\
    v_2 \\
    \vdots \\
    v_n 
\end{pmatrix} =  
v_1  \begin{pmatrix}
    1 \\
    0 \\
    \vdots \\
    0 
\end{pmatrix} + v_2  \begin{pmatrix}
    0 \\
    1 \\
    \vdots \\
    0 
\end{pmatrix} + \ldots +
v_n  \begin{pmatrix}
    0 \\
    0 \\
    \vdots \\
    1 
\end{pmatrix}
\]

\pause
По свойству линейности
\[
\LL \bv =  \LL \begin{pmatrix}
    v_1 \\
    v_2 \\
    \vdots \\
    v_n 
\end{pmatrix} =  
v_1  \LL \begin{pmatrix}
    1 \\
    0 \\
    \vdots \\
    0 
\end{pmatrix} + v_2  \LL \begin{pmatrix}
    0 \\
    1 \\
    \vdots \\
    0 
\end{pmatrix} + \ldots +
v_n  \LL \begin{pmatrix}
    0 \\
    0 \\
    \vdots \\
    1 
\end{pmatrix}
\]

\pause
Достаточно понять, что оператор $\LL$ делает с векторами, содержащими 
одну единичку и нули на остальных местах.

\end{frame}


\begin{frame}{Запишем оператор по столбцам!}


Обозначим $\be_i$ — вектор, у которого на $i$-м месте стоит $1$, а на остальных местах — $0$.
\[
\be_i = \begin{pmatrix}
    0 \\
    \vdots \\
    0 \\
    1 \\
    0  \\
    \vdots \\
    0 \\
\end{pmatrix}    
\]

\pause 
\begin{block}{Определение}
\alert{Матрицей линейного оператора} $\LL: \R^n \to \R^k$ назовём прямоугольную табличку чисел, в которой $i$-ый столбец равен $\LL \be_i$. 
\end{block}
\end{frame}


\begin{frame}{Растягивание компонент}

$\LL : \begin{pmatrix}
  a_1 \\
  a_2 \\
\end{pmatrix} \to 
\begin{pmatrix}
  2a_1 \\
  -3a_2 \\
\end{pmatrix}$

\pause

Действие оператора $\LL$ на базисных векторах $\be_1$ и $\be_2$:

$\LL : \begin{pmatrix}
  1 \\
  0 \\
\end{pmatrix} \to 
\begin{pmatrix}
  2 \\
  0 \\
\end{pmatrix}, \quad
\LL : \begin{pmatrix}
  0 \\
  1 \\
\end{pmatrix} \to 
\begin{pmatrix}
  0 \\
  -3 \\
\end{pmatrix}$

\pause

Матрица оператора, $\LL = 
\begin{pmatrix}
  2 & 0  \\
  0 & -3 \\
\end{pmatrix}.$



\end{frame}



\begin{frame}{Перестановка компонент вектора}

$\LL : \begin{pmatrix}
  a_1 \\
  a_2 \\
  a_3 \\
\end{pmatrix} \to 
\begin{pmatrix}
  a_3 \\
  a_1 \\
  a_2 \\
\end{pmatrix}$

\pause

Действие оператора $\LL$ на базисных векторах $\be_1$, $\be_2$ и $\be_3$:

$\LL : \begin{pmatrix}
  1 \\
  0 \\
  0 \\
\end{pmatrix} \to 
\begin{pmatrix}
  0 \\
  1 \\
  0 \\
\end{pmatrix}, \;
\LL : \begin{pmatrix}
  0 \\
  1 \\
  0 \\
\end{pmatrix} \to 
\begin{pmatrix}
  0 \\
  0 \\
  1 \\
\end{pmatrix} \;
\LL : \begin{pmatrix}
  0 \\
  0 \\
  1 \\
\end{pmatrix} \to 
\begin{pmatrix}
  1 \\
  0 \\
  0 \\
\end{pmatrix}$

\pause

Матрица оператора, $\LL = 
\begin{pmatrix}
  0 & 0 & 1  \\
  1 & 0 & 0 \\
  0 & 1 & 0 \\
\end{pmatrix}.$



\end{frame}


\begin{frame}{Поворот плоскости}


Оператор $\Rot :\R^2 \to \R^2$ поворачивает плоскость на $30^{\circ}$ против часовой стрелки.

$\Rot : \begin{pmatrix}
  a_1 \\
  a_2 \\
\end{pmatrix} \to 
\begin{pmatrix}
  a_1\cos 30^{\circ} - a_2 \sin 30^{\circ}  \\
  a_1\sin 30^{\circ} + a_2 \cos 30^{\circ}  \\
\end{pmatrix}$

\pause

Действие оператора $\Rot$ на базисных векторах $\be_1$ и $\be_2$:

$\Rot : \begin{pmatrix}
  1 \\
  0 \\
\end{pmatrix} \to 
\begin{pmatrix}
\cos 30^{\circ}  \\
\sin 30^{\circ}  \\
\end{pmatrix}, \quad
\Rot : \begin{pmatrix}
  0 \\
  1 \\
\end{pmatrix} \to 
\begin{pmatrix}
 -\sin 30^{\circ}  \\
 \cos 30^{\circ} \\
\end{pmatrix}$

\pause

Матрица оператора, $\Rot = 
\begin{pmatrix}
  \cos 30^{\circ} & -\sin 30^{\circ}  \\
  \sin 30^{\circ} & \cos 30^{\circ} \\
\end{pmatrix}.$



\end{frame}
    


\begin{frame}{Оператор бездельника!}


$\Id : \begin{pmatrix}
  a_1 \\
  a_2 \\
\end{pmatrix} \to 
\begin{pmatrix}
  a_1 \\
  a_2 \\
\end{pmatrix}$

\pause

Действие оператора $\Id$ на базисных векторах $\be_1$ и $\be_2$:

$\Id : \begin{pmatrix}
  1 \\
  0 \\
\end{pmatrix} \to 
\begin{pmatrix}
1  \\
0  \\
\end{pmatrix}, \quad
\Id : \begin{pmatrix}
  0 \\
  1 \\
\end{pmatrix} \to 
\begin{pmatrix}
0 \\
1 \\
\end{pmatrix}$

\pause

Матрица оператора, \alert{единичная матрица}, $\Id = 
\begin{pmatrix}
  1 & 0  \\
  0 & 1 \\
\end{pmatrix}.$



\end{frame}
    



\begin{frame}{Дописывание нуля}


$\LL : \begin{pmatrix}
  a_1 \\
  a_2 \\
\end{pmatrix} \to 
\begin{pmatrix}
  a_1 \\
  0 \\
  a_2 \\
\end{pmatrix}$

\pause

Действие оператора $\LL$ на базисных векторах $\be_1$ и $\be_2$:

$\LL : \begin{pmatrix}
  1 \\
  0 \\
\end{pmatrix} \to 
\begin{pmatrix}
1  \\
0 \\
0  \\
\end{pmatrix}, \quad
\LL : \begin{pmatrix}
  0 \\
  1 \\
\end{pmatrix} \to 
\begin{pmatrix}
0 \\
0 \\
1 \\
\end{pmatrix}$

\pause

Матрица оператора, $\LL = 
\begin{pmatrix}
  1 & 0  \\
  0 & 0 \\
  0 & 1 \\
\end{pmatrix}.$


Матрица размера $3\times 2$ соответствует оператору $\LL: \R^2 \to \R^3$.


\end{frame}
    



\begin{frame}{Удаление компоненты вектора}


$\LL : \begin{pmatrix}
  a_1 \\
  a_2 \\
  a_3 \\
\end{pmatrix} \to 
\begin{pmatrix}
  a_1 \\
  a_3 \\
\end{pmatrix}$

\pause

Действие оператора $\LL$ на базисных векторах $\be_1$, $\be_2$ и $\be_3$:

$\LL : \begin{pmatrix}
  1 \\
  0 \\
  0 \\
\end{pmatrix} \to 
\begin{pmatrix}
1  \\
0  \\
\end{pmatrix}, \;
\LL : \begin{pmatrix}
  0 \\
  1 \\
  0 \\
\end{pmatrix} \to 
\begin{pmatrix}
0 \\
0 \\
\end{pmatrix}, \;
\LL : \begin{pmatrix}
  0 \\
  0 \\
  1 \\
\end{pmatrix} \to 
\begin{pmatrix}
0 \\
1 \\
\end{pmatrix}$

\pause

Матрица оператора, $\LL = 
\begin{pmatrix}
  1 & 0 & 0 \\
  0 & 0 & 1 \\
\end{pmatrix}.$


Матрица размера $2\times 3$ соответствует оператору $\LL: \R^3 \to \R^2$.


\end{frame}
    







\begin{frame}{Нумерация элементов матрицы}

\alert{Сначала строки, потом столбцы!}
\pause

\[
A = \begin{pmatrix} 
    5 & \textcolor{red}{-2} & 8 \\
    7 & 1 & 9 \\
\end{pmatrix}    
\]

Матрица $A$ имеет размер $2 \times 3$ и $a_{12} = -2$.

\pause
Элемент матрицы $A$, лежащий в строке $i$ в столбце $j$, 
обозначают $a_{ij}$.

Матрица имеет размер $n\times k$, если в ней $n$ строк
и $k$ столбцов.

\end{frame}



\begin{frame}{Абстрактное определение}

% Мы не делали разницу между оператором $\LL$ и его матрицей.

%\pause 
% В тех редких ситуациях, когда эту разницу нужно подчеркнуть, 
% потребуются аккуратные обозначения. 

% \pause
Пусть оператор $\LL$ действует из пространства $V$ с базисом 
$\be = \{\be_1, \be_2, \ldots, \be_n\}$ в пространство $W$ с базисом
$\bff = \{\bff_1, \bff_2, \ldots, \bff_k\}$.

\pause
\begin{block}{Определение}
\alert{Матрицей} $\LL_{\be\bff}$ \alert{линейного оператора} $\LL$ называется табличка
чисел, определяемая по следующему алгоритму:

\begin{enumerate}
     \item Находим вектор $\LL \be_j \in W$.
     \pause
    \item Раскладываем этот вектор по базису $\bff$:
     \[
         \LL \be_j = a_{1j} \bff_1 + a_{2j} \bff_2 + \ldots + a_{kj} \bff_k
     \]
     \pause
     \item Помещаем числа $a_{1j}, a_{2j}, \ldots, a_{kj}$ в столбец $j$ таблички.
     \pause
     \item Повторяем шаги 1, 2 и 3 для всех столбцов.
\end{enumerate}

\end{block}

\end{frame}


%% !TEX root = ../linal_lecture_02.tex

\begin{frame} % название фрагмента

\videotitle{Ранг оператора}

\end{frame}



\begin{frame}{Краткий план:}
  \begin{itemize}[<+->]
    \item Множество значений оператора;
    \item Ранг оператора.
  \end{itemize}

\end{frame}


\begin{frame}{Множество значений оператора}

Любой вектор $\bv$ представим в виде:
\[
\bv = v_1 \be_1 + v_2 \be_2 + \ldots + v_n \be_n
\]

\pause
По свойству линейности
\[
\LL \bv = v_1 \LL\be_1 + v_2 \LL\be_2 + \ldots + v_n\LL\be_n
\]

\pause
\begin{block}{Утверждение}
Множество значений оператора $\LL$ можно записать в виде линейной оболочки:
\[
\Image \LL  = \Span \{ \LL \be_1, \LL \be_2, \ldots, \LL \be_n  \}  
\]
\end{block}

\end{frame}



\begin{frame}{Ранг оператора}
\begin{block}{Определение}
    \alert{Рангом} линейного оператора $\LL$ называют размерность его образа:
    \[
      \rank \LL = \dim \Image \LL = \dim \Span \{ \LL\be_1, \LL\be_2, \ldots, \LL\be_n\}  
    \]
\end{block}
\end{frame}

\begin{frame}{Удаление компоненты вектора}

$\LL : \begin{pmatrix}
  a_1 \\
  a_2 \\
  a_3 \\
\end{pmatrix} \to 
\begin{pmatrix}
  a_1 \\
  a_3 \\
\end{pmatrix}$

\pause

Действие оператора $\LL$ на базисных векторах $\be_1$, $\be_2$ и $\be_3$:

$\LL : \begin{pmatrix}
  1 \\
  0 \\
  0 \\
\end{pmatrix} \to 
\begin{pmatrix}
1  \\
0  \\
\end{pmatrix}, \;
\LL : \begin{pmatrix}
  0 \\
  1 \\
  0 \\
\end{pmatrix} \to 
\begin{pmatrix}
0 \\
0 \\
\end{pmatrix}, \;
\LL : \begin{pmatrix}
  0 \\
  0 \\
  1 \\
\end{pmatrix} \to 
\begin{pmatrix}
0 \\
1 \\
\end{pmatrix}$

\pause

\[
\Image \LL = \Span \left\{\begin{pmatrix}
1 \\
0 \\
\end{pmatrix}, \;
\begin{pmatrix}
0 \\
0 \\
\end{pmatrix}, \;
\begin{pmatrix}
0 \\
1 \\
\end{pmatrix}
\right\}    
\]

\pause 
Базис для $\Image \LL$: $\left\{\begin{pmatrix}
1 \\
0 \\
\end{pmatrix}, \;
\begin{pmatrix}
0 \\
1 \\
\end{pmatrix} \right\}$

\[
\rank \LL = \dim \Image \LL = 2    
\]

\end{frame}

\begin{frame}{Ранг проекции}


Если оператор $\HH$ проецирует векторы на прямую $\ell$, то $\Image \HH = \Span \ba$, где $\ba$ — любой ненулевой вектор, лежащий на прямой $\ell$.

\pause

Ранг оператора проецирования на прямую равен $\rank \HH = 1$.

\pause

Ранг оператора проецирования $\HH$ равен размерности того множества, на которое проецируют.

\pause

\begin{block}{Определение}
Ранг оператора проецирования $\HH$ также называют \alert{следом оператора проецирования},
$\trace \HH = \rank \HH$.
\end{block}


\end{frame}


\begin{frame}{Ранг поворота}

Оператор $\Rot$ поворачивает плоскость на $30^{\circ}$ градусов против часовой стрелки.

\pause

Поворачивая различные векторы, можно получить любой вектор на плоскости, $\Image \Rot = \R^2$.

\pause

Базис образа: $\{ \be_1, \be_2 \}$, значит $\rank \Rot  = 2$.



\end{frame}
    


\begin{frame}{Ограничения на ранг}

\begin{block}{Утверждение}
Ранг оператора $\LL: \R^n \to \R^k$ не превосходит ни $n$, ни $k$. 
\end{block}

\pause

\begin{block}{Доказательство}
Базис во всём $\R^k$ содержит $k$ элементов, значит базис образа не больше.
\pause

Образ получается как $\Span\{\LL \be_1, \LL\be_2, \ldots, \LL \be_n \}$.
\end{block}
\end{frame}

\begin{frame}{Ранг произведения операторов}

\begin{block}{Утверждение}
Ранг произведения $\LL_1: \R^n \to \R^k$ и $\LL_2: \R^k \to \R^d$ не превосходит ранга сомножителей, $\rank (\LL_2 \LL_1) \leq \min \{\rank \LL_1, \rank \LL_2 \}$.
\end{block}

\pause
\begin{block}{Доказательство}
\[
\Image(\LL_2\LL_1) \subset \Image(\LL_2)  
\]
\pause
Если $\Image \LL_1 = \Span\{ \bv_1, \bv_2, \ldots, \bv_p \}$, то $\Image (\LL_2 \LL_1 ) =\Span\{ \LL_2 \bv_1, \LL_2 \bv_2, \ldots, \LL_2 \bv_p \}$.

\end{block}
\end{frame}
  






\begin{frame}{Ранг матрицы}

\begin{block}{Определение}
\alert{Рангом матрицы} называют ранг соответствующего оператора.
\end{block}

\pause

\begin{block}{Утверждение}
Ранг матрицы равен максимальному количеству линейно независимых столбцов матрицы. 
\end{block}

\pause
\begin{block}{Доказательство}
Именно эти линейно-независимые столбцы и будут базисом в линейной оболочке $\Image \LL$. 
\end{block}


\end{frame}







\begin{frame}

\lecturetitle{Умножение матрицы на вектор}

\todo{Это видеофрагмент с доской, слайдов здесь нет :)}

\end{frame}



\begin{frame}
\lecturetitle{Умножение матрицы на матрицу}
\todo{Это видеофрагмент с доской, слайдов здесь нет :)}
\end{frame}
    


\begin{frame}
\lecturetitle{Три взгляда на умножение матриц}
\todo{Это видеофрагмент с доской, слайдов здесь нет :)}
\end{frame}
    

\begin{frame}
\lecturetitle{Решение системы уравнений методом Гаусса}
\todo{Это видеофрагмент с доской, слайдов здесь нет :)}
\end{frame}
    

%% !TEX root = ../linal_lecture_02.tex

\begin{frame} % название фрагмента

\videotitle{Системы линейных уравнений}

\end{frame}



\begin{frame}{Краткий план:}
  \begin{itemize}[<+->]
    \item Однородная система и ядро оператора;
    \item Структура множества решений.
  \end{itemize}

\end{frame}


\begin{frame}{Варианты записи системы}

Скалярный: 
$\begin{cases}
5x_1 + 6x_2 = 8 \\
3x_1 + 7x_2 = 9 \\
\end{cases}$

\pause

Векторный:
$x_1 \begin{pmatrix}
    5 \\
    3 
\end{pmatrix} + 
x_2 \begin{pmatrix}
    6 \\
    7
\end{pmatrix} = 
\begin{pmatrix}
    8 \\
    9
\end{pmatrix}$.
\pause

Матричный:
$\begin{pmatrix}
    5 & 6 \\
    3 & 7 \\
\end{pmatrix} \cdot 
\begin{pmatrix}
    x_1 \\
    x_2 \\
\end{pmatrix} = \begin{pmatrix}
    8 \\
    9 \\
\end{pmatrix}$.


\end{frame}


\begin{frame}{Однородная и неоднородная системы}

\begin{block}{Определение}
    Система уравнений $A\bx = 0$ называется \alert{однородной}.
\end{block}
\pause

Однородная система: 
$\begin{pmatrix}
    5 & 6 \\
    3 & 7 \\
\end{pmatrix} \cdot 
\begin{pmatrix}
    x_1 \\
    x_2 \\
\end{pmatrix} = \begin{pmatrix}
    0 \\
    0 \\
\end{pmatrix}$.
\pause

Неоднородная система: 
$\begin{pmatrix}
    5 & 6 \\
    3 & 7 \\
\end{pmatrix} \cdot 
\begin{pmatrix}
    x_1 \\
    x_2 \\
\end{pmatrix} = \begin{pmatrix}
    8 \\
    9 \\
\end{pmatrix}$.


\end{frame}



\begin{frame}{Ядро оператора}

\begin{block}{Определение}
\alert{Ядром} линейного оператора $\LL: \R^n \to \R^k$ называется множество векторов,
которые под действием $\LL$ превращаются в $\bzero \in \R^k$: 
\[
\ker \LL = \{ \bv \in \R^n \mid \LL \bv = \bzero \}    
\]
\end{block}

\pause
Чтобы найти ядро $\LL$ нужно решить однородную систему $\LL \bv = \bzero$.

\end{frame}



\begin{frame}{Метод Гаусса}

Основная идея: по очереди избавиться от всех неизвестных.
\pause

\begin{block}{Алгоритм}
\begin{enumerate}
    \item Выберем первое уравнение так, чтобы в нём была переменная $x_1$.
    \pause
    \item Вычитаем первое уравнение из остальных так, чтобы в них пропала переменная $x_1$.
    \pause
    \item Зафиксируем первое уравнение и работаем с остальными. 
\end{enumerate}
\end{block}


\pause
В финальной системе в каждом следующем уравнении меньше неизвестных, чем в предыдущем.



\end{frame}



\begin{frame}{Ступенчатый вид}


После применения метода Гаусса система примет ступенчатый вид:
\[
\left[
\begin{array}{ccccc|c}
\textcolor{red}{2} & 0 & 3 & -1 & 5 & 2 \\
0 & \textcolor{red}{1} & 1 & -1 & -2 & 3 \\
0 & 0 & 0 & \textcolor{red}{3} & 0 & -1 \\
\end{array}
\right]
\]

\pause
Неизвестные, лежащие в начале «ступеньки», называются \alert{главными}, а остальные —
\alert{свободными}.

Главные переменные можно выразить через свободные.

\end{frame}





\begin{frame}{Количество решений}
\begin{block}{Утверждение}
Система уравнений $A \bx = \bb$ имеет ноль, одно или бесконечное количество решений. 
\end{block}
\pause

\begin{block}{Доказательство}
После применения метода Гаусса последнее уравнение, в котором хотя бы один коэффициент 
отличен от нуля, окажется одного из трёх видов:
\[
\begin{array}{l}
A: 0x_1 + 0x_2 + 0x_3 + 0x_4 = 7, \text{ нет решений.}\\    
B: 0x_1 + 0x_2 + 0x_3 + 5x_4 = 7, \text{ хотя бы одно решение.} \\
C: 0x_1 + 3x_2 + 2x_3 + 5x_4 = 7, \text{ бесконечное количество.} \\
\end{array}
\]
\end{block}
\pause

В случае C мы получаем в последнем уравнении свободу выбора $x_3$ и $x_4$.

\end{frame}



\begin{frame}{Структура множества решений}

\begin{block}{Утверждение}
Если решений бесконечное множество, то ответ можно записать в виде:
\[
\bx = \ba + \alpha_1 \bv_1 + \ldots + \alpha_k \bv_k,
\]
где $\ba$, $\bv_1$, \ldots, $\bv_k$ — конкретные векторы, а
$\alpha_1$, \ldots, $\alpha_k$ — произвольные числа. 
\end{block}

\pause
Для однородной системы $\ba=\bzero$, 
    а число $k$ является размерностью множества решений, $k=\dim\ker A$.

\end{frame}




\begin{frame}{Резюме}


\begin{itemize}[<+->]
\item Линейная комбинация и линейная оболочка.
\item Абстрактное векторное пространство.
\item Матрица линейного оператора.
\item Ранг оператора. 
\item Умножение матрицы на вектор и на матрицу.
\item Решение системы методом Гаусса.
\item Бонус: задача о шахматной доске.
\end{itemize}
\pause
\alert{Следующая лекция:} определитель и обратная матрица.



\end{frame}
    


\begin{frame}
\lecturetitle{Задача о шахматной доске}
\todo{Это видеофрагмент с доской, слайдов здесь нет :)}
\end{frame}


\end{document}
