\documentclass[14 pt,xcolor=dvipsnames]{beamer}

% !TEX root = linal_lecture_01.tex

\usepackage{epsdice}

\usepackage[absolute,overlay]{textpos}

\usepackage[orientation=portrait,size=custom,width=25.4,height=19.05]{beamerposter}

%25,4 см 19,05 см размеры слайда в powerpoint

\usetheme{metropolis}
\metroset{
  %progressbar=none,
  numbering=none,
  subsectionpage=progressbar,
  block=fill
}

%\usecolortheme{seahorse}

\usepackage{fontspec}
\usepackage{polyglossia}
\setmainlanguage{russian}


\usepackage{fontawesome5} % removed [fixed]
\setmainfont[Ligatures=TeX]{Myriad Pro}
\setsansfont{Myriad Pro}

% why do we need \newfontfamily:
% http://tex.stackexchange.com/questions/91507/
\newfontfamily{\cyrillicfonttt}{Myriad Pro}
\newfontfamily{\cyrillicfont}{Myriad Pro}
%\newfontfamily{\cyrillicfontbs}{Myriad Pro}
\newfontfamily{\cyrillicfontsf}{Myriad Pro}


% https://tex.stackexchange.com/questions/175860/why-does-unicode-math-break-the-kerning-of-accents-in-combination-with-amssymb
% "You shouldn't be using amssymb together with unicode-math"
\usepackage{amsmath,amsxtra,amsthm} % amssymb

%\usepackage{bm}

\usepackage{fdsymbol} % \nperp

\usepackage{unicode-math}


\usepackage{centernot}

\usepackage{graphicx}
\graphicspath{{img/}}

\usepackage{wrapfig}
\usepackage{animate}
\usepackage{tikz}
%\usetikzlibrary{shapes.geometric,patterns,positioning,matrix,calc,arrows,shapes,fit,decorations,decorations.pathmorphing}
\usepackage{pifont}
\usepackage{comment}
\usepackage[font=small,labelfont=bf]{caption}
\captionsetup[figure]{labelformat=empty}
\includecomment{techno}

\usefonttheme[onlymath]{serif}


%Расположение

\setbeamersize{text margin left=15 mm,text margin right=5mm} 
\setlength{\leftmargini}{38 pt}

%\usepackage{showframe}
%\usepackage{enumitem}
%\setlist{leftmargin=5.5mm}


%Цвета от дирекции

\definecolor{dirblack}{RGB}{58, 58, 58}
\definecolor{dirwhite}{RGB}{245, 245, 245}
\definecolor{dirred}{RGB}{149, 55, 53}
\definecolor{dirblue}{RGB}{0, 90, 171}
\definecolor{dirorange}{RGB}{235, 143, 76}
\definecolor{dirlightblue}{RGB}{75, 172, 198}
\definecolor{dirgreen}{RGB}{155, 187, 89}
\definecolor{dircomment}{RGB}{128, 100, 162}

\setbeamercolor{title separator}{bg=dirlightblue!50, fg=dirblue}

%Цвета блоков

% Голубой блок!
\setbeamercolor{block title}{bg=dirblue!30,fg=dirblack}
\setbeamercolor{block title example}{bg=dirlightblue!50,fg=dirblack}
\setbeamercolor{block body example}{bg=dirlightblue!20,fg=dirblack}

\AtBeginEnvironment{exampleblock}{\setbeamercolor{itemize item}{fg=dirblack}}
%\setbeamertemplate{blocks}[rounded][shadow]

% Набор команд для удобства верстки

\newcommand{\RR}{\mathbb{R}}
\newcommand{\ZZ}{\mathbb{Z}}
\newcommand{\la}{\lambda}

% Набор команд для структуризации

%\newcommand{\quest}{\faQuestionCircleO}
%\faPencilSquareO \faPuzzlePiece \faQuestionCircleO  \faIcon*[regular]{file} {\textcolor{dirblue}
%\newcommand{\quest}{\textcolor{dirblue}{\boxed{\textbf{?}}}
\newcommand{\task}{\faIcon{tasks}}
\newcommand{\exmpl}{\faPuzzlePiece}
\newcommand{\dfn}{\faIcon{pen-square}}
\newcommand{\quest}{\textcolor{dirblue}{\faQuestionCircle[regular]}}
\newcommand{\acc}[1]{\textcolor{dirred}{#1}}
\newcommand{\accm}[1]{\textcolor{dirred}{#1}}
\newcommand{\acct}[1]{\textcolor{dirblue}{#1}}
\newcommand{\acctm}[1]{\textcolor{dirblue}{#1}}
\newcommand{\accex}[1]{\textcolor{dirblack}{\bf #1}}
\newcommand{\accexm}[1]{\textcolor{dirblack}{ \mathbf{#1}}}
\newcommand{\acclp}[1]{\textcolor{dirorange}{\it #1}}
\newcommand{\todo}[1]{\textcolor{dircomment}{\bf #1}}
\newcommand{\graylink}[1]{{\fontsize{11}{12}\selectfont \textcolor{gray}{#1}}}
\newcommand{\figcaption}[1]{{\fontsize{18}{20}\selectfont #1}}


\newcommand{\videotitle}[1]{
    {\fontsize{33}{30}\selectfont \textcolor{dirblue}{\textbf{#1}} }

    %\todo{название видеофрагмента}
}

\newcommand{\lecturetitle}[1]{
  {\fontsize{33}{30}\selectfont \textcolor{dirblue}{\textbf{#1}} }

    %\todo{название лекции}
}





\newcommand{\spcbig}{\vspace{-10 pt}}
\newcommand{\spcsmall}{\vspace{-5 pt}}

%\usepackage{listings}
%\lstset{
%xleftmargin=0 pt,
%  basicstyle=\small, 
%  language=Python,
  %tabsize = 2,
%  backgroundcolor=\color{mc!20!white}
%}



%\newcommand{\mypart}[1]{\begin{frame}[standout]{\huge #1}\end{frame}}

\setbeamercolor{background canvas}{bg=}

% frame title setup
\setbeamercolor{frametitle}{bg=,fg=dirblue}
\setbeamertemplate{frametitle}[default][left]

\addtobeamertemplate{frametitle}{\hspace*{-0.5 cm}}{\vspace*{0.25cm}}


%Шрифты
\setbeamerfont{frametitle}{family=\rmfamily,series=\bfseries,size={\fontsize{33}{30}}}
\setbeamerfont{framesubtitle}{family=\rmfamily,series=\bfseries,size={\fontsize{26}{20}}}


% удобнее знать номер слайда, чтобы вносить правки!  

\setbeamercolor{footline}{fg=dircomment}
\setbeamerfont{footline}{series=\bfseries, size={\fontsize{12}{14}}}
%\setbeamertemplate{footline}[page number]

\defbeamertemplate{footline}{custom footline}
{%
  \hspace*{\fill}%
  \usebeamercolor[fg]{page number in head/foot}%
  \usebeamerfont{page number in head/foot}%
  page: \insertpagenumber\,/\,\insertpresentationendpage%
  \hspace{20pt}%
  slide: \insertframenumber\,/\,\inserttotalframenumber%
  %\hspace*{\fill}
  \vskip2pt%
}
\setbeamertemplate{footline}[custom footline]

\usepackage{physics}
\newcommand{\R}{\mathbb{R}}
\newcommand{\Rot}{\mathrm{R}}
\newcommand{\HH}{\mathrm{H}}
\newcommand{\Id}{\mathrm{I}}


\usepackage[outline]{contour}




\usepackage{pgfplots}
\pgfplotsset{compat=newest}

\usepackage{tikz}
\usetikzlibrary{calc}
\usetikzlibrary{quotes,angles}
\usetikzlibrary{arrows}
\usetikzlibrary{arrows.meta}
\usetikzlibrary{positioning,intersections,decorations.markings}
\usetikzlibrary{patterns}

\usepackage{tkz-euclide} 

\newcommand{\grid}{\draw[color=gray,step=1.0,dotted] (-2.1,-2.1) grid (9.6,6.1)}

\newcommand{\ba}{\symbf{a}}
\newcommand{\be}{\symbf{e}}
\newcommand{\bb}{\symbf{b}}
\newcommand{\bc}{\symbf{c}}
\newcommand{\bd}{\symbf{d}}
\newcommand{\bx}{\symbf{x}}
\newcommand{\bv}{\symbf{v}}
\newcommand{\bzero}{\symbf{0}}


\DeclareMathOperator{\Lin}{Span}

\DeclareMathOperator{\Span}{Span}
\DeclareMathOperator{\LL}{L}

%\tikzset{>=latex}

\colorlet{veca}{red}
\colorlet{vecb}{blue}
\colorlet{vecc}{olive}







\begin{document}


\begin{frame} % название лекции


\lecturetitle{Матричная запись}

\end{frame}


% !TEX root = ../linal_lecture_02.tex

\begin{frame} % название фрагмента

\videotitle{Линейная комбинация и независимость}

\end{frame}



\begin{frame}{Краткий план:}
  \begin{itemize}[<+->]
    \item Линейная комбинация векторов;
    \item Зависимые и независимые наборы векторов.
  \end{itemize}

\end{frame}


\begin{frame}{Линейная комбинация}

\begin{block}{Определение} 
Вектор $\bc$ называется \alert{линейной комбинацией} векторов $\bv_1$, $\bv_2$, \ldots, $\bv_k$, 
если его можно представить в виде их суммы с некоторыми действительными весами $\alpha_i$:
\[
  \bc = \alpha_1 \bv_1 + \alpha_2 \bv_2 + \ldots + \alpha_k \bv_k
\]
\end{block}

\pause
Пример. Вектор $\begin{pmatrix}
  4 \\
  5 \\
\end{pmatrix}$ — это линейная комбинация векторов $\begin{pmatrix}
  1 \\
  0 \\
\end{pmatrix}$ и $\begin{pmatrix}
  1 \\
  1 \\
\end{pmatrix}$:

\[
\begin{pmatrix}
  4 \\
  5 \\
\end{pmatrix} = -1 \begin{pmatrix}
  1 \\
  0 \\
\end{pmatrix} + 5 \begin{pmatrix}
  1 \\
  1 \\
\end{pmatrix}  
\]


\end{frame}



\begin{frame}{Линейная комбинация: геометрия}


\begin{center}

\begin{tikzpicture}[
  scale=1.6,
  MyPoints/.style={draw=blue,fill=white,thick},
  Segments/.style={draw=blue!50!red!70,thick},
  MyCircles/.style={green!50!blue!50,thin}, 
  every node/.style={scale=1.2}
  ]
  %\grid;
  \clip (-.5,-.5) rectangle (7.5,6.5);


  %%\draw[->, >=stealth] (-1,0)--(6.5,0) node[right]{$x_1$};
  %\draw[-{Latex[length=4.5mm, width=2.5mm]}, >=stealth] (0,-1)--(0,5) node[above left]{$x_2$};
  %
  %\draw[-{Latex[length=4.5mm, width=2.5mm]}, >=stealth] (-1,0)--(6.5,0) 
  %node[right]{$x_1$};

  % Feel free to change here coordinates of points A and B
  \pgfmathparse{0}		\let\Xa\pgfmathresult
  \pgfmathparse{0}		\let\Ya\pgfmathresult
  \coordinate (A) at (\Xa,\Ya);

  \pgfmathparse{1}		\let\Xb\pgfmathresult
  \pgfmathparse{3}		\let\Yb\pgfmathresult
  \coordinate (B) at (\Xb,\Yb);

  \pgfmathparse{3}		\let\Xc\pgfmathresult
  \pgfmathparse{1}		\let\Yc\pgfmathresult
  \coordinate (C) at (\Xc,\Yc);

  \pgfmathparse{7}		\let\Xd\pgfmathresult
  \pgfmathparse{5}		\let\Yd\pgfmathresult
  \coordinate (D) at (\Xd,\Yd);

  \pgfmathparse{6}		\let\Xe\pgfmathresult
  \pgfmathparse{2}		\let\Ye\pgfmathresult
  \coordinate (E) at (\Xe,\Ye);


  % Let I be the midpoint of [AB]
  \pgfmathparse{(\Xb+\Xa)/2} \let\XI\pgfmathresult
  \pgfmathparse{(\Yb+\Ya)/2} \let\YI\pgfmathresult
  \coordinate (I) at (\XI,\YI);	


  \draw[-{Latex[length=4.5mm, width=2mm]}, >=stealth, vecb,thick] (A)--(B) node[midway,left]{$\bb$};

  \draw[vecb,dashed] (E)--(D);


  \draw[-{Latex[length=4.5mm, width=2mm]}, >=stealth, veca,thick] (A)--(C) node[midway,below]{$\ba$};

  \draw[ veca,dashed] (C)--(E) node[midway,below]{$\ba$};


  \draw[veca,dashed] (B)--(D);


  \draw[-{Latex[length=4.5mm, width=2.5mm]}, >=stealth, vecc,thick] (A)--(D) node[midway,above]{$\bc$};


  \node [above right] at (2, 5) {$\bc = 2 \cdot \ba + 1 \cdot \bb $}; 


  \end{tikzpicture}
    
\end{center}


\end{frame}



\begin{frame}{Любой вектор — линейная комбинация}

Любой вектор $\bv \in \R^2$ — линейная комбинация векторов $\begin{pmatrix}
    1 \\
    0 \\
  \end{pmatrix}$ и $\begin{pmatrix}
    0 \\
    1 \\
  \end{pmatrix}$:

\[
\begin{pmatrix}
  v_1 \\
  v_2 \\
\end{pmatrix} = 
v_1 \begin{pmatrix}
    1 \\
    0 \\
  \end{pmatrix} + 
  v_2 \begin{pmatrix}
    0 \\
    1 \\
  \end{pmatrix}
\]

\pause
Аналогично, любой вектор  $\bv \in \R^3$ представим в виде:

\[
\begin{pmatrix}
v_1 \\
v_2 \\
v_3 \\
\end{pmatrix} = 
v_1 \begin{pmatrix}
  1 \\
  0 \\
  0 \\
\end{pmatrix} + 
v_2 \begin{pmatrix}
  0 \\
  1 \\
  0 \\
\end{pmatrix} +
v_3 \begin{pmatrix}
  0 \\
  0 \\
  1 \\
\end{pmatrix} 
\]


  

\end{frame}



\begin{frame}
\frametitle{Линейная зависимость}


\begin{block}{Определение}
Набор $A$ из двух и более векторов называется 
\alert{линейно зависимым}, если хотя бы один вектор является линейной комбинацией остальных.


Набор $A = \{\bzero\}$ из одного нулевого вектора также называется \alert{линейно зависимым}.
\end{block}


\end{frame}




\begin{frame}{Линейная зависимость: геометрия}

\begin{center}
\begin{tikzpicture}[
scale=1.5,
MyPoints/.style={draw=black,fill=black,thick},
Segments/.style={draw=blue!50!red!70,thick},
MyCircles/.style={green!50!blue!50,thin}, 
every node/.style={scale=1}
]

%\grid;

\clip (-1.5,-1.5) rectangle (5.5,5.5);

\begin{scope}[cm={1,1,1.5,0,(0,0)}]
\draw[draw=blue!30, dashed] (-1.2,-4.2) grid[step=1] (3.5,7);
\end{scope}

%{\verb!->!new, arrowhead = 2mm, line width=4pt}
%, arrowhead = 3mm
%, arrowhead = 0.2

% Feel free to change here coordinates of points A and B
\pgfmathparse{0}		\let\Xa\pgfmathresult
\pgfmathparse{0}		\let\Ya\pgfmathresult
\coordinate (A) at (\Xa,\Ya);

\pgfmathparse{2}		\let\Xb\pgfmathresult
\pgfmathparse{0.5}		\let\Yb\pgfmathresult
\coordinate (B) at (\Xb,\Yb);

\pgfmathparse{2}		\let\Xd\pgfmathresult
\pgfmathparse{4}		\let\Yd\pgfmathresult
\coordinate (D) at (\Xd,\Yd);

\pgfmathparse{4}		\let\Xc\pgfmathresult
\pgfmathparse{0}		\let\Yc\pgfmathresult
\coordinate (C) at (\Xc,\Yc);


\pgfmathparse{1}		\let\Xe\pgfmathresult
\pgfmathparse{1}		\let\Ye\pgfmathresult
\coordinate (E) at (\Xe,\Ye);

\pgfmathparse{2.5}		\let\Xf\pgfmathresult
\pgfmathparse{0}		\let\Yf\pgfmathresult
\coordinate (F) at (\Xf,\Yf);

\pgfmathparse{4}		\let\Xg\pgfmathresult
\pgfmathparse{1}		\let\Yg\pgfmathresult
\coordinate (G) at (\Xg,\Yg);




\draw[-{Latex[length=4.5mm, width=2.5mm]}, >=stealth, thick] (A)--(D) node[above left]{$\bd$};

\draw[-{Latex[length=4.5mm, width=2.5mm]}, >=stealth, vecb, thick] (A)--(E) node[right]{$\ba$};

\draw[-{Latex[length=4.5mm, width=2.5mm]}, >=stealth, vecb, thick] (A)--(F) node[below]{$\bb$};

\draw[-{Latex[length=4.5mm, width=2.5mm]}, >=stealth, veca, thick] (A)--(G) node[right]{$\bc$};


\draw[black, dashed] (B)--(D);

\fill[MyPoints]  (0,0) circle (0.8mm);

%\node [right,darkgray] at (-0.5,-2) {$\{\ba, \bb, \bc\}$ — линейно зависимы}; 

%\node [right,darkgray] at (-0.5,-3) {$\{\ba, \bb, \bd\}$ — независимы}; 



\end{tikzpicture}

\end{center}

Набор $\{\ba, \bb, \bc \}$ — линейно зависим.

Набор $\{\ba, \bb, \bd \}$ — линейно независим.

\end{frame}  









\begin{frame}
\frametitle{Линейная зависимость: примеры}



Набор $A = \left\{ \begin{pmatrix}
      0 \\
      2 \\
    \end{pmatrix}, \begin{pmatrix}
      3 \\
      4 \\
    \end{pmatrix} \right\}$ — линейно независимый.

\pause

Набор $B = \left\{ \begin{pmatrix}
      0 \\
      2 \\
      0 \\
    \end{pmatrix}, \begin{pmatrix}
      3 \\
      4 \\
      0 \\
    \end{pmatrix},
    \begin{pmatrix}
      1 \\
      0 \\
      0 \\
    \end{pmatrix} \right\}$ — линейно зависимый:

    \[
      \begin{pmatrix}
        3 \\
        4 \\
        0 \\
      \end{pmatrix} = 2
    \begin{pmatrix}
      0 \\
      2 \\
      0 \\
    \end{pmatrix} + 3
    \begin{pmatrix}
      1 \\
      0 \\
      0 \\
    \end{pmatrix}  
    \]
  

\end{frame}


\begin{frame}
  \frametitle{Линейная зависимость: дубль два}

\begin{block}{Эквивалентное определение} Набор векторов $A = \{ \bv_1, \bv_2, \ldots, \bv_k\}$ называется \alert{линейно зависимым},
  если можно найти такие веса $\alpha_1$, $\alpha_2$, \ldots, $\alpha_k$, что
  \[
  \alpha_1 \bv_1 + \alpha_2 \bv_2 + \ldots + \alpha_k \bv_k = \bzero,  
  \]
  и при этом хотя бы одно из чисел $\alpha_i$ отлично от $0$. 
\end{block}

\pause

\begin{block}{Доказательство эквивалентности}
Вектор с ненулевым коэффициентом $\alpha_i$ перед ним можно выразить через остальные. 
\pause

Если вектор $\bv_2$ выражен через $\bv_1$ и $\bv_3$, $\bv_2 = \alpha_1 \bv_1 + \alpha_3 \bv_3$, 
то искомая нулевая линейная комбинация имеет вид: $\alpha_1 \bv_1 +(-1)\bv_2 + \alpha_3 \bv_3=\bzero$.
\end{block}

\end{frame}

% !TEX root = ../linal_lecture_02.tex

\begin{frame} % название фрагмента

\videotitle{Линейная оболочка}

\end{frame}



\begin{frame}{Краткий план:}
  \begin{itemize}[<+->]
    \item Линейная оболочка векторов;
    \item Базис линейной оболочки векторов;
    \item Размерность линейной оболочки векторов.
  \end{itemize}

\end{frame}


\begin{frame}{Линейная оболочка}

\begin{block}{Определение} 
Множество векторов $V$, содержащее все возможные линейные комбинации векторов $\bv_1$, 
$\bv_2$, \ldots, $\bv_k$, называется их \alert{линейной оболочкой},
\[
  V = \Span\{ \bv_1, \bv_2, \ldots, \bv_k \}
\]
\end{block}

\end{frame}



\begin{frame}{Линейная оболочка векторов: картинка}




\begin{center}

  \begin{tikzpicture}[
    scale=1.5,
    MyPoints/.style={draw=black,fill=black,thick},
    Segments/.style={draw=blue!50!red!70,thick},
    MyCircles/.style={green!50!blue!50,thin}, 
    every node/.style={scale=1}
    ]

    %\grid;

    \clip (-1.5,-1.5) rectangle (5.5,5.5);

    \begin{scope}[cm={1,1,1.5,0,(0,0)}]
    \draw[draw=blue!30, dashed] (-1.2,-4.2) grid[step=1] (3.5,7);
    \end{scope}

    %{\verb!->!new, arrowhead = 2mm, line width=4pt}
    %, arrowhead = 3mm
    %, arrowhead = 0.2

    % Feel free to change here coordinates of points A and B
    \pgfmathparse{0}		\let\Xa\pgfmathresult
    \pgfmathparse{0}		\let\Ya\pgfmathresult
    \coordinate (A) at (\Xa,\Ya);

    \pgfmathparse{2}		\let\Xb\pgfmathresult
    \pgfmathparse{0.5}		\let\Yb\pgfmathresult
    \coordinate (B) at (\Xb,\Yb);

    \pgfmathparse{2}		\let\Xd\pgfmathresult
    \pgfmathparse{4}		\let\Yd\pgfmathresult
    \coordinate (D) at (\Xd,\Yd);

    \pgfmathparse{4}		\let\Xc\pgfmathresult
    \pgfmathparse{0}		\let\Yc\pgfmathresult
    \coordinate (C) at (\Xc,\Yc);


    \pgfmathparse{1}		\let\Xe\pgfmathresult
    \pgfmathparse{1}		\let\Ye\pgfmathresult
    \coordinate (E) at (\Xe,\Ye);

    \pgfmathparse{2.5}		\let\Xf\pgfmathresult
    \pgfmathparse{0}		\let\Yf\pgfmathresult
    \coordinate (F) at (\Xf,\Yf);

    \pgfmathparse{4}		\let\Xg\pgfmathresult
    \pgfmathparse{1}		\let\Yg\pgfmathresult
    \coordinate (G) at (\Xg,\Yg);




    \draw[-{Latex[length=4.5mm, width=2.5mm]}, >=stealth, thick] (A)--(D) node[above left]{$\bd$};

    \draw[-{Latex[length=4.5mm, width=2.5mm]}, >=stealth, vecb, thick] (A)--(E) node[right]{$\ba$};

    \draw[-{Latex[length=4.5mm, width=2.5mm]}, >=stealth, vecb, thick] (A)--(F) node[below]{$\bb$};

    \draw[-{Latex[length=4.5mm, width=2.5mm]}, >=stealth, veca, thick] (A)--(G) node[right]{$\bc$};


    \draw[black, dashed] (B)--(D);

    \fill[MyPoints]  (0,0) circle (0.8mm);

    %\node [right,darkgray] at (0.5,-2) {$\bc \in \operatorname{Lin} (\ba, \bb) $ }; 

    %\node [right,darkgray] at (0.5,-3) {$\bd \notin \operatorname{Lin} (\ba, \bb) $ }; 



    \end{tikzpicture}
  \end{center}
  
Вектор $\bc$ лежит в плоскости $\Span\{ \ba, \bb \}$.

Вектор $\bd$ не лежит в плоскости $\Span\{ \ba, \bb \}$.


\end{frame}



\begin{frame}
\frametitle{Линейная зависимость}


\begin{block}{Определение}
Набор $A$ из двух и более векторов называется 
\alert{линейно зависимым}, если хотя бы один вектор является линейной комбинацией остальных.


Набор $A = \{\bzero\}$ из одного нулевого вектора также называется \alert{линейно зависимым}.
\end{block}


\end{frame}


\begin{frame}
\frametitle{Линейная зависимость: примеры}



Набор $A = \left\{ \begin{pmatrix}
      0 \\
      2 \\
    \end{pmatrix}, \begin{pmatrix}
      3 \\
      4 \\
    \end{pmatrix} \right\}$ — линейно независимый.

\pause

Набор $A = \left\{ \begin{pmatrix}
      0 \\
      2 \\
      0 \\
    \end{pmatrix}, \begin{pmatrix}
      3 \\
      4 \\
      0 \\
    \end{pmatrix},
    \begin{pmatrix}
      1 \\
      0 \\
      0 \\
    \end{pmatrix} \right\}$ — линейно зависимый:

    \[
      \begin{pmatrix}
        3 \\
        4 \\
        0 \\
      \end{pmatrix} = 2
    \begin{pmatrix}
      0 \\
      2 \\
      0 \\
    \end{pmatrix} + 3
    \begin{pmatrix}
      1 \\
      0 \\
      0 \\
    \end{pmatrix}  
    \]
  

\end{frame}


\begin{frame}
  \frametitle{Линейная зависимость: дубль два}

\begin{block}{Эквивалентное пределение} Набор векторов $A = \{ \bv_1, \bv_2, \ldots, \bv_k\}$ называется \alert{линейно зависимым},
  если можно найти такие веса $\alpha_1$, $\alpha_2$, \ldots, $\alpha_k$, что
  \[
  \alpha_1 \bv_1 + \alpha_2 \bv_2 + \ldots + \alpha_k \bv_k = \bzero,  
  \]
  и при этом хотя бы одно из чисел $\alpha_i$ отлично от $0$. 
\end{block}

\pause

\begin{block}{Доказательство эквивалентности}
Вектор с ненулевым коэффициентом $\alpha_i$ перед ним можно выразить через остальные. 
\pause

Если вектор $\bv_2$ выражен через $\bv_1$ и $\bv_3$, $\bv_2 = \alpha_1 \bv_1 + \alpha_3 \bv_3$, 
то искомая нулевая линейная комбинация имеет вид: $\alpha_1 \bv_1 +(-1)\bv_2 + \alpha_3 \bv_3=\bzero$.
\end{block}

\end{frame}



\end{document}
