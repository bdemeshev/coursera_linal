% !TEX root = ../linal_lecture_05.tex

\begin{frame} % название фрагмента

\videotitle{Диагонализация квадратичной формы}

\end{frame}



\begin{frame}{Краткий план:}
  \begin{itemize}[<+->]
    \item Симметричная матрица и собственные числа.
    \item Диагонализация квадратичной формы.
  \end{itemize}

\end{frame}


\begin{frame}{Всегда диагонализуема!}
    \begin{block}{Утверждение}
        Если $A$ — симметричная матрица, $A^T = A$, то у неё всегда найдётся
        ровно $n$ \alert{действительных} собственных чисел $\lambda_i$ \pause 
        и ровно $n$ линейно независимых \alert{ортогональных} собственных векторов.
    \end{block}
    \pause
    \begin{block}{Следствие}
        У симметричной $A$ можно найти $n$ ортогональных собственных векторов единичной длины.

        Симметричная матрица $A$ всегда диагонализуема!
    \end{block}
\end{frame}    


\begin{frame}
    \frametitle{Чем хороши ортогональные векторы?}

    Векторы $\bv_1$, \ldots, $\bv_n$ — ортогональные и единичной длины.
    \[
    P = \begin{pmatrix}
        \vert &  & \vert \\
        \bv_1 & \ldots & \bv_n \\
        \vert &  & \vert \\
    \end{pmatrix}    \pause \quad
    P^T = \begin{pmatrix}
\text{———} \hspace{-0.2cm} & \bv_1 & \hspace{-0.2cm} \text{———} \\
 & \vdots &  \\
\text{———} \hspace{-0.2cm} & \bv_n & \hspace{-0.2cm} \text{———} \\
        \end{pmatrix}    \pause
    \]
%
    \[
    P^T P = \begin{pmatrix}        
        1 & 0 & \ldots & 0 \\
        0 & 1 & \ldots & 0 \\
        0 & 0 & \ldots & 0 \\
        0 & 0 & \ldots & 1 \\
    \end{pmatrix}  = \Id   \pause
    \]
%
    \[
    P^T = P^{-1}    
    \]
    
\end{frame}

\begin{frame}
    \frametitle{Диагонализация формы}

    \begin{block}{Утвеждение}
        Квадратичная форма $f(\bx) = \bx^T A \bx$ с симметричной $A$ представима в виде
        \[
        f(\bx) = \bx^T PDP^{-1} \bx,    
        \] 
        где $D$ — диагональная матрица из собственных чисел матрицы $A$, 
        а $P$ — матрица из линейно независимых собственных векторов матрицы $A$.
    \end{block}
    \pause
\begin{block}{Утвеждение}
    Всегда можно выбрать ортогональные собственные векторы $A$ единичной длины, 
    при этом представление примет вид:
    \[
    f(\bx) = \bx^T PDP^T \bx = (P^T \bx)^T D (P^T \bx),    
    \] 
\end{block}
    
\end{frame}


\begin{frame}
    \frametitle{Диагонализация формы}
\begin{block}{Утвеждение}
    Всегда можно выбрать ортогональные собственные векторы единичной длины, 
    при этом представление примет вид:
    \[
    f(\bx) = \bx^T PDP^T \bx = (P^T \bx)^T D (P^T \bx),    
    \] 
    где $D$ — диагональная матрица из собственных чисел матрицы $A$, 
    а $P$ — матрица из собственных векторов матрицы $A$.
\end{block}
    \pause
    Это просто удачная замена переменных $\by=P^T \bx$!\pause
    \[
    f(\bx) =  (P^T \bx)^T D (P^T \bx) = \by^T D \by = 
    \]
    \[
    = \lambda_1 y_1^2 + \lambda_2 y_2^2 + \ldots + \lambda_n y_n^2
    \]
\end{frame}

\begin{frame}
    \frametitle{Определённость формы}
    Пример, $f(\bx) = 5 y_1^2 + 6y_2^2 - 9y_3^2$. \pause

    Квадратичная форма $f$ неопределена. 
    \pause
\begin{block}{Утверждение}
    Квадратичная форма 
    \[
      f(\bx) = \lambda_1 y_1^2 + \lambda_2 y_2^2 + \ldots + \lambda_n y_n^2  
    \]
    является\ldots\pause 

    положительно определённой, если все $\lambda_i >0$.\pause

    отрицательно определённой, если все $\lambda_i <0$.\pause

    положительно полуопределённой, если все $\lambda_i \geq 0$.\pause

    отрицательно полуопределённой, если все $\lambda_i \leq 0$.\pause

    неопределённой, если найдётся $\lambda_i >0$ и $\lambda_j < 0$.
\end{block}

\end{frame}

\begin{frame}
    \frametitle{Кусочек доказательства}

    \begin{block}{Утверждение}
        Для симметричной матрицы $A$, $A^T=A$, собственные вектора, 
        соответствующие разным $\lambda$, ортогональны. 
    \end{block}

    \pause

\begin{block}{Доказательство}
К примеру, $A\bx = 5\bx$ и $A\by = 7\by$. \pause
\[
\begin{array}{c} 
    \langle A\bx, \by \rangle = \langle 5\bx, \by \rangle = 5 \langle \bx, \by \rangle \pause \\
    \langle \bx, A\by \rangle = \langle \bx, 7\by \rangle = 7 \langle \bx, \by \rangle \pause \\
    \langle A\bx, \by \rangle =  \langle \bx, A^T\by \rangle =  \langle \bx, A \by \rangle \pause \\
 \end{array}
\]

Равенство возможно, только если $\bx \perp \by$:
\[
    5 \langle \bx, \by \rangle = 7 \langle \bx, \by \rangle
\]

\end{block}


    

\end{frame}