% !TEX root = ../linal_lecture_05.tex

\begin{frame} % название фрагмента

\videotitle{Критерий Сильвестра}

\end{frame}



\begin{frame}{Краткий план:}
  \begin{itemize}[<+->]
    \item Критерий Сильвестра.
    \item Расширенный критерий Сильвестра.
  \end{itemize}

\end{frame}


\begin{frame}
    \frametitle{Обозначение}

    Будем вычёркивать из матрицы $A$ строки и столбцы с одинаковыми номерами.\pause

    Скажем, оставим в матрице $A$ только $1$-ю, $3$-ю и $5$-ю строки и 
    $1$-й, $3$-й и $5$-й столбцы. \pause

    Определитель полученной подматрицы обозначим $m_{135}$. \pause

    Пример. 
    \[
    A = \begin{pmatrix}
        5 & 2 & 3 & -1 \\
        2 & \blue{6} & 2 & \blue{1} \\
        3 & 2 & 9 & 5 \\
        -1 & \blue{1} & 5 & \blue{8} \\
    \end{pmatrix}, \; m_{24} = \begin{vmatrix}
        \blue{6} & \blue{1} \\
        \blue{1} & \blue{8} \\
    \end{vmatrix} = 47.
    \]
    
\end{frame}


\begin{frame}
    \frametitle{Критерий Сильвестра}

    \begin{block}{Утверждение}
        Симметричная матрица $A$ является положительно определённой, если и только если

        $m_1 > 0$, $m_{12} > 0$, $m_{123} > 0$, $m_{1234}>0$, \ldots   \pause     
    \end{block}

    Пример. 
\[
A = \begin{pmatrix}
    \blue{5} & 2 & \blue{3}  \\
    2 & 6 & \blue{2} \\
    \blue{3} & \blue{2} & \blue{9} \\
\end{pmatrix}
\]
\[
    m_1 = 5, \; m_{12} = \begin{vmatrix}
        5 & 2 \\
        2 & 6
    \end{vmatrix} = 26, \; 
    m_{123} = \begin{vmatrix}
        5 & 2 & 3 \\
        2 & 6 & 2 \\
        3 & 2 & 9
    \end{vmatrix}=  184
\]
    
\end{frame}


\begin{frame}
    \frametitle{Наблюдение}

    \begin{block}{Утверждение}
    Если помножить на $(-1)$ все элементы матрицы $A$ размера $n\times n$, то определитель матрица $A$\ldots \pause

    поменяет знак, если $n$ — нечётное; \pause

    сохранит знак, если $n$ — чётное. 
    \end{block}

    


\end{frame}

    

\begin{frame}
\frametitle{Критерий Сильвестра}

\begin{block}{Утверждение}
    Симметричная матрица $A$ является отрицательно определённой, если и только если

    $m_1 < 0$, $m_{12} > 0$, $m_{123} < 0$, $m_{1234}>0$, \ldots   \pause   
\end{block}

Пример. 
\[
B = \begin{pmatrix}
    \blue{-5} & -2 & \blue{-3}  \\
    -2 & -6 & \blue{-2} \\
    \blue{-3} & \blue{-2} & \blue{-9} \\
\end{pmatrix}
\]
\[
    m_1 = -5, \; m_{12} = \begin{vmatrix}
        -5 & -2 \\
        -2 & -6
    \end{vmatrix} = 26, \; 
    m_{123} = \begin{vmatrix}
        -5 & -2 & -3 \\
        -2 & -6 & -2 \\
        -3 & -2 & -9
    \end{vmatrix}=  -184
\]



\end{frame}


