\documentclass[14 pt,xcolor=dvipsnames]{beamer}

% !TEX root = linal_lecture_01.tex

\usepackage{epsdice}

\usepackage[absolute,overlay]{textpos}

\usepackage[orientation=portrait,size=custom,width=25.4,height=19.05]{beamerposter}

%25,4 см 19,05 см размеры слайда в powerpoint

\usetheme{metropolis}
\metroset{
  %progressbar=none,
  numbering=none,
  subsectionpage=progressbar,
  block=fill
}

%\usecolortheme{seahorse}

\usepackage{fontspec}
\usepackage{polyglossia}
\setmainlanguage{russian}


\usepackage{fontawesome5} % removed [fixed]
\setmainfont[Ligatures=TeX]{Myriad Pro}
\setsansfont{Myriad Pro}

% why do we need \newfontfamily:
% http://tex.stackexchange.com/questions/91507/
\newfontfamily{\cyrillicfonttt}{Myriad Pro}
\newfontfamily{\cyrillicfont}{Myriad Pro}
%\newfontfamily{\cyrillicfontbs}{Myriad Pro}
\newfontfamily{\cyrillicfontsf}{Myriad Pro}


% https://tex.stackexchange.com/questions/175860/why-does-unicode-math-break-the-kerning-of-accents-in-combination-with-amssymb
% "You shouldn't be using amssymb together with unicode-math"
\usepackage{amsmath,amsxtra,amsthm} % amssymb

%\usepackage{bm}

\usepackage{fdsymbol} % \nperp

\usepackage{unicode-math}


\usepackage{centernot}

\usepackage{graphicx}
\graphicspath{{img/}}

\usepackage{wrapfig}
\usepackage{animate}
\usepackage{tikz}
%\usetikzlibrary{shapes.geometric,patterns,positioning,matrix,calc,arrows,shapes,fit,decorations,decorations.pathmorphing}
\usepackage{pifont}
\usepackage{comment}
\usepackage[font=small,labelfont=bf]{caption}
\captionsetup[figure]{labelformat=empty}
\includecomment{techno}

\usefonttheme[onlymath]{serif}


%Расположение

\setbeamersize{text margin left=15 mm,text margin right=5mm} 
\setlength{\leftmargini}{38 pt}

%\usepackage{showframe}
%\usepackage{enumitem}
%\setlist{leftmargin=5.5mm}


%Цвета от дирекции

\definecolor{dirblack}{RGB}{58, 58, 58}
\definecolor{dirwhite}{RGB}{245, 245, 245}
\definecolor{dirred}{RGB}{149, 55, 53}
\definecolor{dirblue}{RGB}{0, 90, 171}
\definecolor{dirorange}{RGB}{235, 143, 76}
\definecolor{dirlightblue}{RGB}{75, 172, 198}
\definecolor{dirgreen}{RGB}{155, 187, 89}
\definecolor{dircomment}{RGB}{128, 100, 162}

\setbeamercolor{title separator}{bg=dirlightblue!50, fg=dirblue}

%Цвета блоков

% Голубой блок!
\setbeamercolor{block title}{bg=dirblue!30,fg=dirblack}
\setbeamercolor{block title example}{bg=dirlightblue!50,fg=dirblack}
\setbeamercolor{block body example}{bg=dirlightblue!20,fg=dirblack}

\AtBeginEnvironment{exampleblock}{\setbeamercolor{itemize item}{fg=dirblack}}
%\setbeamertemplate{blocks}[rounded][shadow]

% Набор команд для удобства верстки

\newcommand{\RR}{\mathbb{R}}
\newcommand{\ZZ}{\mathbb{Z}}
\newcommand{\la}{\lambda}

% Набор команд для структуризации

%\newcommand{\quest}{\faQuestionCircleO}
%\faPencilSquareO \faPuzzlePiece \faQuestionCircleO  \faIcon*[regular]{file} {\textcolor{dirblue}
%\newcommand{\quest}{\textcolor{dirblue}{\boxed{\textbf{?}}}
\newcommand{\task}{\faIcon{tasks}}
\newcommand{\exmpl}{\faPuzzlePiece}
\newcommand{\dfn}{\faIcon{pen-square}}
\newcommand{\quest}{\textcolor{dirblue}{\faQuestionCircle[regular]}}
\newcommand{\acc}[1]{\textcolor{dirred}{#1}}
\newcommand{\accm}[1]{\textcolor{dirred}{#1}}
\newcommand{\acct}[1]{\textcolor{dirblue}{#1}}
\newcommand{\acctm}[1]{\textcolor{dirblue}{#1}}
\newcommand{\accex}[1]{\textcolor{dirblack}{\bf #1}}
\newcommand{\accexm}[1]{\textcolor{dirblack}{ \mathbf{#1}}}
\newcommand{\acclp}[1]{\textcolor{dirorange}{\it #1}}
\newcommand{\todo}[1]{\textcolor{dircomment}{\bf #1}}
\newcommand{\graylink}[1]{{\fontsize{11}{12}\selectfont \textcolor{gray}{#1}}}
\newcommand{\figcaption}[1]{{\fontsize{18}{20}\selectfont #1}}


\newcommand{\videotitle}[1]{
    {\fontsize{33}{30}\selectfont \textcolor{dirblue}{\textbf{#1}} }

    %\todo{название видеофрагмента}
}

\newcommand{\lecturetitle}[1]{
  {\fontsize{33}{30}\selectfont \textcolor{dirblue}{\textbf{#1}} }

    %\todo{название лекции}
}





\newcommand{\spcbig}{\vspace{-10 pt}}
\newcommand{\spcsmall}{\vspace{-5 pt}}

%\usepackage{listings}
%\lstset{
%xleftmargin=0 pt,
%  basicstyle=\small, 
%  language=Python,
  %tabsize = 2,
%  backgroundcolor=\color{mc!20!white}
%}



%\newcommand{\mypart}[1]{\begin{frame}[standout]{\huge #1}\end{frame}}

\setbeamercolor{background canvas}{bg=}

% frame title setup
\setbeamercolor{frametitle}{bg=,fg=dirblue}
\setbeamertemplate{frametitle}[default][left]

\addtobeamertemplate{frametitle}{\hspace*{-0.5 cm}}{\vspace*{0.25cm}}


%Шрифты
\setbeamerfont{frametitle}{family=\rmfamily,series=\bfseries,size={\fontsize{33}{30}}}
\setbeamerfont{framesubtitle}{family=\rmfamily,series=\bfseries,size={\fontsize{26}{20}}}


% удобнее знать номер слайда, чтобы вносить правки!  

\setbeamercolor{footline}{fg=dircomment}
\setbeamerfont{footline}{series=\bfseries, size={\fontsize{12}{14}}}
%\setbeamertemplate{footline}[page number]

\defbeamertemplate{footline}{custom footline}
{%
  \hspace*{\fill}%
  \usebeamercolor[fg]{page number in head/foot}%
  \usebeamerfont{page number in head/foot}%
  page: \insertpagenumber\,/\,\insertpresentationendpage%
  \hspace{20pt}%
  slide: \insertframenumber\,/\,\inserttotalframenumber%
  %\hspace*{\fill}
  \vskip2pt%
}
\setbeamertemplate{footline}[custom footline]

\usepackage{physics}
\newcommand{\R}{\mathbb{R}}
\newcommand{\CC}{\mathbb{C}}

\newcommand{\Rot}{\mathrm{R}}
\newcommand{\HH}{\mathrm{H}}
\newcommand{\Id}{\mathrm{I}}



\usepackage[outline]{contour}


\usepackage{pgfplots}
\pgfplotsset{compat=newest}

\usepackage{tikz}
\usetikzlibrary{calc}
\usetikzlibrary{quotes,angles}
\usetikzlibrary{arrows}
\usetikzlibrary{arrows.meta}
\usetikzlibrary{positioning,intersections,decorations.markings}
\usetikzlibrary{patterns}

\usepackage{tkz-euclide} 

\newcommand{\grid}{\draw[color=gray,step=1.0,dotted] (-2.1,-2.1) grid (9.6,6.1)}

\newcommand{\ba}{\symbf{a}}
\newcommand{\be}{\symbf{e}}
\newcommand{\bb}{\symbf{b}}
\newcommand{\bc}{\symbf{c}}
\newcommand{\bd}{\symbf{d}}
\newcommand{\bx}{\symbf{x}}
\newcommand{\by}{\symbf{y}}
\newcommand{\bhy}{\symbf{\hat{y}}}
\newcommand{\bff}{\symbf{f}} % \bf is already def
\newcommand{\bv}{\symbf{v}}
\newcommand{\bzero}{\symbf{0}}
\newcommand{\red}[1]{\textcolor{red}{#1}}
\newcommand{\green}[1]{\textcolor{green}{#1}}
\newcommand{\blue}[1]{\textcolor{blue}{#1}}


\DeclareMathOperator{\eig}{Eig}

\DeclareMathOperator{\Lin}{Span}
\DeclareMathOperator{\col}{col}
\DeclareMathOperator{\row}{row}

\DeclareMathOperator{\adj}{adj}

\DeclareMathOperator{\sign}{sign}

\DeclareMathOperator{\charp}{char}

\DeclareMathOperator{\Span}{Span}
\DeclareMathOperator{\Image}{Image}


\DeclareMathOperator{\LL}{L}

%\tikzset{>=latex}

\colorlet{veca}{red}
\colorlet{vecb}{blue}
\colorlet{vecc}{olive}


\tikzset{cross/.style={cross out, draw=black, minimum size=2*(#1-\pgflinewidth), inner sep=0pt, outer sep=0pt},
%default radius will be 1pt. 
cross/.default={5pt}}





\begin{document}

% \maketitle


\begin{frame} % название лекции


\lecturetitle{Квадратичные формы}

\end{frame}


% % !TEX root = ../linal_lecture_05.tex

\begin{frame} % название фрагмента

\videotitle{Квадратичная форма}

\end{frame}



\begin{frame}{Краткий план:}
  \begin{itemize}[<+->]
    \item Определение квадратичной формы.
    \item Определённость формы.
  \end{itemize}

\end{frame}



\begin{frame}
    \frametitle{Квадратичная форма}    

    \begin{block}{Определение}
       Многочлен от нескольких переменных $f(x_1, x_2, \ldots, x_n)$, который содержит только слагаемые вида $x_i^2$ и $x_i x_j$ 
       \alert{квадратичной формой}.
    \end{block}

    \pause
    Функция $f(x,y) = x^2 + 6xy - 7y^2$ — квадратичная форма.

    \pause
    Функция $f(x, y, z) = x^2 + 6xz - 8xy + 3z + 9$ — не квадратичная форма. 
\end{frame}


\begin{frame}{Зачем нужны квадратичные формы?}
    
    Многие функции хорошо аппроксимируются суммой вида
    \[
    f(x, y) \approx 6 + 2x + 4y + 7x^2 + 8xy - 9y^2
    \]
    \pause
    Свойства квадратичной формы позволяют понять свойства многих функций!
    
\end{frame}


\begin{frame}{Квадратичная форма и матрицы}
\[
\begin{pmatrix}
    x_1 & x_2 & x_3 
\end{pmatrix} \cdot \begin{pmatrix}
    5 & \red{-1} & \blue{-3} \\
    \red{-1} & 7 & 2 \\
    \blue{-3} & 2 & 11 \\
\end{pmatrix} \cdot 
\begin{pmatrix}
    x_1 \\
     x_2 \\ 
     x_3 \\
\end{pmatrix} = \pause
\]
\[
 = 5x_1^2 + 7x_2^2 + 11x_3^2 \red{- 2}x_1 x_2 \blue{- 6}x_1 x_3 + 4 x_2 x_3 \pause
\]
    
\begin{block}{Утверждение}
    Любая квадратичная форма $f(\bx)$ может быть записана в виде 
    \[
      f(\bx) = \bx^T A \bx,  
    \]
    где $A$ — симметричная матрица, $A^T = A$.
\end{block}


\end{frame}


\begin{frame}
    \frametitle{Квадратичные формы в нуле}

    \begin{block}{Утверждение}
        Любая квадратичная форма $f$ равна $0$ в точке $\bzero$,
        \[
        f(\bzero)  = \bzero^T \cdot A \cdot \bzero = 0.
        \]
    \end{block}
    \pause
    Нас будет интересовать знак квадратичной формы $f(\bx)$ при $\bx \neq \bzero$.

\end{frame}


\begin{frame}
    \frametitle{Положительно определённая форма}

    \begin{block}{Определение}
        Форма $f$ называется \alert{положительно определённой}, если $f(\bx) > 0$ при $\bx\neq\bzero$.
    \end{block}

    

\end{frame} % долго рендерится из-за 3d графиков!

\begin{frame}
\lecturetitle{Метод полных квадратов}
\todo{Это видеофрагмент с доской, слайдов здесь нет :)}
\end{frame}
    

% !TEX root = ../linal_lecture_05.tex

\begin{frame} % название фрагмента

\videotitle{Диагонализация квадратичной формы}

\end{frame}



\begin{frame}{Краткий план:}
  \begin{itemize}[<+->]
    \item Симметричная матрица и собственные числа.
    \item Диагонализация квадратичной формы.
  \end{itemize}

\end{frame}


\begin{frame}{Всегда диагонализуема!}
    \begin{block}{Утверждение}
        Если $A$ — симметричная матрица, $A^T = A$, то у неё всегда найдётся
        ровно $n$ \alert{действительных} собственных чисел $\lambda_i$ \pause 
        и ровно $n$ линейно независимых \alert{ортогональных} собственных векторов.
    \end{block}
    \pause
    \begin{block}{Следствие}
        У симметричной $A$ можно найти $n$ ортогональных собственных векторов единичной длины.

        Симметричная матрица $A$ всегда диагонализуема!
    \end{block}
\end{frame}    


\begin{frame}
    \frametitle{Чем хороши ортогональные векторы?}

    Векторы $\bv_1$, \ldots, $\bv_n$ — ортогональные и единичной длины.
    \[
    P = \begin{pmatrix}
        \vert &  & \vert \\
        \bv_1 & \ldots & \bv_n \\
        \vert &  & \vert \\
    \end{pmatrix}    \pause \quad
    P^T = \begin{pmatrix}
\text{———} \hspace{-0.2cm} & \bv_1 & \hspace{-0.2cm} \text{———} \\
 & \vdots &  \\
\text{———} \hspace{-0.2cm} & \bv_n & \hspace{-0.2cm} \text{———} \\
        \end{pmatrix}    \pause
    \]
%
    \[
    P^T P = \begin{pmatrix}        
        1 & 0 & \ldots & 0 \\
        0 & 1 & \ldots & 0 \\
        0 & 0 & \ldots & 0 \\
        0 & 0 & \ldots & 1 \\
    \end{pmatrix}  = \Id   \pause
    \]
%
    \[
    P^T = P^{-1}    
    \]
    
\end{frame}

\begin{frame}
    \frametitle{Диагонализация формы}

    \begin{block}{Утвеждение}
        Квадратичная форма $f(\bx) = \bx^T A \bx$ с симметричной $A$ представима в виде
        \[
        f(\bx) = \bx^T PDP^{-1} \bx,    
        \] 
        где $D$ — диагональная матрица из собственных чисел матрицы $A$, 
        а $P$ — матрица из линейно независимых собственных векторов матрицы $A$.
    \end{block}
    \pause
\begin{block}{Утвеждение}
    Всегда можно выбрать ортогональные собственные векторы $A$ единичной длины, 
    при этом представление примет вид:
    \[
    f(\bx) = \bx^T PDP^T \bx = (P^T \bx)^T D (P^T \bx),    
    \] 
\end{block}
    
\end{frame}


\begin{frame}
    \frametitle{Диагонализация формы}
\begin{block}{Утвеждение}
    Всегда можно выбрать ортогональные собственные векторы единичной длины, 
    при этом представление примет вид:
    \[
    f(\bx) = \bx^T PDP^T \bx = (P^T \bx)^T D (P^T \bx),    
    \] 
    где $D$ — диагональная матрица из собственных чисел матрицы $A$, 
    а $P$ — матрица из собственных векторов матрицы $A$.
\end{block}
    \pause
    Это просто удачная замена переменных $\by=P^T \bx$!\pause
    \[
    f(\bx) =  (P^T \bx)^T D (P^T \bx) = \by^T D \by = 
    \]
    \[
    = \lambda_1 y_1^2 + \lambda_2 y_2^2 + \ldots + \lambda_n y_n^2
    \]
\end{frame}

\begin{frame}
    \frametitle{Определённость формы}
    Пример, $f(\bx) = 5 y_1^2 + 6y_2^2 - 9y_3^2$. \pause

    Квадратичная форма $f$ неопределена. 
    \pause
\begin{block}{Утверждение}
    Квадратичная форма 
    \[
      f(\bx) = \lambda_1 y_1^2 + \lambda_2 y_2^2 + \ldots + \lambda_n y_n^2  
    \]
    является\ldots\pause 

    положительно определённой, если все $\lambda_i >0$.\pause

    отрицательно определённой, если все $\lambda_i <0$.\pause

    положительно полуопределённой, если все $\lambda_i \geq 0$.\pause

    отрицательно полуопределённой, если все $\lambda_i \leq 0$.\pause

    неопределённой, если найдётся $\lambda_i >0$ и $\lambda_j < 0$.
\end{block}

\end{frame}

\begin{frame}
    \frametitle{Кусочек доказательства}

    \begin{block}{Утверждение}
        Для симметричной матрицы $A$, $A^T=A$, собственные вектора, 
        соответствующие разным $\lambda$, ортогональны. 
    \end{block}

    \pause

\begin{block}{Доказательство}
К примеру, $A\bx = 5\bx$ и $A\by = 7\by$. \pause
\[
\begin{array}{c} 
    \langle A\bx, \by \rangle = \langle 5\bx, \by \rangle = 5 \langle \bx, \by \rangle \pause \\
    \langle \bx, A\by \rangle = \langle \bx, 7\by \rangle = 7 \langle \bx, \by \rangle \pause \\
    \langle A\bx, \by \rangle =  \langle \bx, A^T\by \rangle =  \langle \bx, A \by \rangle \pause \\
 \end{array}
\]

Равенство возможно, только если $\bx \perp \by$:
\[
    5 \langle \bx, \by \rangle = 7 \langle \bx, \by \rangle
\]

\end{block}


    

\end{frame}

% !TEX root = ../linal_lecture_05.tex

\begin{frame} % название фрагмента

\videotitle{Критерий Сильвестра}

\end{frame}



\begin{frame}{Краткий план:}
  \begin{itemize}[<+->]
    \item Критерий Сильвестра.
    \item Расширенный критерий Сильвестра.
  \end{itemize}

\end{frame}


\begin{frame}
    \frametitle{Обозначение}

    Будем вычёркивать из матрицы $A$ строки и столбцы с одинаковыми номерами.\pause

    Скажем, оставим в матрице $A$ только $1$-ю, $3$-ю и $5$-ю строки и 
    $1$-й, $3$-й и $5$-й столбцы. \pause

    Определитель полученной подматрицы обозначим $m_{135}$. \pause

    Пример. 
    \[
    A = \begin{pmatrix}
        5 & 2 & 3 & -1 \\
        2 & \blue{6} & 2 & \blue{1} \\
        3 & 2 & 9 & 5 \\
        -1 & \blue{1} & 5 & \blue{8} \\
    \end{pmatrix}, \; m_{24} = \begin{vmatrix}
        \blue{6} & \blue{1} \\
        \blue{1} & \blue{8} \\
    \end{vmatrix} = 47.
    \]
    
\end{frame}


\begin{frame}
    \frametitle{Критерий Сильвестра}

    \begin{block}{Утверждение}
        Симметричная матрица $A$ является положительно определённой, если и только если

        $m_1 > 0$, $m_{12} > 0$, $m_{123} > 0$, $m_{1234}>0$, \ldots   \pause     
    \end{block}

    Пример. 
\[
A = \begin{pmatrix}
    \blue{5} & 2 & \blue{3}  \\
    2 & 6 & \blue{2} \\
    \blue{3} & \blue{2} & \blue{9} \\
\end{pmatrix}
\]
\[
    m_1 = 5, \; m_{12} = \begin{vmatrix}
        5 & 2 \\
        2 & 6
    \end{vmatrix} = 26, \; 
    m_{123} = \begin{vmatrix}
        5 & 2 & 3 \\
        2 & 6 & 2 \\
        3 & 2 & 9
    \end{vmatrix}=  184
\]
    
\end{frame}


\begin{frame}
    \frametitle{Наблюдение}

    \begin{block}{Утверждение}
    Если помножить на $(-1)$ все элементы матрицы $A$ размера $n\times n$, то определитель матрица $A$\ldots \pause

    поменяет знак, если $n$ — нечётное; \pause

    сохранит знак, если $n$ — чётное. 
    \end{block}

    


\end{frame}

    

\begin{frame}
\frametitle{Критерий Сильвестра}

\begin{block}{Утверждение}
    Симметричная матрица $A$ является отрицательно определённой, если и только если

    $m_1 < 0$, $m_{12} > 0$, $m_{123} < 0$, $m_{1234}>0$, \ldots   \pause   
\end{block}

Пример. 
\[
B = \begin{pmatrix}
    \blue{-5} & -2 & \blue{-3}  \\
    -2 & -6 & \blue{-2} \\
    \blue{-3} & \blue{-2} & \blue{-9} \\
\end{pmatrix}
\]
\[
    m_1 = -5, \; m_{12} = \begin{vmatrix}
        -5 & -2 \\
        -2 & -6
    \end{vmatrix} = 26, \; 
    m_{123} = \begin{vmatrix}
        -5 & -2 & -3 \\
        -2 & -6 & -2 \\
        -3 & -2 & -9
    \end{vmatrix}=  -184
\]



\end{frame}





\begin{frame}
\lecturetitle{Расширенный критерий Сильвестра: пример}
\todo{Это видеофрагмент с доской, слайдов здесь нет :)}
\end{frame}

% !TEX root = ../linal_lecture_05.tex

\begin{frame} % название фрагмента

\videotitle{Матрица Грама}

\end{frame}



\begin{frame}{Краткий план:}
  \begin{itemize}[<+->]
    \item Матрица Грама.
    \item Матрица Грама и проекция.
    \item Ортогональный базис.
  \end{itemize}

\end{frame}

\begin{frame}
    \frametitle{Матрица Грама}

    \begin{block}{Определение}
        Возьмём векторы $\bx_1$, $\bx_2$, \ldots, $\bx_k$ из $\R^n$. 
        Матрица их попарных скалярных произведений называется \alert{матрицей Грама},
        \[
            M =  \begin{pmatrix}
                \langle \bx_1, \bx_1 \rangle &  \langle \bx_1, \bx_2 \rangle & \ldots & \langle \bx_1, \bx_k \rangle \\
                \langle \bx_2, \bx_1 \rangle &  \langle \bx_2, \bx_2 \rangle & \ldots & \langle \bx_2, \bx_k \rangle \\
                \ldots & \ldots & \ldots & \ldots \\
                \langle \bx_k, \bx_1 \rangle &  \langle \bx_k, \bx_2 \rangle & \ldots & \langle \bx_k, \bx_k \rangle \\ 
            \end{pmatrix} = X^T X\pause
        \]      

        А определитель этой матрицы называется \alert{определителем Грама}, $G=\det M$.
    \end{block}


\end{frame}


\begin{frame}
    \frametitle{Свойства матрицы Грама}

    \begin{block}{Утверждение}
        Векторы $\bx_1$, $\bx_2$, \ldots, $\bx_k$ линейно независимы если и только если определитель Грама отличен от нуля, $G\neq 0$. \pause
    \end{block}

    \begin{block}{Утверждение}
        Матрица Грама положительно полуопределена. \pause
    \end{block}


    \begin{block}{Утверждение}
        Если $\bx_1$, $\bx_2$, \ldots, $\bx_n$ лежат в $\R^n$, то определитель Грама $G$ равен квадрату объёма параллелепипеда, 
        образованного векторами $\bx_1$,  $\bx_2$, \ldots, $\bx_n$.
    \end{block}

\end{frame}

\begin{frame}{Положительная полуопределённость}

    \begin{block}{Утверждение}
        Матрица Грама положительно полуопределена. \pause
    \end{block} 

    \begin{block}{Доказательство}
        \[
        \bv^T M \bv  = \sum_{ij} v_i v_j \langle \bx_i, \bx_j \rangle = \sum_{ij} \langle v_i \bx_i, v_j \bx_j \rangle =   \pause
        \]
        \[
        =  \langle \sum_i v_i  \bx_i, \sum_j v_j \bx_j \rangle = \langle \ba, \ba \rangle \geq 0
        \]
    \end{block}
    
\end{frame}



\begin{frame}
    \frametitle{Поиск проекции}
    Хотим найти проекцию $\bhy$ вектора  $\by$ на $\Span\{ \bx_1, \bx_2, \ldots, \bx_k \}$. \pause

    Проекция $\bhy$ — линейная комбинация $\bx_1, \bx_2, \ldots, \bx_k$,
\[
\bhy = v_1 \bx_1 + \ldots + v_k \bx_k = X \bv    \pause
\]
Условия первого порядка:
\[
X^T X  \bv = X^T y \pause \; \text{ или } \; M \bv = X^Ty    \pause 
\]    
\[
    \bv = M^{-1} X^Ty.    
\]
\end{frame}


\begin{frame}
    \frametitle{Ортогональные вектора}

    \begin{block}{Утверждение}
        Если векторы $\bx_1$, $\bx_2$, \ldots, $\bx_k$ ортогональны, то их матрица Грама —
        диагональная.
        \[
            M  = \begin{pmatrix}
                \langle \bx_1, \bx_1 \rangle & 0 & \ldots & 0 \\
                0 & \langle \bx_2, \bx_2 \rangle &  \ldots & 0 \\
                \ldots & \ldots & \ldots & \ldots \\
                0 & 0 & \ldots & \langle \bx_k, \bx_k \rangle \\
            \end{pmatrix}
        \]      

    \end{block}
    

\end{frame}


\begin{frame}
\lecturetitle{Ортогонализация Грамма-Шмидта: пример}
\todo{Это видеофрагмент с доской, слайдов здесь нет :)}
\end{frame}


% \input{chapters/lec_040_fra_080_orthogonalization.tex}


% \input{chapters/lec_040_fra_090_qr.tex}


\begin{frame}
\lecturetitle{Бонус: задача про переливание красок}
\todo{Это видеофрагмент с доской, слайдов здесь нет :)}
\end{frame}
    


\end{document}
