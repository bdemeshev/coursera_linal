\documentclass[14 pt,xcolor=dvipsnames]{beamer}

% !TEX root = linal_lecture_01.tex

\usepackage{epsdice}

\usepackage[absolute,overlay]{textpos}

\usepackage[orientation=portrait,size=custom,width=25.4,height=19.05]{beamerposter}

%25,4 см 19,05 см размеры слайда в powerpoint

\usetheme{metropolis}
\metroset{
  %progressbar=none,
  numbering=none,
  subsectionpage=progressbar,
  block=fill
}

%\usecolortheme{seahorse}

\usepackage{fontspec}
\usepackage{polyglossia}
\setmainlanguage{russian}


\usepackage{fontawesome5} % removed [fixed]
\setmainfont[Ligatures=TeX]{Myriad Pro}
\setsansfont{Myriad Pro}

% why do we need \newfontfamily:
% http://tex.stackexchange.com/questions/91507/
\newfontfamily{\cyrillicfonttt}{Myriad Pro}
\newfontfamily{\cyrillicfont}{Myriad Pro}
%\newfontfamily{\cyrillicfontbs}{Myriad Pro}
\newfontfamily{\cyrillicfontsf}{Myriad Pro}


% https://tex.stackexchange.com/questions/175860/why-does-unicode-math-break-the-kerning-of-accents-in-combination-with-amssymb
% "You shouldn't be using amssymb together with unicode-math"
\usepackage{amsmath,amsxtra,amsthm} % amssymb

%\usepackage{bm}

\usepackage{fdsymbol} % \nperp

\usepackage{unicode-math}


\usepackage{centernot}

\usepackage{graphicx}
\graphicspath{{img/}}

\usepackage{wrapfig}
\usepackage{animate}
\usepackage{tikz}
%\usetikzlibrary{shapes.geometric,patterns,positioning,matrix,calc,arrows,shapes,fit,decorations,decorations.pathmorphing}
\usepackage{pifont}
\usepackage{comment}
\usepackage[font=small,labelfont=bf]{caption}
\captionsetup[figure]{labelformat=empty}
\includecomment{techno}

\usefonttheme[onlymath]{serif}


%Расположение

\setbeamersize{text margin left=15 mm,text margin right=5mm} 
\setlength{\leftmargini}{38 pt}

%\usepackage{showframe}
%\usepackage{enumitem}
%\setlist{leftmargin=5.5mm}


%Цвета от дирекции

\definecolor{dirblack}{RGB}{58, 58, 58}
\definecolor{dirwhite}{RGB}{245, 245, 245}
\definecolor{dirred}{RGB}{149, 55, 53}
\definecolor{dirblue}{RGB}{0, 90, 171}
\definecolor{dirorange}{RGB}{235, 143, 76}
\definecolor{dirlightblue}{RGB}{75, 172, 198}
\definecolor{dirgreen}{RGB}{155, 187, 89}
\definecolor{dircomment}{RGB}{128, 100, 162}

\setbeamercolor{title separator}{bg=dirlightblue!50, fg=dirblue}

%Цвета блоков

% Голубой блок!
\setbeamercolor{block title}{bg=dirblue!30,fg=dirblack}
\setbeamercolor{block title example}{bg=dirlightblue!50,fg=dirblack}
\setbeamercolor{block body example}{bg=dirlightblue!20,fg=dirblack}

\AtBeginEnvironment{exampleblock}{\setbeamercolor{itemize item}{fg=dirblack}}
%\setbeamertemplate{blocks}[rounded][shadow]

% Набор команд для удобства верстки

\newcommand{\RR}{\mathbb{R}}
\newcommand{\ZZ}{\mathbb{Z}}
\newcommand{\la}{\lambda}

% Набор команд для структуризации

%\newcommand{\quest}{\faQuestionCircleO}
%\faPencilSquareO \faPuzzlePiece \faQuestionCircleO  \faIcon*[regular]{file} {\textcolor{dirblue}
%\newcommand{\quest}{\textcolor{dirblue}{\boxed{\textbf{?}}}
\newcommand{\task}{\faIcon{tasks}}
\newcommand{\exmpl}{\faPuzzlePiece}
\newcommand{\dfn}{\faIcon{pen-square}}
\newcommand{\quest}{\textcolor{dirblue}{\faQuestionCircle[regular]}}
\newcommand{\acc}[1]{\textcolor{dirred}{#1}}
\newcommand{\accm}[1]{\textcolor{dirred}{#1}}
\newcommand{\acct}[1]{\textcolor{dirblue}{#1}}
\newcommand{\acctm}[1]{\textcolor{dirblue}{#1}}
\newcommand{\accex}[1]{\textcolor{dirblack}{\bf #1}}
\newcommand{\accexm}[1]{\textcolor{dirblack}{ \mathbf{#1}}}
\newcommand{\acclp}[1]{\textcolor{dirorange}{\it #1}}
\newcommand{\todo}[1]{\textcolor{dircomment}{\bf #1}}
\newcommand{\graylink}[1]{{\fontsize{11}{12}\selectfont \textcolor{gray}{#1}}}
\newcommand{\figcaption}[1]{{\fontsize{18}{20}\selectfont #1}}


\newcommand{\videotitle}[1]{
    {\fontsize{33}{30}\selectfont \textcolor{dirblue}{\textbf{#1}} }

    %\todo{название видеофрагмента}
}

\newcommand{\lecturetitle}[1]{
  {\fontsize{33}{30}\selectfont \textcolor{dirblue}{\textbf{#1}} }

    %\todo{название лекции}
}





\newcommand{\spcbig}{\vspace{-10 pt}}
\newcommand{\spcsmall}{\vspace{-5 pt}}

%\usepackage{listings}
%\lstset{
%xleftmargin=0 pt,
%  basicstyle=\small, 
%  language=Python,
  %tabsize = 2,
%  backgroundcolor=\color{mc!20!white}
%}



%\newcommand{\mypart}[1]{\begin{frame}[standout]{\huge #1}\end{frame}}

\setbeamercolor{background canvas}{bg=}

% frame title setup
\setbeamercolor{frametitle}{bg=,fg=dirblue}
\setbeamertemplate{frametitle}[default][left]

\addtobeamertemplate{frametitle}{\hspace*{-0.5 cm}}{\vspace*{0.25cm}}


%Шрифты
\setbeamerfont{frametitle}{family=\rmfamily,series=\bfseries,size={\fontsize{33}{30}}}
\setbeamerfont{framesubtitle}{family=\rmfamily,series=\bfseries,size={\fontsize{26}{20}}}


% удобнее знать номер слайда, чтобы вносить правки!  

\setbeamercolor{footline}{fg=dircomment}
\setbeamerfont{footline}{series=\bfseries, size={\fontsize{12}{14}}}
%\setbeamertemplate{footline}[page number]

\defbeamertemplate{footline}{custom footline}
{%
  \hspace*{\fill}%
  \usebeamercolor[fg]{page number in head/foot}%
  \usebeamerfont{page number in head/foot}%
  page: \insertpagenumber\,/\,\insertpresentationendpage%
  \hspace{20pt}%
  slide: \insertframenumber\,/\,\inserttotalframenumber%
  %\hspace*{\fill}
  \vskip2pt%
}
\setbeamertemplate{footline}[custom footline]



\begin{document}

% \maketitle


\begin{frame} % название лекции


\lecturetitle{Векторы и действия с ними}


\end{frame}


\begin{frame} % название фрагмента

\videotitle{Вектор: длина и скалярное произведение}



\end{frame}




\begin{frame}{Парадокс Монти Холла}


    Рассмотрим этот вопрос с точки зрения вероятности:
    
    
    \begin{block}{\task Задача}
    С какой вероятностью Вы выиграете автомобиль, если
    
    \begin{itemize}[<+->]
    \item[\bf 1)] не будете менять выбор;
    
    \item[\bf 2)] измените выбор случайным образом;
    
    \item[\bf 3)] измените свой выбор?
    \end{itemize}
    
    \end{block}
    
    \pause[\thebeamerpauses]
    \quest Какая стратегия является выигрышной на Ваш взгляд?
    
    \end{frame}
    



\end{document}









\title{Классическая и дискретная вероятность}

\AtBeginSection[]{\frame{\frametitle{Конечное вероятностное пространство}\tableofcontents[current]}}

\date{}
%\author{Промыслов Валентин Валерьевич}
%\institute{ФКН, НИУ ВШЭ}

\begin{document}

\maketitle

\section{Как мы понимаем вероятность?}

\begin{frame}{Как мы будем понимать вероятность?}




\begin{wrapfigure}{r}{0.4\textwidth}
    \includegraphics[width=0.39\textwidth]{dice.png}
 
\end{wrapfigure}

Все началось с азартных игр. %де Мере


\begin{itemize}[<+->]
\item Стоит ли ставить на то, что \acct{хотя бы раз} из четырех бросков кости выпадет шестерка \epsdice{6}\hspace{1pt}?
\item А если бросать по две кости $24$ раза и ставить на выпадение \acct{двойной шестерки}?

\end{itemize} 



\end{frame}

\begin{frame}{Как мы будем понимать вероятность?}

\begin{wrapfigure}{r}{0.35\textwidth}

\vspace{-0 pt}
\begin{tikzpicture} 

\begin{scope}
    \clip [rounded corners=.5cm] (0,0) rectangle coordinate (centerpoint) (7.7,7.7cm); 
    \node [inner sep=0pt] at (centerpoint) {\includegraphics[width=0.33\textwidth]{large-preview-blaise.pascal}}; 
\end{scope}


\end{tikzpicture}
\caption{ \bf \normalsize Блез Паскаль}
   
\end{wrapfigure}

\phantom{Наблюдения Паскаля:}

\pause

Наблюдения Паскаля:

\begin{itemize}[<+->]

\item Всего~существует~$\acctm{6^4=1296}$ различных~способов бросить кость $4$ раза.

\item Шестерка \epsdice{6} выпадает при четырех бросках в $\acctm{\approx 51.77\%}$ случаев.

\item Бросить два кубика $24$ раза можно $\acctm{36^{24}}$ способами.

\item Двойная шестерка \epsdice{6} \epsdice{6} выпадает при $24$ бросках в $\acctm{\approx 49.14 \%}$ случаев.

\end{itemize}

\end{frame}

\begin{frame}{Парадокс Монти Холла}

Перед Вами три двери: за одной автомобиль, за двумя другими --- козы. \phantom{ за котоой за котоой находится коза.й находится коза.}
\vspace{-30 pt}
\begin{center}
\includegraphics[width=0.6\textwidth]{doors0-3.png}
\end{center}

\end{frame}

\begin{frame}{Парадокс Монти Холла}

Вы выбираете одну из дверей, но ведущий не говорит Вам, что за ней находится, а открывает \acct{другую дверь}, за которой находится коза.
\vspace{-30 pt}
\begin{center}
\includegraphics[width=0.6\textwidth]{doors01-3.png}
\end{center}


\end{frame}


\begin{frame}{Парадокс Монти Холла}

После этого он задает Вам вопрос:\\
"Не хотите ли Вы изменить свой выбор?"\\
\phantom{открывает, за которой находится коза.}
\vspace{-30 pt}
\begin{center}
\includegraphics[width=0.6\textwidth]{doors1-3.png}
\end{center}

\end{frame}




\begin{frame}{Парадокс Монти Холла}


Рассмотрим этот вопрос с точки зрения вероятности:


\begin{block}{\task Задача}
С какой вероятностью Вы выиграете автомобиль, если

\begin{itemize}[<+->]
\item[\bf 1)] не будете менять выбор;

\item[\bf 2)] измените выбор случайным образом;

\item[\bf 3)] измените свой выбор?
\end{itemize}

\end{block}

\pause[\thebeamerpauses]
\quest Какая стратегия является выигрышной на Ваш взгляд?

\end{frame}

\begin{frame}{Парадокс Монти Холла}

Мы будем решать задачу в предположении, что:

\begin{itemize}[<+->]
\item автомобиль размещён за \acct{любой} из дверей с одинаковой вероятностью;
\item ведущий \acct{знает, где находится автомобиль};
\item ведущий в любом случае открывает дверь \acct{с козой} (но не ту, которую выбрал игрок) и предлагает игроку изменить выбор;
\item если у ведущего есть выбор, то он выбирает \acct{любую} из двух дверей с одинаковой вероятностью.

\end{itemize}



\end{frame}


\begin{frame}{Парадокс Монти Холла}

Если не менять выбор, то выиграть можно только в том случае, если с самого начала была выбрана дверь с автомобилем. \phantom{дверь с автомобилем.}

\centerline{\includegraphics[width=0.6\textwidth]{doors3-3.png}}
\vspace{-15 mm}
\acct{Победа обеспечена в $\frac{1}{3}$ случаев.} 

\phantom{Невероятно, но всегда стоит менять свой выбор!}

\end{frame}

\begin{frame}{Парадокс Монти Холла}

%Если не менять выбор, то выиграть можно только в том случае, если с самого начала была выбрана дверь с автомобилем.

Если изменить выбор случайным образом, то придется выбирать из двух дверей. \phantom{была выбрана дверь с автомобилем.}

\centerline{\includegraphics[width=0.6\textwidth]{doors4-3.png}}
\vspace{-15 mm}
\acct{Победа обеспечена в половине случаев.}\phantom{$\frac{1}{3}$}

\phantom{Невероятно, но всегда стоит менять свой выбор!}

\end{frame}

\begin{frame}{Парадокс Монти Холла}

Если всегда менять свой выбор, то проиграть можно только в том случае, если в начале была выбрана дверь с автомобилем. \phantom{дверь с автомобилем.}

\centerline{\includegraphics[width=0.6\textwidth]{doors5-3.png}}
\vspace{-15 mm}
\acct{Победа обеспечена в $\frac{2}{3}$ случаев.}\phantom{$\frac{1}{3}$}
\pause

Невероятно, но всегда стоит менять свой выбор!


\end{frame}

\begin{frame}{Выводы}

\begin{itemize}[<+->]
\item Эмпирически можно понимать вероятность как частоту наступления некоторого события.
\item Вероятность не всегда интуитивна, но эффективна.
\end{itemize}

\end{frame}

\section{Конечное вероятностное пространство}

\begin{frame}{Элементарные исходы}

Назовем все возможные результаты случайного эксперимента \acc{элементарными исходами}.
\vskip 10 pt
\pause 

\begin{wrapfigure}[2]{r}{0.4\textwidth}
    \vspace{-50 pt} \includegraphics[width=0.38\textwidth]{coins2.png}
 
\end{wrapfigure}

Подбрасывание монеты:

орел или решка.


\pause


\begin{wrapfigure}{r}{0.4\textwidth}
    \vspace{-50 pt} \includegraphics[width=0.38\textwidth]{die4.png}
 
\end{wrapfigure}

\vskip 90 pt

Бросок кубика:

\epsdice{1},\epsdice{2},\epsdice{3},\epsdice{4},\epsdice{5} или \epsdice{6}.

\end{frame}

\begin{frame}{Пространство элементарных событий}

\begin{block}{\dfn Определение}
\acc{Пространство элементарных событий} --- множество $\Omega=\{\omega_1, \omega_2,\dots,\omega_n\}$ всех возможных исходов случайного эксперимента. 
\end{block}
\pause

\begin{itemize}[<+->]
\item При подбрасывании монеты 
$$\Omega_M=\{\includegraphics[scale=0.11]{coinT1.png},\includegraphics[scale=0.11]{coinH1.png}\}=\{"O", "P"\};$$

\item При броске кубика $$\Omega_K=\{\epsdice{1},\epsdice{2},\epsdice{3},\epsdice{4},\epsdice{5},\epsdice{6} \}=\{1,2,3,4,5,6\}$$
\end{itemize}

\end{frame}


\begin{frame}{Конечное вероятностное пространство}

Сопоставим каждому элементарному событию $\omega_i$ его вероятность $P(\omega_i)=p_i$,\quad $0\leqslant p_i\leqslant 1$ так, чтобы $$\sum_{i=1}^n p_i=1.$$
Функция $P$\ называется \acc{функцией вероятности}.

\pause
\begin{block}{\dfn Определение}
\acc{Конечное вероятностное пространство} --- это пространство элементарных событий $\Omega$ вместе с функцией вероятности $P$ на нём.
\end{block}

\end{frame}


\begin{frame}{Конечное вероятностное пространство}

\begin{itemize}[<+->]
\item Подбрасывание монеты: $$\Omega_M=\{"O", "P"\},\quad P("O")=P("P")=1/2;$$

\item Бросок кубика:
$$\Omega_K=\{1, 2,3,4,5,6 \},$$
$$P(1)=P(2)=\dots=P(6)=1/6;$$

\item Подбрасывание двух монет:
$$\Omega_{MM}=\{"OO","OP","PO", "PP"\},$$
$$P("OO")=P("OP")=P("PO")=P("PP")=1/4.$$
\end{itemize}

\end{frame}

\begin{frame}{Событие}

\begin{block}{\dfn Определение}
\acc{Событие} --- любое подмножество $\Omega.$
\end{block}
\pause


\begin{itemize}[<+->]
\item Событие <<на кубике выпало четное число очков>> --- это множество $\{\epsdice{2},\epsdice{4},\epsdice{6}\}=\{2,4,6\}\subset \Omega_K.$
\item Событие <<при подбрасывании двух монеток выпал хотя бы один орёл>> --- это множество $\{ "OO","OP","PO"\}\subset \Omega_M.$
\end{itemize}

\end{frame}

\begin{frame}{Вероятность события}

\begin{block}{\dfn Определение}
\acc{Вероятность события} --- сумма вероятностей всех его элементарных исходов.
\end{block}

\pause

\begin{itemize}[<+->]
\item Вероятность события $A$ --- <<на кубике выпало четное число очков>> равняется 
$P(A)=P(2)+P(4)+ P(6)=$

$=1/6+1/6+1/6=1/2.$
\end{itemize}

\pause
\quest Чему равняется вероятность события $B$ --- <<при подбрасывании двух монеток выпал хотя бы один орёл>>?

\pause

$P(B)=P("OO")+P("OP")+P("PO")=$
$=1/4+1/4+1/4=3/4.$



%\pause Но не всегда все исходы равновероятны!

\end{frame}

\begin{frame}{Вероятностные пространства}

\begin{center}
\includegraphics[width=\textwidth]{probspace3.pdf}
\end{center}

\pause

В классическом вероятностном пространстве найти вероятность события совсем просто:
$$P(A)=\frac{|A|}{|\Omega|}.$$


\end{frame}

\begin{frame}{Выводы}

\begin{itemize}[<+->]
\item Конечное вероятностное пространство позволяет описать эксперименты с конечным числом исходов.
\item Простейшая модель конечного вероятностного пространства --- классическая вероятность.
\item Вероятность события --- сумма вероятностей всех входящих в него исходов.
\item Не всегда все исходы равновероятны.
\end{itemize}

\end{frame}

\section{События как множества}

\begin{frame}{События как множества}

Поскольку событие --- это множество, к нему можно применять теоретико-множественные операции!

\begin{block}{\exmpl Пример}
Пусть $A$ --- <<число очков на кубике четно>>,  $A=\{2,4,6\}$;

$B$ --- <<число очков на кубике больше трех>>, $B=\{4,5,6\}$).
\pause

\begin{itemize}[<+->]
\item $A\cap B=\{4,6\}$ --- <<число очков на кубике четно \accex{и} больше трех>>.
\item $A\cup B=\{2,4,5,6\}$ --- <<число очков на кубике четное \accex{или} больше трех>>.
\item $A\backslash B=\{2\}$ --- <<число очков на кубике четное \accex{но не} больше трех>>
\end{itemize}

\end{block}
 



\end{frame}

\begin{frame}{События как множества}

В теории вероятностей используется своя терминология для операций:

\begin{itemize}[<+->]
\item объединение событий $A\cap B$ называют их \acct{суммой};
\item пересечение событий $A\cup B$ --- \acct{произведением};
\item дополнение $A^c=\bar{A}=\Omega \backslash A$ --- \acct{противоположным событием};
\item если $A \subseteq B$, то говорят, что событие $A$ \acct{влечет} событие $B.$
\end{itemize}

\end{frame}



\begin{frame}{Основные свойства вероятности}

\begin{itemize}[<+->]
\item $P(A^c)=1-P(A)$, в частности, $P(\varnothing)=0$ и $P(A)\leqslant 1$ для любого $A$;
\item если $A \subseteq B$, то $P(A) \leqslant P(B)$ и $ P(B \backslash A)= P(B)- P(A)$;
\item для любых событий $A$ и $B$
$$P(A \cup B)=P(A)+P(B)-P(A \cap B).$$
\end{itemize}

\end{frame}

\begin{frame}{Выводы}

\begin{itemize}[<+->]
\item Язык множеств имеет точную интерпретацию в терминах вероятности.
\item Равенства и неравенства на мощности множеств остаются верны, если вместо множества иметь в виду событие, а вместо количества элементов -- вероятность.
\end{itemize}

\end{frame}


\section{Решение задачи кавалера де Мере}


%%%%%%%%%%%%%%%%%%%%%%%%%%%%%%%%%%%%%%%%%%%%%%%%%%%%%%%%%%%%%%%%%%%%%%%%%%%%%%%%%%

\end{document}
