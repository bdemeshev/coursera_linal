% !TEX root = ../linal_lecture_01.tex


\begin{frame} % название фрагмента

\videotitle{Транспонирование оператора и ортогональность}

\end{frame}

\begin{frame}{Транспонирование}

У любого оператора $\LL$ есть брат $\LL^T$.

Определение. \alert{Транспонированным оператором}, $\LL^T$, называется оператор, для которого

$\langle \LL \ba, \bb\rangle = \langle \ba, \LL^T \bb\rangle$
    

\end{frame}



\begin{frame}{Транспонирование}


Почему $\LL$ и $\LL^T$ братья? 

\begin{itemize}[<+->]
    \item $(\LL^T)^T = \LL$
    \item Доказательство:
    
    \item $\langle \LL \ba , \bb \rangle = \langle  \ba , \LL^T \bb \rangle$  
    
    \item $\langle \ba , \LL \bb \rangle = \langle \LL^T \ba,  \bb \rangle$  

    \item $\langle \LL^T \ba,  \bb \rangle = \langle \ba , \LL \bb \rangle $  

\end{itemize}


\end{frame}




\begin{frame}{Транспонирование растяжения}


\begin{itemize}[<+->]
    \item 
Исходный оператор $\LL : \begin{pmatrix}
  a_1 \\
  a_2 \\
\end{pmatrix} \to
\begin{pmatrix}
  2a_1 \\
  -3a_2 \\
\end{pmatrix}
$

\item 
$\langle \LL \ba , \bb \rangle = 2a_1 b_1 - 3a_2 b_2 = \langle  \ba , \LL \bb \rangle$

\item $\LL^T = \LL$

\end{itemize}

\end{frame}




\begin{frame}{Транспонирование поворота}


\begin{itemize}[<+->]
    \item 
Исходный оператор $\LL$ — поворот на плоскоти на $30^{\circ}$ против часовой стрелки.

\item $\langle \LL \ba , \bb \rangle = \langle \ba , \LL^T \bb \rangle$

\item Поворот не меняет длины. 

\item $\angle (\LL \ba , \bb ) = \angle (\ba , \LL^T \bb)$

\item $\LL^T$  — поворот на плоскоти на $30^{\circ}$ по часовой стрелке.

\item $\LL^T=\LL^{1}$

\end{itemize}

\end{frame}
    


\begin{frame}{Транспонирование дописывания нуля}


\begin{itemize}[<+->]
\item Исходный оператор $\LL: \R^2 \to \R^3$, $\ba \in \R^2$, $\bb\in \R^3$:

$\LL: \begin{pmatrix}
    a_1 \\
    a_2 \\
\end{pmatrix} \to \begin{pmatrix}
    a_1 \\
    a_2 \\
    0 \\
\end{pmatrix}$

\item $\langle \LL \ba , \bb \rangle = \langle (a_1, a_2, 0) , (b_1, b_2, b_3) \rangle =  a_1 b_1 + a_2 b_2 + 0 b_3$


\item Третья компонента $\bb$ не важна!


\item $\LL^T  : \R^3 \to \R^2$:

$\LL: \begin{pmatrix}
    b_1 \\
    b_2 \\
    b_3
\end{pmatrix} \to \begin{pmatrix}
    b_1 \\
    b_2 \\
\end{pmatrix}$


\item $\langle \ba , \LL \bb \rangle = \langle (a_1, a_2) , (b_1, b_2) \rangle =  a_1 b_1 + a_2 b_2$

\item $\LL^T$ — удаление третьей компоненты вектора.

\end{itemize}

\end{frame}
    

\begin{frame}{Транспонирование проекции}

\begin{itemize}[<+->]

    \item 
Исходный оператор $H: \R^2 \to \R^2$, проекция на прямую $x_1 + 2x_2 = 0$.

    \item Временно $\norm{\ba} = \norm{\bb} = 1$.
    
    \item  $\langle H \ba ,  \bb \rangle  =  \langle \ba ,  H\bb \rangle$
    \item $H^T = H$.
\end{itemize}

\begin{tikzpicture}[
scale=1.2,
MyPoints/.style={draw=blue,fill=white,thick},
Segments/.style={draw=blue!50!red!70,thick},
MyCircles/.style={green!50!blue!50,thin}, 
every node/.style={scale=1.2}
]
%\grid;

%\clip (-1.5,-1.5) rectangle (7.5,5.5);


%%\draw[->, >=stealth] (-1,0)--(6.5,0) node[right]{$x_1$};
%\draw[-{Latex[length=4.5mm, width=2.5mm]}, >=stealth] (0,-1)--(0,5) node[above left]{$x_2$};
%
%\draw[-{Latex[length=4.5mm, width=2.5mm]}, >=stealth] (-1,0)--(6.5,0) 
%node[right]{$x_1$};

% Feel free to change here coordinates of points A and B
\pgfmathparse{0}		\let\Xa\pgfmathresult
\pgfmathparse{0}		\let\Ya\pgfmathresult
\coordinate (A) at (\Xa,\Ya);

\pgfmathparse{5.5}		\let\Xb\pgfmathresult
\pgfmathparse{5.5}		\let\Yb\pgfmathresult
\coordinate (B) at (\Xb,\Yb);

\pgfmathparse{5}		\let\Xc\pgfmathresult
\pgfmathparse{2}		\let\Yc\pgfmathresult
\coordinate (C) at (\Xc,\Yc);

\pgfmathparse{-1}		\let\Xd\pgfmathresult
\pgfmathparse{1}		\let\Yd\pgfmathresult
\coordinate (D) at (\Xd,\Yd);

\pgfmathparse{3}		\let\Xe\pgfmathresult
\pgfmathparse{5}		\let\Ye\pgfmathresult
\coordinate (E) at (\Xe,\Ye);

\pgfmathparse{-0.75}		\let\Xf\pgfmathresult
\pgfmathparse{0.75}		\let\Yf\pgfmathresult
\coordinate (F) at (\Xf,\Yf);

\pgfmathparse{3.25}		\let\Xg\pgfmathresult
\pgfmathparse{4.75}		\let\Yg\pgfmathresult
\coordinate (G) at (\Xg,\Yg);

\pgfmathparse{4}		\let\Xh\pgfmathresult
\pgfmathparse{4}		\let\Yh\pgfmathresult
\coordinate (H) at (\Xh,\Yh);

\pgfmathparse{6}		\let\Xj\pgfmathresult
\pgfmathparse{2}		\let\Yj\pgfmathresult
\coordinate (J) at (\Xj,\Yj);

\draw[-{Latex[length=4.5mm, width=2.5mm]}, >=stealth, vecb,thick] (A)--(B) node[midway,left]{$\ba$};


\draw[-{Latex[length=4.5mm, width=2.5mm]}, >=stealth, veca,thick] (A)--(J) node[midway,below]{$\bb$};

\draw[thick] (A)--(D);
\draw[thick] (H)--(E);

\draw[thick, dashed] (J)--(E);


\draw[{Latex[length=4.5mm, width=2.5mm]}-{Latex[length=4.5mm, width=2.5mm]}, >=stealth, thick] (F)--(G) node[midway,above left]{$\langle \ba, \bb \rangle$};




\tkzMarkRightAngle[size=0.3, mark = none](A,H,J);

\tkzMarkAngle[size=1, mark = none](C,A,B);


%\node [below, darkgray] at (1,1) {$\varphi$}; 

%\node [below right, darkgray] at (-1.5,-1) {если  $\| \ba \| = 1 $, то $\langle \ba, \bb \rangle$  -- длина* проекции $\bb$ на $\ba$}; 



%\draw
%(3,-1) coordinate (a) node[right] {$a$}
%-- (0,0) coordinate (b) node[left] {b}
%-- (2,2) coordinate (c) node[above right] {c}


\end{tikzpicture}
    

\end{frame}






\begin{frame}{Ортогональный оператор}

\begin{itemize}[<+->]
\item 
Определение. Оператор $\LL: \R^n \to \R^n$ называется \alert{ортогональным}, если
\begin{itemize}
    \item оператор сохраняет длины, $\norm{\LL \ba} = \norm{\ba}$
    \item оператор сохраняет углы, $\angle(\LL\ba, \LL \bb) = \angle(\ba, \bb)$
\end{itemize}
\item Эквивалентное определение-2: 

$\langle \LL \ba, \LL \bb\rangle = \langle \ba, \bb \rangle$.

\item Эквивалентное определение-3: 

$\LL^T = \LL^{-1}$.


\end{itemize}


\end{frame}


\begin{frame}{Ортогональный оператор: примеры}


\begin{itemize}[<+->]
    \item Перестановка двух компонент вектора
    
    $\LL: \begin{pmatrix}
        a_1 \\
        a_2
    \end{pmatrix} \to
    \begin{pmatrix}
        a_2 \\
        a_1
    \end{pmatrix}$

    \item Поворот на плоскости на $30^{\circ}$ против часовой стрелки.

\end{itemize}

\end{frame}
    


\begin{frame}{Полтора доказательства}


\begin{itemize}[<+->]
    \item Определения 1 и 2 эквивалентны.
    
    \item Скалярное произведение задаёт углы и длины:

    $\norm{\ba} = \sqrt{\langle \ba, \ba\rangle} \quad \cos\angle (\ba, \bb)= \frac{\langle \ba, \bb\rangle} {\norm{\ba} \norm{\bb}}$

    \item Длина и угол задают скалярное произведение:

    $\langle \ba, \bb \rangle = \norm{\ba} \norm{\bb} \cos \angle (\ba, \bb)$
    \item Из определения 3 следует определение 1.
    
    \item $\langle \LL \ba, \bb \rangle = \langle \ba, \LL^T \bb \rangle= \langle \ba, \LL^{-1}\bb \rangle$

    \item $\langle \LL \ba, \LL \bc \rangle = \langle \ba, \bc \rangle$

\end{itemize}
    
\end{frame}

