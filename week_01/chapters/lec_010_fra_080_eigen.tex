% !TEX root = ../linal_lecture_01.tex


\begin{frame} % название фрагмента

\videotitle{Собственные векторы и собственные числа}

\end{frame}


\begin{frame}{Собственные векторы и собственные числа}

\begin{block}{Определение}
Если для оператора $\LL: \R^n \to \R^n$ найдётся такой ненулевой вектор $\bv$, 
что $\LL \bv=\lambda \cdot \bv$, где $\lambda \in \R$, то:
  \begin{itemize}
    \item вектор $\bv$ называется \alert{собственным вектором};
    \item число $\lambda$ называется \alert{собственным числом}.
  \end{itemize} 
\end{block}

\begin{center}
\begin{tikzpicture}[
scale=1.6,
MyPoints/.style={draw=blue,fill=white,thick},
Segments/.style={draw=blue!50!red!70,thick},
MyCircles/.style={green!50!blue!50,thin}, 
every node/.style={scale=1.2}
]
%\grid;
% \clip (-.5,-.5) rectangle (6.5,5.5);


%%\draw[->, >=stealth] (-1,0)--(6.5,0) node[right]{$x_1$};
%\draw[-{Latex[length=4.5mm, width=2.5mm]}, >=stealth] (0,-1)--(0,5) node[above left]{$x_2$};
%
%\draw[-{Latex[length=4.5mm, width=2.5mm]}, >=stealth] (-1,0)--(6.5,0) 
%node[right]{$x_1$};

% Feel free to change here coordinates of points A and B
\pgfmathparse{0}		\let\Xa\pgfmathresult
\pgfmathparse{0}		\let\Ya\pgfmathresult
\coordinate (A) at (\Xa,\Ya);

\pgfmathparse{5}		\let\Xb\pgfmathresult
\pgfmathparse{0}		\let\Yb\pgfmathresult
\coordinate (B) at (\Xb,\Yb);

\pgfmathparse{2.5}		\let\Xc\pgfmathresult
\pgfmathparse{5}		\let\Yc\pgfmathresult
\coordinate (C) at (\Xc,\Yc);

\pgfmathparse{1.5}		\let\Xd\pgfmathresult
\pgfmathparse{3}		\let\Yd\pgfmathresult
\coordinate (D) at (\Xd,\Yd);

\pgfmathparse{2}		\let\Xe\pgfmathresult
\pgfmathparse{1.5}		\let\Ye\pgfmathresult
\coordinate (E) at (\Xe,\Ye);



% Let I be the midpoint of [AB]
\pgfmathparse{(\Xb+\Xa)/2} \let\XI\pgfmathresult
\pgfmathparse{(\Yb+\Ya)/2} \let\YI\pgfmathresult
\coordinate (I) at (\XI,\YI);	


\draw[-{Latex[length=4.5mm, width=2.5mm]}, >=stealth, darkgray,thick] (A)--(B) node[above]{$\operatorname{L} \bb \neq \lambda \bb$};


\draw[-{Latex[length=4.5mm, width=2.5mm]}, >=stealth, vecb,thick] (A)--(E) node[midway,below]{$\bb$};


\draw[-{Latex[length=4.5mm, width=2.5mm]}, >=stealth, darkgray,thick] (A)--(C) node[above]{$\operatorname{L} \ba = \lambda \ba$};

\draw[-{Latex[length=4.5mm, width=2.5mm]}, >=stealth, veca,thick] (A)--(D) node[midway,left]{$\ba$};


\end{tikzpicture}
\end{center}    


\end{frame}



\begin{frame}{Собственные вектора: растягивание осей}

Оператор 
$\LL : \begin{pmatrix}
  a_1 \\
  a_2 \\
\end{pmatrix} \to
\begin{pmatrix}
  2a_1 \\
  -3a_2 \\
\end{pmatrix}
$.
\pause

Собственные векторы, домножаемые на $\lambda = 2$:
\[
v=\begin{pmatrix}
    x \\
    0 \\
  \end{pmatrix}
\]
\pause

Собственные векторы, домножаемые на $\lambda = -3$:
\[
  v=\begin{pmatrix}
    0 \\
    x \\
  \end{pmatrix}
\]


\end{frame}
  



\begin{frame}{Собственные вектора: перестановка $a_i$}

Оператор $\LL$ — перестановка компонент вектора:
$\LL : \begin{pmatrix}
    a_1 \\
    a_2 \\
    a_3\\
    a_4 \\
\end{pmatrix} \to 
\begin{pmatrix}
    a_1 \\
    a_4 \\
    a_3\\
    a_2 \\
\end{pmatrix}$.
\pause

Собственные векторы, домножаемые на $\lambda = 1$:

Cодержат одинаковые числа на переставляемых местах,
\[
v=\begin{pmatrix}
    a_1 \\
    x \\
    a_3 \\
    x \\
\end{pmatrix}
\]

\end{frame}



\begin{frame}{Собственные векторы: поворот}

Не у каждого линейного оператора есть собственные векторы!
\pause
  
Исходный оператор $\Rot: \R^2 \to \R^2$, поворот на $30^{\circ}$ против часовой стрелки.


\begin{center}
\begin{tikzpicture}[
scale=1.4,
MyPoints/.style={draw=blue,fill=white,thick},
Segments/.style={draw=blue!50!red!70,thick},
MyCircles/.style={blue!50,dashed}, 
every node/.style={scale=1.2}
]
%\grid;
%\draw[color=gray,step=1.0,dotted] (-7.1,-2.1) grid (7.6,6.1);


%%\draw[->, >=stealth] (-1,0)--(6.5,0) node[right]{$x_1$};
%\draw[-{Latex[length=4.5mm, width=2.5mm]}, >=stealth] (0,-1)--(0,5) node[above left]{$x_2$};
%
%\draw[-{Latex[length=4.5mm, width=2.5mm]}, >=stealth] (-1,0)--(6.5,0) 
%node[right]{$x_1$};

% Feel free to change here coordinates of points A and B
\pgfmathparse{0}		\let\Xa\pgfmathresult
\pgfmathparse{0}		\let\Ya\pgfmathresult
\coordinate (A) at (\Xa,\Ya);

\pgfmathparse{1.5}		\let\Xb\pgfmathresult
\pgfmathparse{3}		\let\Yb\pgfmathresult
\coordinate (B) at (\Xb,\Yb);

\pgfmathparse{3}		\let\Xc\pgfmathresult
\pgfmathparse{1}		\let\Yc\pgfmathresult
\coordinate (C) at (\Xc,\Yc);

\pgfmathparse{3}		\let\Xd\pgfmathresult
\pgfmathparse{4}		\let\Yd\pgfmathresult
\coordinate (D) at (\Xd,\Yd);

\pgfmathparse{75}		\let\angle\pgfmathresult;
\pgfmathparse{sqrt(10)}		\let\rad\pgfmathresult;


\pgfmathparse{\Xb*cos(\angle)  - \Yb*sin(\angle)}		\let\Xe\pgfmathresult
\pgfmathparse{\Xb*sin(\angle)  + \Yb*cos(\angle)}		\let\Ye\pgfmathresult
\coordinate (E) at (\Xe,\Ye);

\pgfmathparse{\Xd*cos(\angle)  - \Yd*sin(\angle)}		\let\Xf\pgfmathresult
\pgfmathparse{\Xd*sin(\angle)  + \Yd*cos(\angle)}		\let\Yf\pgfmathresult
\coordinate (F) at (\Xf,\Yf);


% Let I be the midpoint of [AB]
\pgfmathparse{(\Xb+\Xa)/2} \let\XI\pgfmathresult
\pgfmathparse{(\Yb+\Ya)/2} \let\YI\pgfmathresult
\coordinate (I) at (\XI,\YI);	


\draw[-{Latex[length=4.5mm, width=1.5mm]}, >=stealth, veca,thick] (A)--(B) node[midway,right]{$\bv$};


\draw[-{Latex[length=4.5mm, width=1.5mm]}, >=stealth, vecb,thick] (A)--(E) node[midway,below left]{$\bb$};

\tkzMarkAngle[size=1, mark = none, arrows=->,line width=1.5pt, mkcolor=red ](B,A,E);

\draw[dashed] (B) to[bend right] (E);

\node [above right] at (-0.5, 1) {$\operatorname{R}$}; 

\node [above right] at (-2, 4) {$\operatorname{R} \bv = \bb$}; 


\end{tikzpicture}
\end{center}    
  
\pause


Ни собственных векторов, ни чисел нет!



\end{frame}




\begin{frame}{Собственные вектора: проекция}

Оператор $\LL$ — проекция на прямую $\ell$.
\pause

Собственные векторы, домножаемые на $\lambda = 1$:

Любой вектор $v$, лежащий на прямой $\ell$.
\pause

Собственные векторы, домножаемые на  $\lambda = 0$:

Любой вектор $v$, ортогональный прямой $\ell$.


\end{frame}
    

\begin{frame}{Резюме}


\begin{itemize}[<+->]
\item Вектор — столбец чисел.
\item Скалярное произведение «знает» о длине и угле.
\item Линейный оператор — «уважает» сложение векторов.
\item Примеры: поворот, проекция, перестановка компонент, растягивание осей.
\item Обращение и транспонирование. 
\item Собственные векторы растягиваются в собственное число раз.  
\item Бонус: как выиграть в Ним?
\end{itemize}
\pause
\alert{Следующая лекция:} запись оператора с помощью матрицы.



\end{frame}








