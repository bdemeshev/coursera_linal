% !TEX root = ../linal_lecture_04.tex

\begin{frame} % название фрагмента

\videotitle{Собственные числа и векторы}

\end{frame}



\begin{frame}{Краткий план:}
  \begin{itemize}[<+->]
    \item Собственные числа и собственные векторы матрицы.
    \item Характеристический многочлен.
    \item Алгебраическая кратность.
  \end{itemize}

\end{frame}


\begin{frame}{От оператора к матрице}

\begin{block}{Определение}
Если для оператора $\LL: \R^n \to \R^n$ найдётся такой ненулевой вектор $\bv$, 
что $\LL \bv=\lambda \cdot \bv$, где $\lambda \in \R$, то:
  \begin{itemize}
    \item вектор $\bv$ называется \alert{собственным вектором};
    \item число $\lambda$ называется \alert{собственным числом}.
  \end{itemize} 
\end{block}

\begin{center}
\begin{tikzpicture}[
scale=1.6,
MyPoints/.style={draw=blue,fill=white,thick},
Segments/.style={draw=blue!50!red!70,thick},
MyCircles/.style={green!50!blue!50,thin}, 
every node/.style={scale=1.2}
]
%\grid;
% \clip (-.5,-.5) rectangle (6.5,5.5);


%%\draw[->, >=stealth] (-1,0)--(6.5,0) node[right]{$x_1$};
%\draw[-{Latex[length=4.5mm, width=2.5mm]}, >=stealth] (0,-1)--(0,5) node[above left]{$x_2$};
%
%\draw[-{Latex[length=4.5mm, width=2.5mm]}, >=stealth] (-1,0)--(6.5,0) 
%node[right]{$x_1$};

% Feel free to change here coordinates of points A and B
\pgfmathparse{0}		\let\Xa\pgfmathresult
\pgfmathparse{0}		\let\Ya\pgfmathresult
\coordinate (A) at (\Xa,\Ya);

\pgfmathparse{5}		\let\Xb\pgfmathresult
\pgfmathparse{0}		\let\Yb\pgfmathresult
\coordinate (B) at (\Xb,\Yb);

\pgfmathparse{2.5}		\let\Xc\pgfmathresult
\pgfmathparse{5}		\let\Yc\pgfmathresult
\coordinate (C) at (\Xc,\Yc);

\pgfmathparse{1.5}		\let\Xd\pgfmathresult
\pgfmathparse{3}		\let\Yd\pgfmathresult
\coordinate (D) at (\Xd,\Yd);

\pgfmathparse{2}		\let\Xe\pgfmathresult
\pgfmathparse{1.5}		\let\Ye\pgfmathresult
\coordinate (E) at (\Xe,\Ye);



% Let I be the midpoint of [AB]
\pgfmathparse{(\Xb+\Xa)/2} \let\XI\pgfmathresult
\pgfmathparse{(\Yb+\Ya)/2} \let\YI\pgfmathresult
\coordinate (I) at (\XI,\YI);	


\draw[-{Latex[length=4.5mm, width=2.5mm]}, >=stealth, darkgray,thick] (A)--(B) node[above]{$\operatorname{L} \bb \neq \lambda \bb$};


\draw[-{Latex[length=4.5mm, width=2.5mm]}, >=stealth, vecb,thick] (A)--(E) node[midway,below]{$\bb$};


\draw[-{Latex[length=4.5mm, width=2.5mm]}, >=stealth, darkgray,thick] (A)--(C) node[above]{$\operatorname{L} \ba = \lambda \ba$};

\draw[-{Latex[length=4.5mm, width=2.5mm]}, >=stealth, veca,thick] (A)--(D) node[midway,left]{$\ba$};


\end{tikzpicture}
\end{center}    


\end{frame}
        



\begin{frame}
    \frametitle{Собственные числа и векторы матрицы}

    \begin{block}{Определение}
        \alert{Собственными числами} и \alert{собственными векторами} матрицы размера $n\times n$ называются собственные числа и векторы 
        соответствующего линейного оператора.
    \end{block}
    \pause

    Для абстрактного векторного пространства $V$ матрица $\LL_{\be \be}$ линейного оператора $\LL: V\to V$ зависит от выбора базиса $\be$.
    При этом выбор базиса $\be$ никак не влияет на собственные числа и собственные векторы. 
\end{frame}


\begin{frame}
    \frametitle{Количество собственных векторов}

    Из уравнения $\LL \bv = \lambda \bv$ находим вектор $\bv$ и число $\lambda$.

    \pause

    Если найдётся один собственный вектор $\bv \neq \bzero$, то любой вектор $\bv' = c \cdot \bv$ также будет собственным:
    \pause
    \[
    \LL \bv' = \LL c \bv = c \LL \bv = c \lambda \bv = \lambda \bv'.
    \]

    \pause
    Система уравнений $\LL \bv = \lambda \bv$ должна иметь бесконечное количество решений!


\end{frame}


\begin{frame}
    \frametitle{Как найти собственные числа?}

    Перепишем систему $\LL \bv = \lambda \bv$ в виде $(\LL - \lambda \Id) \bv = \bzero$.
    
    \pause

    Система имеет бесконечное количество решений, если и только если $\det (\LL - \lambda \Id) = 0$.

    \pause

    \begin{block}{Алгоритм}
        \begin{enumerate}
            \item Из уравнения $\det (\LL - \lambda \Id) = 0$ находим собственные числа $\lambda_1$, \ldots, $\lambda_k$. \pause
            \item Для каждого $\lambda_i$ решаем систему $(\LL - \lambda_i \Id) \bv = \bzero$ относительно $\bv$, то есть находим все собственные векторы. 
        \end{enumerate}
    \end{block}

    
\end{frame}


 \begin{frame}
     \frametitle{Характеристический многочлен}

     \begin{block}{Определение}
         Многочлен $\charp_{\LL}(\lambda) = \det (\LL - \lambda \Id)$ называется
         \alert{характеристическим многочленом} линейного оператора $\LL$.
     \end{block}

     \pause

     Характеристическим многочленом матрицы называется характеристический многочлен соответствующего линейного оператора.
    
 \end{frame}


 \begin{frame}{Характеристический многочлен: пример}

    Рассмотрим матрицу $A = \begin{pmatrix}
        4 & 6 & 0 \\
        6 & 4 & 0 \\
        0 & 0 & 7 \\
    \end{pmatrix}.$

    \pause
    \[
    \charp_A(\lambda) = \det (A - \lambda \Id) = \begin{vmatrix}
4-\lambda & 6 & 0 \\
6 & 4-\lambda & 0 \\
0 & 0 & 7-\lambda \\        
    \end{vmatrix} = \pause
    \]
    \[ = (7-\lambda) \begin{vmatrix}
4-\lambda & 6  \\
6 & 4-\lambda \\
\end{vmatrix} = (7-\lambda)((4-\lambda)^2 - 36) = \pause
\]
\[
 = -(\lambda - 7)(\lambda + 2)(\lambda - 10) = 
 -\lambda^3  + 15\lambda^2   -36 \lambda  - 140  
    \]

 \end{frame}


 \begin{frame}
     \frametitle{Характеристический многочлен}

     По характеристическому многочлену можно найти: \pause

     \begin{enumerate}
         \item Собственные числа $A$ из уравнения $\charp_{A}(\lambda) = 0$.
         \[
            \charp_{A}(\lambda)  = -(\lambda - 7)(\lambda + 2)(\lambda - 10)    
         \]
         \[
         \lambda_1 = 7, \; \lambda_2 = -2, \; \lambda_3 = 10.    
         \]
         
         \pause
         \item Определитель $A$ из равенства $\charp_{A}(0) = \det (A - 0\cdot \Id)$.
         \[
            \charp_{A}(\lambda) =-\lambda^3  + 15\lambda^2   -36 \lambda  - 140 
         \]
         \[
         \det A = \charp_A(0)=-140.
         \]

     \end{enumerate}
 
 
 \end{frame}


 \begin{frame}
     \frametitle{Алгебраическая кратность}

     \begin{block}{Утверждение}
    По основной теореме алгебры любой многочлен $f$ с действительными коэффициентами можно единственным образом представить в виде:
    \[
    f(x) = (x-x_1)^{k_1} \cdot \ldots  \cdot (x-x_p)^{k_p} g(x),
    \]
    где $x_1$, \ldots, $x_p \in \R$ — различные корни многочлена $f$, а многочлен $g$ действительных корней не имеет. 
         
     \end{block}
 
     \pause
     \begin{block}{Определение}
        Число $k_i$ называется \alert{алгебраической кратностью} корня $x_i$.
     \end{block}
 
 \end{frame}


 \begin{frame}
     \frametitle{Алгебраическая кратность: пример}

     Если $\charp_A(\lambda) = -(\lambda - 7)^2(\lambda + 3)$, то 
     собственное число $\lambda = 7$ имеет алгебраическую кратность $2$, 
     а собственное число $\lambda = -3$ имеет алгебраическую кратность $1$.
 
     \pause
     Если $\LL : \R^n \to \R^n$, то сумма алгебраических кратностей $k_i$ 
     действительных собственных чисел $\lambda_i \in \R$
     не превосходит $n$:
    \[
    \sum_{i=1}^p k_i \leq n.
    \]

 \end{frame}



% \begin{frame}
% \frametitle{Немного о комплексных числах}


% \end{frame}


% \begin{frame}
%     \frametitle{Немного о комплексных числах}

%     \begin{block}{Утверждение}
%     По основной теореме алгебры любой многочлен $f$ с действительными коэффициентами можно единственным образом представить в виде:
%     \[
%     f(x) =a (x-x_1)^{k_1}\ldots (x-x_p)^{k_p},
%     \]
%     где $x_1$, \ldots, $x_p \in \mathbb{C}$ — различные корни многочлена $f$. 

%      \end{block}
%     \pause
%     Любой оператор $\LL: \R^n \to \R^n$ имеет собственные числа, если допустить, что $\lambda \in \mathbb{C}$.

%     \pause
% Если $\LL : \R^n \to \R^n$, то сумма алгебраических кратностей $k_i$ 
%  собственных чисел $\lambda_i \in \mathbb{C}$
%  равна $n$, $\sum_{i=1}^p k_i = n$.

    

% \end{frame}

 
\begin{frame}
    \frametitle{Теорема Гамильтона-Кэли}

    \begin{block}{Утверждение}
        Если подставить матрицу $A$ в характеристический многочлен $\charp_A(\lambda)$, то получится матрица из нулей,
        \[
        \charp_A(A) = \bzero;     
        \]
    \end{block}

%    \pause

 %   Эта теорема позволяет, например, снижать степень матрицы. 

    \pause
    \vspace{10pt}
    Пример. Если $\charp_A(\lambda) = \lambda^2 - 3\lambda + 8$, то $A^2 - 3A + 8\Id = \bzero$ и 
    $A^2 = 3A - 8\Id$. 
    

\end{frame}