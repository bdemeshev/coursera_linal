% !TEX root = ../linal_lecture_04.tex

\begin{frame} % название фрагмента

\videotitle{След матрицы}

\end{frame}



\begin{frame}{Краткий план:}
  \begin{itemize}[<+->] 
    \item Сумма диагональных элементов.
    \item Свойства следа.
  \end{itemize}

\end{frame}


% \begin{frame}{Комплексные числа как операторы}

% \begin{block}{Определение}
%     Множество $\CC$ линейных операторов на плоскости, включающее повороты, 
%     растягивания вектора в произвольное количество раз и композиции этих двух действий,
%     называется множеством \alert{комплексных чисел}.
% \end{block}

% \pause 
% Поворот на угол $\phi$ описывается матрицей $\begin{pmatrix}
%     \cos \phi & -\sin \phi \\
%     \sin \phi & \cos \phi \\
% \end{pmatrix}$.

% \pause
% Растяжение в $r$ раз описывается матрицей $\begin{pmatrix}
%     r & 0 \\
%     0 & r \\
% \end{pmatrix}.$

% \end{frame}
        


% \begin{frame}{Комплексные числа как матрицы}

% \begin{block}{Эквивалентное пределение}
%     Множество $\CC$ матриц вида $\begin{pmatrix}
%         r\cos \phi & -r\sin \phi \\
%         r\sin \phi & r\cos \phi \\
%     \end{pmatrix}$ называется множеством \alert{комплексных чисел}.
% \end{block}

% \pause 
% Все матрицы в $\CC$ имеют вид $\begin{pmatrix}
%     a  & -b  \\
%     b  & a  \\
% \end{pmatrix}$.

% \pause 
% Для краткости вместо матрицы 
% $\begin{pmatrix}
%     a  & -b  \\
%     b  & a  \\
% \end{pmatrix}$ пишут $a+bi$.

% \end{frame}


% \begin{frame}
%     \frametitle{Комплексные числа как образ $\be_1$}

    
%     \[
%         \begin{pmatrix}
%         a  & -b  \\
%         b  & a  \\
%     \end{pmatrix} \cdot \begin{pmatrix}
%         1 \\
%         0 
%     \end{pmatrix} = 
%     \begin{pmatrix}
%         a \\
%         b 
%     \end{pmatrix}
%     \]


% \end{frame}

    
% \begin{frame}{Никакой мистики!}

    
%     Вращение на $90^{\circ}$, $\begin{pmatrix}
%         0 & -1 \\
%         1 & 0
%     \end{pmatrix} \sim i$.\pause

%     Растягивание в 7 раз, $\begin{pmatrix}
%         7 & 0 \\
%         0 & 7
%     \end{pmatrix} \sim 7$.

%     Растягивание в $\sqrt{2}$ раз и 
%     вращение на $45^\circ$,
%     \[
%     \begin{pmatrix}
%         \sqrt{2}\cos 45^{\circ} & -\sqrt{2}\sin 45^{\circ} \\
%         \sqrt{2}\sin 45^{\circ} & \sqrt{2}\cos 45^{\circ}
%     \end{pmatrix} = \begin{pmatrix}
%         1 & -1 \\
%         1 & 1
%     \end{pmatrix}  \sim 1 + i 
%     \]
% \end{frame}


% \begin{frame}{Никакой мистики!}

%     \begin{block}{Чёрная магия}
%         \[
%         i \cdot i \sim \begin{pmatrix}
%             0 & -1 \\
%             1 & 0
%         \end{pmatrix} \cdot \begin{pmatrix}
%             0 & -1 \\
%             1 & 0
%         \end{pmatrix} = \begin{pmatrix}
%             -1 & 0 \\
%             0 & -1
%         \end{pmatrix} \sim -1
%     \]  
%     \end{block}
%     \pause
%     \begin{block}{с разоблачением}
%             Если повернуть на $90^{\circ}$, а затем повернуть ещё на $90^{\circ}$, то
%             развернёшься в обратную сторону.
%         \end{block}

% \end{frame}


% \begin{frame}{Комплексные числа как $a+bi$}

%     \begin{block}{}
%         Множество $\CC$ вида
%         \[
%         \{ a+ bi \mid a, b \in \R \}    
%         \]
%         с естественным сложением и умножением по правилу $i^2 = -1$ называется множеством \alert{комплексных чисел}.
%     \end{block}



% \end{frame}



% \begin{frame}
%     \frametitle{Основная теорема алгебры}
%     \begin{block}{Утверждение}
%         Любой многочлен $f(z)$ степени $n$ имеет ровно $n$ корней,
%         если считать корни $z\in \CC$ с учётом алгебраической кратности.
%     \end{block}

    

% \end{frame}




% \begin{frame}
%     \frametitle{След линейного оператора}

%     \begin{block}{Определение}
%         \alert{Следом} линейного оператора $\LL:\R^n\to\R^n$ называют сумму всех его комплексных собственных чисел,
%         \[
%         \trace \LL = \lambda_1 + \lambda_2 + \ldots + \lambda_n.
%         \]
%     \end{block}

%     \pause
%     Пример. Если $\charp_{A}(\lambda) = -(\lambda-1)(\lambda-5)^2$, то $\trace A = 1 + 5 + 5 = 11$.

%     \pause
%     Пример. Если $\charp_{A}(\lambda) = -(\lambda-1)(\lambda-2+3i)(\lambda-2-3i)$, то $\trace A = 1 + (2-3i) + (2+3i) = 5$.


% \end{frame}






% \begin{frame}
%     \frametitle{След матрицы}
%     \begin{block}{Определение}
%         Следом матрицы $\LL$ размера $n\times n$ называют след соответствующего линейного оператора. 
%     \end{block}
    
%     \pause
%     \begin{block}{Утверждение}
%         След матрицы $\LL$ равен сумме её диагональных элементов. 
%         \[
%             \trace \LL = \ell_{11} + \ell_{22} + \ldots + \ell_{nn}
%         \]
%     \end{block}

%     \pause
%     Пример. $\trace \begin{pmatrix}
%         4 & 6 \\
%         9 & 1
%     \end{pmatrix} = 4 + 1 = 5$.

% \end{frame}


\begin{frame}
    \begin{block}{Определение}
        \alert{Следом квадратной матрицы} $\LL$ называют сумму её диагональных элементов. 
        \[
            \trace \LL = \ell_{11} + \ell_{22} + \ldots + \ell_{nn}
        \]
    \end{block}

    \pause
    Пример. $\trace \begin{pmatrix}
        4 & 6 \\
        9 & 1
    \end{pmatrix} = 4 + 1 = 5$.

\end{frame}





\begin{frame}
    \frametitle{Основное свойство следа}

    \begin{block}{Утверждение}
        Если матрицы $A$ и $B$ имеют размер $n\times k$, то
        \[
        \trace A^T B = \sum_{ij} a_{ij} b_{ij} = \trace B^T A
        \]

    \end{block}
    \pause

    Пример. $A = \begin{pmatrix}
        a_1 & a_2 \\
        a_3 & a_4 \\ 
    \end{pmatrix},  \;
    B = \begin{pmatrix}
        b_1 & b_2 \\
        b_3 & b_4 \\ 
    \end{pmatrix}$.

    \[
    \trace A^T B = a_1 b_1 + a_2 b_2 + a_3 b_3 + a_4 b_4    
    \]

\end{frame}


\begin{frame}
    \frametitle{Основное свойство следа}

\begin{block}{Утверждение}
    Если матрицы $A$ и $B$ имеют размер $n\times k$, то
    \[
    \trace A^T B = \sum_{ij} a_{ij} b_{ij} = \trace B^T A
    \]

\end{block}
\pause

    \begin{block}{Доказательство}
        \[
        \trace A^T B = \sum_i \langle \row_i A^T, \col_i B \rangle =
        \]
        \[ 
         = \sum_i \langle \col_i A, \col_i B \rangle = \sum_{ij} a_{ij} b_{ij}
        \]
    \end{block}
    

\end{frame}


\begin{frame}
    \frametitle{И ещё немного свойств}

    \[
    \trace AB = \trace BA    
    \]
    \pause

    След — линейный оператор, превращающий матрицы размера $n\times n$ в числа!
    \pause

    \[
    \trace \lambda A = \lambda \trace A    
    \]


    \[
    \trace (A+B) = \trace A + \trace B
    \]


\end{frame}


\begin{frame}
    \frametitle{Зачем нужен след?}

    \pause
    Элегантно позволяет записывать сложные выражения.
    
    \[
    \sum_{ij} a_{ij}^2 = \trace A^T A    
    \]

    \pause
    
    Упрощает теоретические выкладки. 

\end{frame}


