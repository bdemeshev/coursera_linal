% !TEX root = ../linal_lecture_04.tex

\begin{frame} % название фрагмента

\videotitle{Вокруг собственных чисел}

\end{frame}



\begin{frame}{Краткий план:}
  \begin{itemize}[<+->]
    \item Реинкарнация теоремы Виета. 
    \item Обратимость и собственные числа.
    \item Собственные числа проектора.
  \end{itemize}

\end{frame}



\begin{frame}
    \frametitle{Характеристический многочлен и след}

Рассмотрим пример характеристического многочлена:
\[
\charp_B(\lambda) = \det(B -\lambda\Id) = \begin{vmatrix}
    \blue{4 - \lambda} & 5 & 6 \\
    1 & \blue{2 - \lambda} & 2 \\
    4 & 3 & \blue{7 -\lambda} \\
\end{vmatrix} =   \pause
\]
\[
= (4-\lambda)(2-\lambda)(7-\lambda) + \ldots = -\lambda^3 + \lambda^2 ( 4 + 2 + 7) + \ldots    
\]
\pause
\[
= -\lambda^3 + \trace A \cdot \lambda^2 + \ldots    
\]

\end{frame}



\begin{frame}
    \frametitle{Характеристический многочлен и след}


    \begin{block}{Утверждение}
        В характеристическом многочлене $\charp_A(\lambda)$ матрицы $A$ размера $n\times n$ 
        перед $\lambda^{n-1}$ стоит $(-1)^{n-1} \trace A$:
        \[
            \charp_A(\lambda) = (-1)^n \lambda^n  + (-1)^{n-1}\trace A \lambda^{n-1} + \ldots 
        \]
        
    \end{block}
    \pause

    Пример. 
    \[
        A= \begin{pmatrix}
            4 & 5 \\
            1 & 2 \\
        \end{pmatrix}, \; \charp_A(\lambda) = \lambda^2 - 6 \lambda + 3
    \]
    \pause
    Пример.
\[
    B= \begin{pmatrix}
        4 & 5 & 6 \\
        1 & 2 & 2 \\
        4 & 3 & 7 \\
    \end{pmatrix}, \; \charp_B(\lambda) = -\lambda^3 + 13 \lambda^2 + \ldots 
\]


\end{frame}



\begin{frame}{Реинкарнация теоремы Виета}
\begin{block}{Утверждение}
    Если у матрицы $A$ размера $n\times n$ ровно $n$ действительных собственных чисел $\lambda_1$, 
    $\lambda_2$, \ldots, $\lambda_n$, то \pause
    \[
    \charp_A(\lambda) = (\lambda_1 - \lambda) (\lambda_2 - \lambda) \cdot \ldots \cdot (\lambda_n - \lambda)
    \]
\end{block}
\pause
\begin{block}{Следствия}
\[
\det A = \prod_{i=1}^n \lambda_i; \pause
\]
\[
\trace A = \sum_{i=1}^n \lambda_i; 
\]    
\end{block}


\end{frame}

\begin{frame}
    \frametitle{Пополним критерий вырожденности!}

    Матрица $A$ размера $n\times n$ называется \alert{вырожденной}, если:

    \begin{enumerate}
        \item $\det A = 0$; 
        \item Система $A \bx = \bzero$ имеет бесконечное количество решений; 
        \item Система $A \bx = \bb$ имеет ноль или бесконечное количество решений; 
        \item $\rank A < n$; 
        \item Столбцы $A$ линейно зависимы; 
        \item Строки $A$ линейно зависимы; 
        \item $A^{-1}$ не существует; 
        \item \alert{У матрицы $A$ есть $\lambda=0$.}
    \end{enumerate}


\end{frame}



\begin{frame}
    \frametitle{Собственные числа проектора}

    Оператор $\HH: \R^n \to \R^n$ проецирует $\R^n$ на некоторую линейную оболочку $M$. \pause
    
    \begin{block}{Утверждение}
        Собственные числа проектора $H$ равны $0$ или $1$. \pause
        
        Собственными векторами c $\lambda=0$ будут векторы, ортогональные $M$. \pause

        Собственными векторами c $\lambda=1$ будут векторы из  $M$. \pause

        У проектора ровно $n$ линейно независимых собственных векторов. 
    \end{block}

    

\end{frame}


\begin{frame}
\frametitle{Ранг и след проектора}

Оператор $\HH: \R^n \to \R^n$ проецирует $\R^n$ на некоторую линейную оболочку $M$. \pause

    Ранг проектора — число элементов в базисе $M$. \pause

    След проектора — кратность собственного числа $\lambda=1$:
    \[
    \trace H = \lambda_1 + \lambda_2 + \ldots + \lambda_n    
    \]
    \pause

    \begin{block}{Утверждение}
        Для проектора $\HH$ след и ранг равны размерности множества, на которое проецирует $\HH$,
        \[
        \rank \HH = \trace \HH.    
        \]
    \end{block}


\end{frame}